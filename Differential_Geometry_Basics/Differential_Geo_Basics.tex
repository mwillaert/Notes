\documentclass[a4paper,12pt,parskip=half*,chapterprefix=true,numbers=noendperiod]{scrreprt}

% LANGUAGES AND SYMBOLS
\usepackage[utf8]{inputenc}
\usepackage[T1]{fontenc}
\usepackage[english]{babel}
\usepackage{lmodern}
\usepackage{xparse}

% EVERYDAY PACKAGES
\usepackage{lipsum}
\usepackage[shortlabels]{enumitem}
\usepackage{verbatim}
\usepackage[normalem]{ulem}
\usepackage[dvipsnames]{xcolor}
\usepackage{csquotes}
\usepackage{pdfpages}
\usepackage{easy-todo}
\usepackage{setspace}
%\usepackage[all]{nowidow}

\usepackage{graphicx}
\usepackage{caption}
\usepackage{subcaption}

% MATHEMATICS
\usepackage{mathtools}
\usepackage{stmaryrd}
\usepackage{amsthm, thmtools}
\usepackage{framed}
\usepackage{nameref}
\usepackage[colorlinks,menucolor=blue,linkcolor=blue, citecolor=blue, urlcolor=blue]{hyperref} 
\usepackage[capitalise]{cleveref}% if problems, load cleveref last
\crefname{subsection}{Subsection}{Subsections}
\Crefname{subsection}{Subsection}{Subsections}
%\usepackage[retainorgcmds]{IEEEtrantools}
\usepackage{amssymb}
\usepackage[mathscr]{euscript}
\usepackage{esint}
\usepackage{esvect}
\usepackage{relsize}

\usepackage{tikz}
\usepackage{tikz-qtree}
\usepackage[framemethod=tikz]{mdframed}
\usetikzlibrary{cd}
\usetikzlibrary{matrix}
\usetikzlibrary{backgrounds}

%Packages for references, Indices

\usepackage{imakeidx}[intoc]
\usepackage{biblatex}
\addbibresource{../refs.bib}

%Additionnal Packages

\usepackage{float}
\usepackage{cancel}

%Macors

\newcommand{\opname}{\operatorname}

%Theorems

\newtheorem{theorem}{Theorem}[section]
\newtheorem{proposition}{Proposition}[section]
\newtheorem{lemma}{Lemma}[section]
\newtheorem{corollary}{Corollary}[section]

\theoremstyle{definition}
\newtheorem{definition}{Definition}[section]
\newtheorem{example}{Example}[section]
\newtheorem{notation}{Notation}[section]

\theoremstyle{remark}
\newtheorem*{remark}{Remark}

\title{Basics of Differential Geometry}
\author{Maxime Willaert}
\begin{document}

\maketitle

\tableofcontents

\part{Topology}

\chapter{Intro}

Main references are \cite{Munkres:Top,Lee:IntTopMan}.

\section{Things left to learn}

The different sets of axioms one can use to define a topological space, as in \cite{Wiki:AxiomTop}. A topological space is most commonly defined by specifying its open sets. But one can also define a topology in the following ways:
\begin{enumerate}[(i)]
	\item By specifying its neighborhoods or closed sets.
	\item By specifying the interior, closure, exterior, boundary or 'derived set' operators.
	\item Through nets or filters (which are equivalent).
\end{enumerate}
The notions of \textbf{nets} and \textbf{filters}, their equivalence and their relationship with the notion of a topology, should be explored (see \cite{Wiki:FiltersTop}). The different notions of convergence that can be defined in a topology, and the degree to which they determine this topology is also an interesting question \cite{Wiki:SequentialSpace,Wiki:ConvergenceSpace}.

General topology (more set-theoretic than algebraic, and not focused on finite-dimensional topological manifolds) as in \cite{Munkres:Top}.

\chapter{Topological spaces}

\begin{definition}[Topology]
A \textbf{topology} on a set $X$ is a collection $\tau$ of subsets of $X$ such that
\begin{enumerate}[(i)]
	\item $\tau$ contains $\emptyset$ and $X$;
	\item The union of the elements of any subset of $\tau$ is again in $\tau$;
	\item The intersection of the elements of any finite subset of $\tau$ is again in $\tau$.
\end{enumerate}
A \textbf{topological space} is a pair $(X,\tau)$ consisting of a set $X$ together with a topology $\tau$ on $X$. The elements of $\tau$ are called the \textbf{open sets} of $(X,\tau)$.

Given two topological spaces $(X,\tau)$, $(X',\tau')$, a map $f:(X,\tau)\to(X',\tau')$ is said to be \textbf{continuous} if for any $U\in\tau'$, $f^{-1}(U)\in\tau$. In most instances we can omit $\tau$ when referring to the topological space $(X,\tau)$ with no risk of confusion.
\end{definition}

\begin{definition}[Neighborhoods]
	Let $X$ be a topological space.
	\begin{enumerate}[(i)]
		\item Let $K$ be a subset of $X$, another subset $N$ of $X$ is said to be a \textbf{neighborhood} of $K$ if there exists an open subset $U$ of $X$ such that $K\subseteq U\subseteq N$. An \textbf{open neighborhood} of $K$ is an open subset of $X$ that contains $K$.
		\item Let $x$ be a point of $X$. An (open) neighborhood of $x$ is an (open) neighborhood of the singleton $\{x\}$.
	\end{enumerate}	
\end{definition}

\begin{definition}[Closed subsets]
A subset $F$ of $X$ is said to be \textbf{closed} if its complement $X-F$ is open.
\end{definition}

\begin{proposition}
Let $X$ be a topological space.
\begin{enumerate}[(i)]
	\item $\emptyset$ and $X$ are closed.
	\item Any intersection of closed subsets of $X$ is closed.
	\item A finite unions of closed subsets of $X$ is closed.
\end{enumerate}
\end{proposition}

\begin{proposition}
	A map between topological spaces is continuous if and only the preimage of any closed subset is closed.
\end{proposition}
\begin{proof}
	This is because for any map $f:X\to Y$ and any subset $A\subseteq Y$, $f^{-1}(Y-A)=X-f^{-1}(A)$.
\end{proof}

\begin{definition}[Closure and interior]
	Let $A$ be a subset of a topological space $X$.
	\begin{enumerate}[(i)]
		\item The \textbf{closure} of $A$, denoted $\bar{A}$ is the smallest closed subset containing $A$.
		\begin{equation*}
			\bar{A}:=\bigcap\{F\subseteq X|S\text{is closed and }A\subseteq F\}.
		\end{equation*}
		\item The \textbf{interior} of $A$, denoted $\text{Int}(A)$ is the largest open subset contained in $A$.
		\begin{equation*}
			\text{Int}(A):=\bigcup\{U\subseteq X|U\text{is open and }U\subseteq A\}.
		\end{equation*}
		\item The \textbf{exterior} of $A$, denoted by $\text{Ext}(A)$, is defined to by $\text{Ext}(A):=X-\bar{A}$, it is the complement of the closure, that is the largest open that does not overlap with $A$.
		\item The \textbf{boundary} of $A$, denoted by $\partial A$ is defined by $\partial A:=\bar{A}-\text{Int}(A)$.
	\end{enumerate}
\end{definition}

\begin{proposition}
Let $A$ be a subset of an topological space $X$.
\begin{enumerate}[(i)]
	\item A point is in $\text{Int}(A)$ if and only if it has a neighborhood contained in $A$.
	\item A point is in $\text{Ext}(A)$ if and only if it has a neighborhood contained in $X-A$.
	\item A point is in $\partial A$ if and only if any neighborhood of it contains both a point of $A$ and a point of $X-A$.
	\item A point is in $\bar{A}$ if and only if any neighborhood of it contains a point of $A$.
	\item The following are equivalent:
	\begin{itemize}
		\item $A$ is open.
		\item $A=\text{Int}(A)$.
		\item $A$ contains none of its boundary points (hence the 'open terminology').
		\item Any point of $A$ has a neighborhood contained in $A$.
	\end{itemize}
	\item The following are equivalent:
	\begin{itemize}
		\item $A$ is closed.
		\item $A=\bar{A}$.
		\item $A$ contains all of its boundary points (hence the 'closed' terminology).
		\item Any point of $X-A$ has a neighborhood contained in $X-A$.
	\end{itemize}
\end{enumerate}
\end{proposition}

\begin{definition}[Limit and isolated points]
Let $A$ be a subset of a topological space $X$.
\begin{enumerate}[(i)]
	\item A point $p\in X$ (not necessarily in $A$) is a \textbf{limit point} of $A$ if any neighborhood of $p$ contains a point of $A$ other than $p$. Limit points are also called \textbf{cluster points} and \textbf{accumulation} points.
	\item A point $p\in A$ is \textbf{isolated in $A$} if $p$ has a neighborhood $N$ such that $N\cap A=\{p\}$. 
\end{enumerate}
Observe that any point of $A$ is either isolated in $A$ or a limit point of $A$.
\end{definition}

\begin{proposition}
A set is closed if and only if it contains all of its limit points.
\end{proposition}

\begin{definition}[(Nowhere) dense sets]
A subset $A$ of a topological space $X$ is said to be \textbf{dense} in $X$ if $\bar{A}=X$. Given a subset $S$ of $X$, $A$ is said to be dense in $S$ if $A\cap S$ is dense in $S$ (with the subset topology). It is said to be \textbf{nowhere dense} or \textbf{rare} in $X$ if $\bar{A}$ has empty interior. Equivalently $A$ is rare if it is not dense in any nonempty open subset of $X$. Equivalently, $A$ is rare if its exterior is dense in $X$. 
\end{definition}




\part{Manifolds}



\printbibliography

\end{document}