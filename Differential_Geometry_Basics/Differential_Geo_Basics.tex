\documentclass[a4paper,12pt,parskip=half*,chapterprefix=true,numbers=noendperiod]{scrreprt}

% LANGUAGES AND SYMBOLS
\usepackage[utf8]{inputenc}
\usepackage[T1]{fontenc}
\usepackage[english]{babel}
\usepackage{lmodern}
\usepackage{xparse}

% EVERYDAY PACKAGES
\usepackage{lipsum}
\usepackage[shortlabels]{enumitem}
\usepackage{verbatim}
\usepackage[normalem]{ulem}
\usepackage[dvipsnames]{xcolor}
\usepackage{csquotes}
\usepackage{pdfpages}
\usepackage{easy-todo}
\usepackage{setspace}
%\usepackage[all]{nowidow}

\usepackage{graphicx}
\usepackage{caption}
\usepackage{subcaption}

% MATHEMATICS
\usepackage{mathtools}
\usepackage{stmaryrd}
\usepackage{amsthm, thmtools}
\usepackage{framed}
\usepackage{nameref}
\usepackage[colorlinks,menucolor=blue,linkcolor=blue, citecolor=blue, urlcolor=blue]{hyperref} 
\usepackage[capitalise]{cleveref}% if problems, load cleveref last
\crefname{subsection}{Subsection}{Subsections}
\Crefname{subsection}{Subsection}{Subsections}
%\usepackage[retainorgcmds]{IEEEtrantools}
\usepackage{amssymb}
\usepackage[mathscr]{euscript}
\usepackage{esint}
\usepackage{esvect}
\usepackage{relsize}

\usepackage{tikz}
\usepackage{tikz-qtree}
\usepackage[framemethod=tikz]{mdframed}
\usetikzlibrary{cd}
\usetikzlibrary{matrix}
\usetikzlibrary{backgrounds}

%Packages for references, Indices

\usepackage{imakeidx}[intoc]
\usepackage{biblatex}
\addbibresource{../refs.bib}

%Additionnal Packages

\usepackage{float}
\usepackage{cancel}

%Macors

\newcommand{\opname}{\operatorname}

%Theorems

\newtheorem{theorem}{Theorem}[section]
\newtheorem{proposition}{Proposition}[section]
\newtheorem{lemma}{Lemma}[section]
\newtheorem{corollary}{Corollary}[section]

\theoremstyle{definition}
\newtheorem{definition}{Definition}[section]
\newtheorem{example}{Example}[section]
\newtheorem{notation}{Notation}[section]

\theoremstyle{remark}
\newtheorem*{remark}{Remark}

\title{Basics of Differential Geometry}
\author{Maxime Willaert}
\begin{document}

\maketitle

\tableofcontents

\part{Topology}

\chapter{Intro}

Main references are \cite{Munkres:Top,Lee:IntTopMan}.

\section{Things left to learn}

The different sets of axioms one can use to define a topological space, as in \cite{Wiki:AxiomTop}. A topological space is most commonly defined by specifying its open sets. But one can also define a topology in the following ways:
\begin{enumerate}[(i)]
	\item By specifying its neighborhoods or closed sets.
	\item By specifying the interior, closure, exterior, boundary or 'derived set' operators.
	\item Through nets or filters (which are equivalent).
\end{enumerate}
The notions of \textbf{nets} and \textbf{filters}, their equivalence and their relationship with the notion of a topology, should be explored (see \cite{Wiki:FiltersTop}). The different notions of convergence that can be defined in a topology, and the degree to which they determine this topology is also an interesting question \cite{Wiki:SequentialSpace,Wiki:ConvergenceSpace}.

General topology (more set-theoretic than algebraic, and not focused on finite-dimensional topological manifolds) as in \cite{Munkres:Top}.

\chapter{Topological spaces}

\section{Basic definitions}

\begin{definition}[Topology]
A \textbf{topology} on a set $X$ is a collection $\tau$ of subsets of $X$ such that
\begin{enumerate}[(i)]
	\item $\tau$ contains $\emptyset$ and $X$;
	\item The union of the elements of any subset of $\tau$ is again in $\tau$;
	\item The intersection of the elements of any finite subset of $\tau$ is again in $\tau$.
\end{enumerate}
A \textbf{topological space} is a pair $(X,\tau)$ consisting of a set $X$ together with a topology $\tau$ on $X$. The elements of $\tau$ are called the \textbf{open sets} of $(X,\tau)$.

Given two topological spaces $(X,\tau)$, $(X',\tau')$, a map $f:(X,\tau)\to(X',\tau')$ is said to be \textbf{continuous} if for any $U\in\tau'$, $f^{-1}(U)\in\tau$. In most instances we can omit $\tau$ when referring to the topological space $(X,\tau)$ with no risk of confusion.
\end{definition}

\begin{definition}[Neighborhoods]
	Let $X$ be a topological space.
	\begin{enumerate}[(i)]
		\item Let $K$ be a subset of $X$, another subset $N$ of $X$ is said to be a \textbf{neighborhood} of $K$ if there exists an open subset $U$ of $X$ such that $K\subseteq U\subseteq N$. An \textbf{open neighborhood} of $K$ is an open subset of $X$ that contains $K$.
		\item Let $x$ be a point of $X$. An (open) neighborhood of $x$ is an (open) neighborhood of the singleton $\{x\}$.
	\end{enumerate}	
\end{definition}

\begin{definition}[Closed subsets]
A subset $F$ of $X$ is said to be \textbf{closed} if its complement $X-F$ is open.
\end{definition}

\begin{proposition}
Let $X$ be a topological space.
\begin{enumerate}[(i)]
	\item $\emptyset$ and $X$ are closed.
	\item Any intersection of closed subsets of $X$ is closed.
	\item A finite unions of closed subsets of $X$ is closed.
\end{enumerate}
\end{proposition}

\begin{proposition}
	A map between topological spaces is continuous if and only the preimage of any closed subset is closed.
\end{proposition}
\begin{proof}
	This is because for any map $f:X\to Y$ and any subset $A\subseteq Y$, $f^{-1}(Y-A)=X-f^{-1}(A)$.
\end{proof}

\begin{definition}[Closure and interior]
	Let $A$ be a subset of a topological space $X$.
	\begin{enumerate}[(i)]
		\item The \textbf{closure} of $A$, denoted $\bar{A}$ is the smallest closed subset containing $A$.
		\begin{equation*}
			\bar{A}:=\bigcap\{F\subseteq X|S\text{is closed and }A\subseteq F\}.
		\end{equation*}
		\item The \textbf{interior} of $A$, denoted $\text{Int}(A)$ is the largest open subset contained in $A$.
		\begin{equation*}
			\text{Int}(A):=\bigcup\{U\subseteq X|U\text{is open and }U\subseteq A\}.
		\end{equation*}
		\item The \textbf{exterior} of $A$, denoted by $\text{Ext}(A)$, is defined to by $\text{Ext}(A):=X-\bar{A}$, it is the complement of the closure, that is the largest open that does not overlap with $A$.
		\item The \textbf{boundary} of $A$, denoted by $\partial A$ is defined by $\partial A:=\bar{A}-\text{Int}(A)$.
	\end{enumerate}
\end{definition}

\begin{proposition}
Let $A$ be a subset of an topological space $X$.
\begin{enumerate}[(i)]
	\item A point is in $\text{Int}(A)$ if and only if it has a neighborhood contained in $A$.
	\item A point is in $\text{Ext}(A)$ if and only if it has a neighborhood contained in $X-A$.
	\item A point is in $\partial A$ if and only if any neighborhood of it contains both a point of $A$ and a point of $X-A$.
	\item A point is in $\bar{A}$ if and only if any neighborhood of it contains a point of $A$.
	\item The following are equivalent:
	\begin{itemize}
		\item $A$ is open.
		\item $A=\text{Int}(A)$.
		\item $A$ contains none of its boundary points (hence the 'open terminology').
		\item Any point of $A$ has a neighborhood contained in $A$.
	\end{itemize}
	\item The following are equivalent:
	\begin{itemize}
		\item $A$ is closed.
		\item $A=\bar{A}$.
		\item $A$ contains all of its boundary points (hence the 'closed' terminology).
		\item Any point of $X-A$ has a neighborhood contained in $X-A$.
	\end{itemize}
\end{enumerate}
\end{proposition}

\begin{definition}[Limit and isolated points]
Let $A$ be a subset of a topological space $X$.
\begin{enumerate}[(i)]
	\item A point $p\in X$ (not necessarily in $A$) is a \textbf{limit point} of $A$ if any neighborhood of $p$ contains a point of $A$ other than $p$. Limit points are also called \textbf{cluster points} and \textbf{accumulation} points.
	\item A point $p\in A$ is \textbf{isolated in $A$} if $p$ has a neighborhood $N$ such that $N\cap A=\{p\}$. 
\end{enumerate}
Observe that any point of $A$ is either isolated in $A$ or a limit point of $A$.
\end{definition}

\begin{proposition}
A set is closed if and only if it contains all of its limit points.
\end{proposition}

\begin{definition}[(Nowhere) dense sets]
A subset $A$ of a topological space $X$ is said to be \textbf{dense} in $X$ if $\bar{A}=X$. Given a subset $S$ of $X$, $A$ is said to be dense in $S$ if $A\cap S$ is dense in $S$ (with the subset topology). It is said to be \textbf{nowhere dense} or \textbf{rare} in $X$ if $\bar{A}$ has empty interior. Equivalently $A$ is rare if it is not dense in any nonempty open subset of $X$. Equivalently, $A$ is rare if its exterior is dense in $X$. 
\end{definition}

\section{Maps between topological spaces}

\begin{proposition}
	Some basic properties of continuous maps.
	\begin{enumerate}[(i)]
		\item A constant map is continuous.
		\item Compositions of continuous maps are continuous.
		\item Let $X$ and $Y$ be two topological spaces. A map $f:X\to Y$ is continuous if and only if for any open subset $U$ of $X$, $f|_U$ is continuous. In particular the restriction of a continuous map to any open subset is continuous.
	\end{enumerate}
\end{proposition}

\begin{definition}
	Let $f:X\to Y$ be a map between two topological spaces.
	\begin{enumerate}[(i)]
		\item $f$ is \textbf{closed} if it takes closed subsets of $X$ to closed subsets of $Y$.
		\item $f$ is \textbf{open} if it takes open subsets of $X$ to open subsets of $Y$.
		\item $f$ is a \textbf{homeomorphism} if $f$ is a continuous bijection whose inverse is continuous.
		\item $f$ is a \textbf{local homeomorphism} if any point of $X$ admits an open neighborhood $U$ such that $f|_U:U\to f(U)$ is a homeomorphism. Clearly a homeomorphism is a local homeomorphism.
	\end{enumerate}
\end{definition}

\begin{proposition}[Properties of local homeomorphisms]
	Let $f:X\to Y$ be a local homeomorphism.
	\begin{enumerate}[(i)]
		\item $f$ is open and closed.
		\item If $f$ is bijective, then $f$ is a homeomorphism.
	\end{enumerate}
\end{proposition}

\begin{proposition}[Characterization of homeomorphism]
	Let $f:X\to Y$ be a bijective continous map. The following are equivalent:
	\begin{enumerate}[(i)]
		\item $f$ is a homeomorphism.
		\item $f$ is open.
		\item $f$ is cosed.
	\end{enumerate}
\end{proposition}

\begin{proposition}[Characterizing maps]
	Let $f:X\to Y$ be a map between two topological spaces.
	\begin{enumerate}[(i)]
		\item $f$ is continous if and only if $f(\bar{A})\subseteq \overline{f(A)}$ for all $A\subseteq X$.
		\item $f$ closed if and only if $f(\bar{A})\supseteq \overline{f(A)}$ for all $A\subseteq X$.
		\item $f$ is continous if and only if $f^{-1}(\text{Int}B)\subseteq \text{Int}(f^{-1}(B))$ for all $B\subseteq Y$.
		\item $f$ is open if and only if $f^{-1}(\text{Int}B)\supseteq \text{Int}(f^{-1}(B))$ for all $B\subseteq Y$.
	\end{enumerate}
\end{proposition}

\section{Generating topologies}


\begin{definition}[Comparing topologies]
	Let $\tau$, $\tau'$ be topologies on a set $X$. When $\tau\subseteq \tau'$ we say that $\tau$ is \textbf{coarser} (or \textbf{smaller}) than $\tau'$, or that $\tau'$ is \textbf{finer} (or \textbf{larger}) than $\tau$. We say that $\tau$ and $\tau'$ are \textbf{comparable} if $\tau\subseteq\tau'$ or $\tau'\subseteq\tau$.
\end{definition}
\begin{example}
	The finest topology on a set $X$ is the \textbf{discrete topology} $\tau=\mathcal{P}(X)$ (all subsets of $X$ are open for the discrete topology). The coarsest topology on a set $X$ is the \textbf{trivial} or \textbf{indiscrete topology} $\tau:=\{\emptyset,X\}$.
\end{example}

\begin{proposition}\label{prop:TopStableInt}
	Let $(\tau_i)_{i\in I}$ be a family of topologies on a set $X$. Then the intersection $\tau:=\bigcap_{i\in I}\tau_i$ is a topology on $X$.
\end{proposition}
\begin{remark}
	The empty intersection would yield the discrete topology.
\end{remark}

\begin{definition}[Preorder]
	A binary relation $\leq$ on a set $A$ is a \textbf{preorder} if it is \textbf{reflexive} and \textbf{transitive}, meaning that for all $a,b,c\in A$
	\begin{enumerate}[(i)]
		\item (Reflexivity) $a\leq a$;
		\item (Transivity) $a\leq b$ and $b\leq c$ implies $a\leq c$.
	\end{enumerate}
	We'll often use the word 'preorder' to refer to a pair $(A,\leq)$ consisting of a set $A$ and a preorder $\leq$ on $A$.
\end{definition}

\begin{definition}[Partial order]
	A partial order on a set $A$ is a preoder $\leq$ on $A$ which is \textbf{antisymmetric} meaning that for all $a,b\in A$, $a\leq b$ and $b\leq a$ implies $a=b$. A \textbf{partially ordered set} or \textbf{poset} is a set together with a partial order (sometimes required to be nonempty).
\end{definition}

See personal notes on \textbf{lattices} \cite{personal:Lattices}. Given a subset $S$ of a poset $(P,\leq)$, the \textbf{supremum} of $S$ is sometimes called its \textbf{join}, while the \textbf{infimum} of $S$ is sometimes called its \textbf{meet}.

Now let $\text{Top}(X)$ denote the set of topologies on $X$, which becomes a poset when equipped with the inclusion relation $\subseteq$. $(\text{Top}(X),\subseteq)$ admits top and bottom elements in the form of the discrete and indiscrete topologies. Furthermore, by proposition \ref{prop:TopStableInt}, $(\text{Top}(X),\subseteq)$ admits arbitrary meets (infima), which automatically implies that $(\text{Top}(X),\subseteq)$ admits arbitrary joins (the supremum of a subset $S$ of $\text{Top}(X)$ will be obtained as the infinimum of the upper bounds of $S$, see \cite{personal:Lattices}). Thus we see that $(\text{Top}(X),\subseteq)$ is a complete lattice.

\begin{proposition}
	$(\text{Top}(X),\subseteq)$ is a complete lattice.
	\begin{enumerate}[(i)]
		\item Its upper bound is $\mathcal{P}(X)$, its lower bound is $\{\emptyset,X\}$.
		\item The infimum of a family $(\tau_i)_{i\in I}$ of topologies on $X$ is given by their intersection $\bigcap_{i\in I}\tau_i$.
		\item The supremum $\bigvee_{i\in I}\tau_i$ of a family $(\tau_i)_{i\in I}$ of topologies on $X$ is the infimum of its upperbounds, i.e.
		\begin{equation*}
			\bigvee_{i\in I}\tau_i=\bigcap\{\tau\in\text{Top}(X)|\tau_i\subseteq\tau\text{ for all }i\in I\}.
		\end{equation*}
		Here the $\bigvee$ notation is warranted because $\bigvee_{i\in I}\tau_i$ will \emph{not} coincide with $\bigcup_{i\in I}\tau_i$ in general.
	\end{enumerate}
\end{proposition}

\begin{definition}[Generated topology]
	Let $\xi$ be a collection of subsets of $X$. The \textbf{topology generated by $\xi$}, denoted by $\langle\xi\rangle$, is defined to be the coarsest topology containing $\xi$, i.e.
	\begin{equation*}
		\langle\xi\rangle=\bigcap\{\tau\in\text{Top}(X)|\xi\subseteq\tau\}.
	\end{equation*}
	Given a topology $\tau$ on $X$, a collection $\xi$ of subsets of $X$ that generates $\tau$ in the above sense is called a \textbf{subbasis} of $\tau$.
\end{definition}
\begin{proposition}
	The topology generated by $\xi$ is the collection of all unions of finite (possibly empty) intersections of elements of $\xi$.
\end{proposition}
\begin{remark}
	Allowing for empty intersections of elements of $\xi$ is of crucial importance, as $X$ (the empty intersection) belongs to the generated topology.
\end{remark}

\begin{definition}[Basis for a given topology]
	Let $\tau$ be a topology on a set $X$. A \textbf{basis} for $\tau$ is a subcollection $\beta\subseteq\tau$ such that any open of $\tau$ is a union of elements of $\beta$ (note that the empty union is the empty set). In that case it is easy to see that $\tau$ is the topology generated by $\beta$.
\end{definition}
\begin{proposition}
	$\beta\subseteq\tau$ is a basis of $\tau$ if and only if for any open $U$ and any point $p\in U$ there exists $B\in\beta$ such that $p\in B\subseteq U$.
\end{proposition}

A criterion that determines whether a collection of subsets is the basis of some topology.
\begin{proposition}
	Let $\beta$ be a collection of subsets of $X$. Then $\beta$ is the basis of the topology $\langle\beta\rangle$ it generates if and only if
	\begin{enumerate}[(i)]
		\item $\beta$ covers $X$, meaning that $X=\bigcup\beta$;
		\item For any pair $A,B\in\beta$ and any point $p\in A\cap B$ there exists $C\in\beta$ such that $p\in C\subseteq A\cap B$.
	\end{enumerate}
\end{proposition}

\section{Metrizable spaces and convergence}

\begin{definition}[Convergence]
	Let $X$ be a topological space. A sequence $(x_n)_{n\in\mathbb{N}}$ in $X$ \textbf{converges} to a point $x\in X$ if $(x_n)_n$ is \textbf{eventually} in any neighborhood of $x$, i.e. if for any open neighborhood $U$ of $x$, there exists $N\in\mathbb{N}$ such that $x_m\in U$ for all $m\geq N$. We then write $\lim_{n\to\infty}x_n=x$ or $x_n\to x$.
\end{definition}

\begin{definition}[Metric]
	A \textbf{metric} on a set $X$ is a map $d:X\times X\to\mathbb{R}$ with the following property:
	\begin{enumerate}[(i)]
		\item (Positivity) $d(x,y)\geq 0$ for all $x,y\in X$ and $d(x,y)=0$ if and only if $x=y$;
		\item (Symmetry) $d(x,y)=d(y,x)$ for all $x,y\in X$;
		\item (Triangle inequality) $d(x,z)\leq d(x,y)+d(y,z)$ for all $x,y,z\in X$.
	\end{enumerate}
	A \textbf{metric space} is a set equipped with a metric.
\end{definition}

\begin{definition}[Induced topology]
	Let $(X,d)$ be a metric space.
	\begin{enumerate}[(i)]
		\item For $r>0$ and $x\in X$, the \textbf{open ball of radius r around x} is the set
		\begin{equation*}
			B_r(x):=\{y\in X|d(x,y)<r\}.
		\end{equation*}
		\item For $r\geq 0$ and $x\in X$, the \textbf{closed ball of radius r around x} is the set
		\begin{equation*}
			\bar{B}_r(x):=\{y\in X|d(x,y)\leq r\}.
		\end{equation*}
		\item The topology on $X$ induced by $d$ is the topology generated by the open balls, which form a basis for that topology. In other words, a subset of $X$ is open for the induced topology if it contains an open ball around each of its points.
	\end{enumerate}
\end{definition}

\begin{definition}[Metrizable space]
	A topology $\tau$ on a set $X$ is said to be \textbf{metrizable} if it there exists a metric on $X$ that induces $\tau$.
\end{definition}

\begin{proposition}
	A sequence $(x_n)_n$ in a metric space $(X,d)$ converges to $x$ if and only if for any $\epsilon>0$ there exists $N\in\mathbb{N}$ such that $d(x_m,x)<\epsilon$ for all $m\geq N$.
\end{proposition}

\part{Manifolds}



\printbibliography

\end{document}