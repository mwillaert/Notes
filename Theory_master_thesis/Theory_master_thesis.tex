\documentclass{report}
\usepackage{graphicx} % Required for inserting images
\usepackage[pdfusetitle]{hyperref}
\usepackage{amssymb,amsmath,amsthm}

\theoremstyle{definition}
\newtheorem{definition}{Definition}
\newtheorem{notation}{Notation}

\title{Ressources Thesis}
\author{Maxime Willaert}
\date{June 2024}

\begin{document}

\maketitle

\tableofcontents

\section{Planning the Thesis}

A version without principal connections (and with the material I have for now)
\begin{itemize}
    \item Sekigawa Vanhecke
    What we need to prove the actual theorem
    \begin{itemize}
        \item Parallelism for torsionless connection garanteeing integrability of almost complex structure and Kahler form.
        \item Kahler Connections (manifolds in general)
        \item Necessary conditions from expansion
        \item Characterization of locally symmetric spaces
    \end{itemize}
    \item Type S connections are preferred
    \begin{itemize}
        \item The fact that any affine manifold with volume form preserved by geodesic symmetries is L3 (in fact with divergence-preserving geodesic symmetries).
        \item Maybe some stuff on preferred connections (some of the stuff found by Bourgeois and Cahen, or the variational principle IDK).
    \end{itemize}
    \item Basic prerequisites for all that (Sigurdur Helgason/KN):
    \begin{itemize}
        \item Connections on vector bundles (ie covariant derivatives)
        \begin{itemize}
            \item Parallel transport: use pullback of vector bundle and connection
        \end{itemize}
        \begin{itemize}
            \item Curvature (no Bianchi identity?)
        \end{itemize}
        \item Affine connections
        \begin{itemize}
            \item Torsion and Curvature tensors
            \begin{itemize}
                \item Bianchi identities
            \end{itemize}
            \item Geodesics
            \begin{itemize}
                \item Normal coordinates
                \item Geodesic Symmetries
                \item Jacobi fields (equation)
            \end{itemize}
            \item Characterization of local symmetric spaces
        \end{itemize}
        \item Riemannian Connections
        \begin{itemize}
            \item Special Properties of Riemannian connections (especially their curvature).
            \item Kahler manifolds
            \begin{itemize}
                \item Compatible triples
                \begin{itemize}
                    \item Local model of hermitian vector space to motivate + give 2 out of 3 property. Also the integrability conditions only apply to symplectic and complex structures (since they are not "essentially pointwise" like the riemannian structure).
                \end{itemize}
                \item Integrability of almost complex structure (different characterization).
                \item Integrability of Kahler forms (in general differential forms).
            \end{itemize}
        \end{itemize}
        \item Symplectic manifolds? Symplectic connections? Not many reasons to include that but whatevs, we'll see.
    \end{itemize}
\end{itemize}

That's already so much, we probably won't include principal connections in all that.

\subsection{New Planning}

What I want my readers to get out of reading my thesis. I want to present Sekigawa and Vanhecke's theorem in a clean way. I then want to show the small extension I made. Mainly the extension of the results of D'Atri and Nickerson to the affine (equiaffine case), allowing us to deduce that a space of type S has preferred connection.

We try to work from the end. We'll start by listing all the things we could mention, to thin things out afterwards.
\begin{itemize}
    \item Sekigawa-Vanhecke's theorem.
    \begin{itemize}
        \item Taylor expansion in normal coordinates of the geodesic form gives us a curvature condition.
        \item We can then conclude using results of D'Atri and Nickerson, which make key use of the notion of cubic diffeomorphism.
    \end{itemize}
    \item Any affine manifold with divergence-preserving geodesic symmetries (ie whose geodesic symmetries preserve the connection on the determinant bundle) have cyclic parallel ricci tensor.

    Do we obtain all the Ledger Conditions as in D'Atri and Nickerson? ALSO we should clear-up the different equivalent statements of these Ledger conditions (found in different places in the literature, Ledger, Vanhecke, D'Atri-Nickerson).
    \begin{itemize}
        \item As a corollary, this would apply to an equiaffine mfd whose parallel volume forms are symmetric.

        In particular it would apply to any type S manifold.
    \end{itemize}
\end{itemize}

Necessary notions:
\begin{itemize}
    \item Connections on Vector Bundles
    \begin{itemize}
        \item Vector bundles.
        \begin{itemize}
            \item Morphisms
            \item Pullbacks
        \end{itemize}
        \item Connections on vector bundles.
        \begin{itemize}
            \item Definition via covariant derivative.
            \item Expression in local coordinates. Locality. Dependence only on pointwise vector.
            \item Pullback connection.
            \item Parallel transport.
            \item Curvature.
        \end{itemize}
        \item Possible Extra elements:
        \begin{itemize}
            \item Horizontal distribution.
            \item Integrability of the horizontal distribution given by curvature (Frobenius theorem).
            \item Existence of horizontal section.
        \end{itemize}
    \end{itemize}
    \item Affine Connections
    \begin{itemize}
        \item Curvature and Torsion
        \begin{itemize}
            \item Bianchi identities.
            \item Ricci curvature.
        \end{itemize}
        \item Geodesics
        \begin{itemize}
            \item Equation in local coordinates.
            \item Existence and uniqueness.
            \item Possible Extras: 
            \begin{itemize}
                \item Geodesic spray. (A geometric alternative to the analytic approach).
                \item Showing that the geodesic spray determines a torsionless connection. And that there exists a unique torsionless connection with the same geodesic spray (not needed probably).
            \end{itemize}
            \item Exponential mapping.
            \item Normal coordinates.
            \begin{itemize}
                \item The vanishing of the Christoffel symbols at the origin.
            \end{itemize}
            \item Existence of convex neighborhood. (is it needed here?).
        \end{itemize}
        \item Local geodesic symmetries
        \begin{itemize}
            \item Definition and existence from normal coordinates.
            \item Possible extra: torsionless connection is determined by its geodesic symmetries. And symmetric tensors are parallel.
            \begin{itemize}
                \item Pierre's formula.
                \item Normal coordinates.
                \item Introducing symmetric tensors and showing that symmetric tensors are parallel (distinguish with "preserved by geodesic symmetries" in old sense).
            \end{itemize}
            Question: where should we put locally symmetric spaces? Should we put them all in the same place, or stagger them?
            \item Possibility: Affine locally symmetric spaces. And their local characterization.
        \end{itemize}
        \item Riemannian connections.
        \begin{itemize}
            \item Existence of a unique Levi-Civita connection.
            \item Maybe: some identities satisfied by curvature of riemann mfd specifically. Scalar curvature.
            \item Locally symmetric riemann spaces.
            \begin{itemize}
                \item Riemannian mfd is locally symmetric as a riemannian mfd iff it is locally symmetric as an affine mfd.
            \end{itemize}
        \end{itemize}
    \end{itemize}
\end{itemize}

\sectiocally, and scalar cun{Planning}

Things that might be useful to learn:
\begin{itemize}
    \item Helgason (will become my main reference) - highly self-contained.
    \begin{itemize}
        \item A complete treatment of affine connections (and affine symmetric spaces) in terms of covariant derivatives (not principal connections like in KN).
        \item Homogenous spaces.
        \item Basic Lie theory (eg classification of semi-simple Lie algebras).
    \end{itemize}
    \item Ottmar Loos: multiplication point of view on symmetric spaces (very interesting).
    \item Finishing the Lee. Especially homogenous spaces.
    \item KN:
    \begin{itemize}
        \item The whole 1st book: connections, affine connections, riemannian structures.
        \item 2nd book: jacobi fields (for affine connections as well), symmetric spaces, Kahler, homogenous spaces ESPECIALLY invariant connections.
    \end{itemize}
    \item Pierre's work:
    \begin{itemize}
        \item Espaces Symétriques Symplectiques. Homogenous symplectic spaces.
        \item Avec Mélanie: Affine Connections and Symmetry jets.
    \end{itemize}
    \item Sekigawa and Vanhecke: Getting the proof of their theorem on Kahler connections of type S.
    \item D'Atri spaces:
    \begin{itemize}
        \item Ledger Conditions.
        \item Examples of D'Atri Spaces, e.g. a particular kind of homogenous space.
    \end{itemize}
    \item Properties of Lie groups and algebras for constructing examples of homogenous spaces with (invariant) connections of type S.
    \begin{itemize}
        \item Homogenous symplectic spaces.
        \item Kinds of Lie groups/algebras: semi-simple, solvable, nilpotent, unimodular,...
    \end{itemize}
    \item Real Analytic Manifolds.
\end{itemize}

The main result for now:
\begin{itemize}
    \item The necessary conditions obtained from the Taylor expansion in normal coordinates, which become sufficient when the structures are analytic (probably).
    \item It's the best bet for now for constructing examples of spaces with connection of type S, since integrating the 
\end{itemize}

\section{Need to study}

\begin{itemize}
    \item Theory of covering spaces
    \begin{itemize}
        \item Question: How does the unique lifting property behave wrt smoothness? In particular, given a smooth covering $\pi:E\to M$ is the "topological" automorphism group of $\pi$ equal to the "smooth" automorphism group of $\pi$?
        \item In particular the significance of normal covering spaces, and their equivalence to the transitivity of deck transformations.
        \item Deck transformations, relationship with monodromy (as seen in the "Riemann Surfaces" course).
    \end{itemize}
    \item Smooth maps:
    \begin{itemize}
        \item Rank theorems.
        \item Submersions, immersions, embeddings (proper embeddings).
    \end{itemize}
    \item Vector Fields
    \begin{itemize}
        \item Canonical form for commuting vector fields (pretty involved proof).
    \end{itemize}
    \item Distributions and Foliations
    \begin{itemize}
        \item Showing that the union of connected integral manifolds of an involutive distribution, that share a point, admits a unique smooth mfd structure making it into a connected integral manifold of that same distribution (pretty involved proof).
    \end{itemize}
    \item Lie groups
    \begin{itemize}
        \item For a Lie subgroup closed is equivalent to embedded (pretty involved proof).
        \item Quotient Manifold
        \begin{itemize}
            \item The different characterizations of proper actions (a bit technical).
            \item The quotient manifold theorem for smooth, free, proper actions, pretty involved proof.
        \end{itemize}
    \end{itemize}
    \item Cohomology of Lie algebras - which allows us to compute the de Rham cohomology of compact simply connected Lie groups (simply connected so that it is determined by its Lie algebra, compact so that the complex of differential forms can be reduced to the complex of left invariant forms via an averaging process).

    \item Theory of connections.
    \begin{itemize}
        \item Geometric meaning of connection 1-form.
        \item Displaying the geometric interpretation of the curvature as infinitesimal holonomy.

        Also, obtaining a geometric interpretation of the structure equations that apply to the curvature form.
        \item Holonomy theorems (pretty involved proofs):
        \begin{itemize}
            \item Proving that $\Phi^0(u)$ is a connected Lie subgroup of $G$.
            \item The Ambrose-Singer theorem.
            \item Realizing any Lie group as the holonomy group of a principal connection over a paracompact mfd of $\dim\geq 2$.
        \end{itemize}

        Precise regularity conditions on the paths involved when constructing holonomy groups.
        \item Exterior covariant derivative.
        \begin{itemize}
            \item Formula for the exterior covariant derivative of a tensorial form of type $adG$ in terms of connection form.
        \end{itemize}
        \item Ehresmann point of view on linear connections. Especially the fact that giving a covariant derivative is equivalent to the datum of a horizontal distribution or vertical projection which respect the linear structure.

        With all the category/Vector bundle crap.
    \end{itemize}
\end{itemize}

\section{Questions}

\begin{itemize}
    \item Foliations and distributions:
    \begin{itemize}
        \item Which ideals of $\Omega(M)$ are annihilating ideals for some distribution $D$? A necessary condition is the existence for any point $p$ of $M$ of k linearly independent 1-forms $\omega^1,...,\omega^k$ defined on a nbd $U$ of $p$ such that an r-form $\eta$ defined of $U$ belongs to $I$ iff we can write $\eta=\sum_{i=1}^k\omega^i\wedge\beta^i$ for some $(r-1)$-forms $\beta^1,...,\beta^k$.
    
        Note: It seems it would be more natural to define an ideal as a family of subspaces $I_U\subseteq\Omega(U)$ for each open $U$ of $M$ and to require compatibility with the sheaf structure of $\Omega(M)$ (so in particular with restriction).
        \item When is the space of leaves of foliations a smooth manifold? Is that one of the questions tackled by diffeology and noncommutative geometry?
    \end{itemize}
    \item Lie theory:
    \begin{itemize}
        \item How would the Lie correspondence work for infinite-dimensional Lie groups and algebras?
    \end{itemize}
    \item Theory of connections
    \begin{itemize}
        \item Show that the curvature form is given by $\tilde{\Omega}(X,Y)=[X^*,Y^*]-[X,Y]^*$ after using equivalence between $VP$ and $P\times\mathfrak{g}$ (see Vakar+should work since the difference given above must be vertical).
        \item Linear connections.
        \begin{itemize}
            \item Geodesics.
            \begin{itemize}
                \item What are the necessary and sufficient conditions for a vector field on $TM$ to be the geodesic spray of some affine connection?
                \item How can we specify an affine connection by providing a geodesic spray and a torsion?
                \item Is the geodesic spray determined by the germs of geodesic symmetries?
            \end{itemize}
            \item Equivalence problem:
            \begin{itemize}
                \item Is the statement of theorem 7.1 complete? Shouldn't we require the diffeo $f$ to "preserve" the normal coordinates, by requiring $y^i\circ f=x^i$ or at least by requiring $f(x_0)=y_0$?
            \end{itemize}
        \end{itemize}
    \end{itemize}
    \item Symplectic connections
    \begin{itemize}
        \item What does it mean for the curvature to be determined by the ricci curvature?
    \end{itemize}
    \item More profound questions (question fundamental concepts):
    \begin{itemize}
        \item A huge portion of contemporary mathematicians claim that things like category theory, topos theory,... will help us advance in geometry (and physics), why is that?
        \item What are elements of mathematics that have helped geometry so far? Algebra, analysis, the theory of differential equations. Why have these theories been successful? What is the essence of differential geometry, and other geometries? What are the relationships between the different kinds of geometries? Why is the notion of smoothness or regularity, which appears often in analysis and differential geometry, so fundamental?
        \item Questioning core concepts (one of the questions I like the most)
        \begin{itemize}
            \item Why are the notions of curvature and torsion so relevant? What is the relevance of the notion of connection? What is the relevance of symmetric spaces and geodesic symmetries.
        \end{itemize}
    \end{itemize}
\end{itemize}

\section{To Write Still}

[Lee]

\begin{itemize}
    \item Foliations and Lie theory
    \begin{itemize}
        \item Lie subalgebra and Lie subgroups
        \item Infinitesimal actions
        \item Normal subgroups and ideals
        \item Lie correspondence (w/ covering theory and Ado's theorem).
    \end{itemize}
    \item Rest of Lie theory
    \begin{itemize}
        \item Exponential map
        \item Closed subgroup theorem
        \item Quotient mfds
        \item Homogenous spaces
        \item Covering mfds
    \end{itemize}
\end{itemize}

[KN 1]

\begin{itemize}
    \item Lie groups
    \begin{itemize}
        \item Left-invariant forms on Lie group, Mauer-Cartan equation.
        \item Real Analyticity of Lie groups and Homogenous spaces.
    \end{itemize}
    \item Fiber Bundles
    \begin{itemize}
        \item Reduction of group of principal fiber bundle.

        Definition in terms of principal subbundle, definition in terms of trivialization.
        \item Details on construction of associated fiber bundle.

        Note: Remember Vakar's thesis, elements of principal fiber bundle should be thought of as "generalized bases" of the associated fiber bundle.
    \end{itemize}
\end{itemize}

\section{To Do List}

\begin{itemize}
    \item Sketch of proof of Sekigawa Vanhecke via the 3-symmetries of A. Gray.
    \item Existence of non-locally symmetric ricci-parallel Kahler mfds shows that L3 does not imply type S for symplectic connections.
    \item Ricci-Type connections and Type S connections. The 1-form and other stuff (check the 1st condition obtained from Taylor expansion of symplectic form in normal coordinates.
    \item Check whether the special curvature condition of D'Atri and Nickerson also works on equiaffine mfd (or more general?).
\end{itemize}

\part{Basic Differential geometry}

Main references: K-N, Lee


\chapter{Smooth Manifolds}

[Lee]

\section{Topological properties of manifolds}

\subsection{Connectedness [Locally path-connected spaces]}

The connected components of a space

\begin{itemize}
    \item A topological space $X$ admits a unique partition $C$ into disjoint non-empty sets called connected components which are to be thought of as the largest connected subsets of $X$.

    The properties of this partition (so of the connected components are the following):
    \begin{itemize}
        \item Any element $c$ of $C$ is connected.
        \item Given an element $c$ of $C$, any connected subset with non-empty intersection with $c$ will be contained in $c$.
    \end{itemize}

    From this we can easily deduce:
    \begin{itemize}
        \item For a connected component $c$, any connected subset containing $c$ will be equal to $c$ (hence why they are the largest).
        \item A point $x$ in $X$, will be contained in a unique connected component $c_x$ (since $C$ is a partition), and $c_x$ will contain any connected subset containing x (in fact it will be equal to the union of all such sets), hence $c_x$ is the largest connected subset containing $x$.
        \item We can see that the equivalence relation corresponding to $C$ (the partition) is the following: $x\sim y$ iff there exists a connected subset containing both $x$ and $y$.

        Indeed, we see that $x$, $y$ belong to the same connected component iff they verify this relation, and since $C$ is a partition, the relation must be an equivalence relation.
    \end{itemize}

    Note: In the 2nd list we deduce some properties of a partition of $X$ verifying the two points of the 1st list. This is useful knowledge, and the fact that the elements of the 2nd list follow from the 1st already gives us uniqueness, since all partitions verifying the elements of the 1st list will be tied to the same equivalence relation on $X$.

    The proof of the existence of a partition verifying the points of the 1st list [we already have uniqueness] essentially relies on the following property:
    \begin{itemize}
        \item Given a family $(c_i)_{i\in I}$ of connected subsets with non-empty intersection (ie the $c_i$'s have a point in common), then $\cup_i c_i$ is connected.

        This is easily seen using the map-characterization of connectedness: given a continuous map $f:\cup_i c_i\rightarrow {0,1}$, for any $i\in I$, since $c_i$ is connected, $f$ restricted to $c_i$ will be constant and equal to $f(x)$, so we see that $f$ will be constant.
    \end{itemize}

    with this property at hand we can use the follow scheme:
    \begin{itemize}
        \item First notice that the relation described above is indeed an equivalence. 
        
        Transitivity, the only non-obvious point, follows from the union property given above: if $x\sim y$ and $y\sim z$, then we have $A$, $B$ connected containing $x,y$ and $y,z$ respectively. $A\cup B$ contains $x$ and $z$ and is connected since $A$ and $B$ share $y$.
        \item We can thus partition $X$ into the set $C$ of equivalence classes of this relation. What's left is showing that they verify the two properties.
        \item Connectedness: given a point $x$ in $X$, its equivalence class $[x]$ will be the union of all connected subsets containing $x$, and will thus be connected by the union property mentioned above.
        \item Maximality: let $A$ be a connected subset sharing a point $x$ with an element of $C$. Then $c=[x]$ and by definition of the relation, $A$ is contained in $c=[x]$.
    \end{itemize}
\end{itemize}

A space can also be partitioned in path-connected components. The construction and properties are completely analoguous, except that each mention of "connectedness" has to be replaced by "path-connectedness". The entire reasoning can still be carried out because path-connectedness shares the same union property as connectedness:
\begin{itemize}
    \item Let $(c_i)_{i\in I}$ be a family of path-connected subsets with nonempty intersection. Then their union $\cup_i c_i$ is path-connected.

    Indeed, let $x$ be a point shared by all the $c_i$'s. Given $y_i$ in $c_i$ and $y_j$ in $c_j$, there exists a path $\gamma_i$ joining $y_i$ to $x$ in $c_i$ and a path $\gamma_j$ joining $x$ to $y_j$ in $c_j$. The concatenation of $\gamma_i$ and $\gamma_j$ will be a path joining $y_i$ and $y_j$ in $\cup_i c_i$.
\end{itemize}

In particular we also have a very nice characterization of the equivalence relation corresponding to the partition in path-connected components:
\begin{itemize}
    \item $x\sim y$ iff $x$ and $y$ are joined by a path. This is equivalent to requiring that there exists a path-connected subset containing both $x$ and $y$.
\end{itemize}

In general, the path-connected components are distinct from the connected components. This is one of the things that make locally path-connected spaces special.

One thing we always have though, is the fact that a path-connected components is always contained in a connected component. Equivalently, the path-connected equivalence relation is stronger than the connected equivalence relation.
\begin{itemize}
    \item This is a consequence of the fact that a path-connected space $X$ is connected. We can see this by choosing a base point $x$ in $X$ then writing $X$ as the union of (images) of paths starting from $x$.

    Note: this means that a path-connected subset with non-empty intersection with a connected component will be contained in that connected component.
\end{itemize}


\begin{definition}[Locally path-connected spaces]
    A topological space $X$ is locally path-connected if any point of $X$ admits a basis of path-connected neighbourhoods. In other words if for any nbd $V$ of $x$, there exists a path-connected nbd $U$ of $x$ contained in $V$.

    Note: this is NOT equivalent to requiring that any point of $x$ admits a path-connected neighbourhood.[Counter-Example?]
\end{definition}

\begin{itemize}
    \item In such a space, the path-connected and connected components are equal and open.

    We can see that the path-connected components are open and equal the connected components by following these steps:
    \begin{itemize}
        \item Show first that a path-connected component is open. Given such component $p$ and a point $x$ in $p$, $x$ admits an open path-connected nbd $U_x$, and $U_x$ must be contained in $p$.
        \item Take a connected component $c$ in $X$. $c$ can be partitioned into the path-connected components contained in $c$. Since this is an open partition by the previous point, and since $c$ is connected, the partition must in fact contain only one element, and $c$ must be path-connected and open.
    \end{itemize}
\end{itemize}

A corollary:
\begin{itemize}
    \item A locally path-connected space is connected iff it is path-connected.
    \item In particular, an open subset of a locally path-connected space is connected iff it is path-connected.

    Note: the open hypothesis is important, think of the graph of $\sin(1/x)$ in $\mathbb{R}^2$ (a locally path-connected space) which is connected by not path-connected.

    To prove this, we can use the fact that an open subset of a locally path-connected space is locally path-connected (the basis part of the definition is important for this), and the previous point.

    To summarize all the points used here, the reason why a locally path-connected space is connected iff it is path-connected is the fact that it will be partitioned into open path-connected components.
\end{itemize}

\subsection{Proper maps}

\begin{itemize}
    \item A continuous map $f:X\to Y$ between two topological spaces is
    \begin{itemize}
        \item Proper if for any compact $K\subseteq Y$, $f^{-1}(K)$ is compact in $X$.

        Note: In particular this rules out "periodic behaviour", eg $\exp:\mathbb{C}\to\mathbb{C}_0$ is not proper.

        Note: Up to now, I've encountered proper maps in group actions, smooth embeddings and in the theory of Riemann Surfaces.
        \item Closed if for any closed subset $A$ of $X$, $f(X)$ is closed in $Y$.
    \end{itemize}
    \item A topological space $X$ is locally compact if any point of $X$ admits a compact neighbourhood.

    Note: A locally compact hausdorff space will admit a neighbourhood basis comprised of compact (and so closed) subsets.
    
    Note: A topological mfd is always locally compact and hausdorff (this goes for smooth mfds as well).
\end{itemize}

Properties of proper maps when we have some local compacity and separability.
\begin{itemize}
    \item Given a proper map $f:X\to Y$, if $Y$ is locally compact and hausdorff, then $f$ is closed.
    \item\label{ProperMapIntoManifold} In particular given two smooth mfds $X,Y$, a proper map $f:X\to Y$ is closed.
\end{itemize}

\section{Smooth Maps}

Differential of a map defined on a product of manifolds:
\begin{itemize}
    \item \label{DifferentialMapOnProduct} Given a smooth map $\phi: M\times N\to S$ and $(p,q)\in M\times N$, $d(\phi)_{(p,q)}=d(\phi_M)_p+d(\phi_N)_q$ with $phi_M:M\to S,p'\to \phi(p',q)$ and $\phi_N:N\to S,q'\to\phi(p,q')$.
\end{itemize}

\section{From KN}

Convention for the Lie derivative.
\begin{itemize}
    \item For vector field $X$ generating a 1-parameter group of transformation $\phi_t$ and a tensor $T$, we have $L_XT=\lim_{t\to 0}\frac1t (T-\phi_tT)$.
\end{itemize}

\chapter{Submersions, Immersions, Embeddings}

[Lee]

\subsection{Embeddings}

Some sufficient conditions for an injective immersion to be an embedding.

\begin{itemize}
    \item Let $F:M\to N$ be an injective smooth immersion. If any of the following hold, then $F$ is a smooth embedding:
    \begin{itemize}
        \item $F$ is an open or closed map.
        \item $F$ is a proper map.

        Note: follows from previous because in that case, $F$ is closed as a smooth proper map.

        Note: In that case, $F$ is a proper embedding.
        \item $M$ is compact.

        Note: Follows from the previous because a continuous map with compact domain is always proper.

        \item $M$ has empty boundary and $\dim M=\dim N$

        Note: follows from the 1st point, because then $F$ is open.
    \end{itemize}
\end{itemize}

\chapter{Vector Bundles and Vector Fields}

\section{Vector Fields}

\subsection{Complete Vector Fields}

A sufficient condition for the completeness of a vector field: the uniform time lemma.
\begin{itemize}
    \item Let $X$ be a vector field on $M$, and let $\theta$ be its flow. If there exists $\epsilon>0$ such that the domain of $\theta^{(p)}$ contains $(-\epsilon,\epsilon)$ for each $p\in M$ (ie $\theta(t,p)$ is defined for $-\epsilon<t<\epsilon$) then $X$ is complete.
\end{itemize}

From this we easily deduce that any left-invariant vector field on a Lie group is complete.
\begin{itemize}
    \item \label{LeftInvariantComplete} Let $G$ be a Lie group and $X$ be an element of $Lie(G)$ (so a left-invariant vector field), then $X$ is complete.

    To see this, take $X\in Lie(G)$ with flow $\theta$.
    \begin{itemize}
        \item There exists $\epsilon>0$ st $(-\epsilon,\epsilon)$ is included in the domain of $\theta^{(e)}$.
        \item For $g\in G$, since $L_g_*X=X$, $L_g\theta^{(e)}$ is an integral curve of $X$ starting at $g$ so that $(-\epsilon,\epsilon)$ is also included in the domain of $\theta^{(g)}$.
    \end{itemize}
\end{itemize}

\subsection{Commuting Vector Fields}

\begin{itemize}
    \item Vector fields commute iff their flows commute (a geometric interpretation of the commutativity of vector fields).

    Showing this is pretty straightforward once we have the expression of Lie brackets in terms of flows. We just have to pay attention to the domains since the vector fields are not necessarily complete.
    \item \label{StandardFormCommutingFields} The standard form of commuting vector fields (the statement itself is rather long). Let $(V_1,...,V_k)$ be a family of linearly independent commuting vector fields on an open subset $W\subseteq M$.Then for a point $p\in W$, there exists a coordinate chart $(U,(s^i))$ centered at $p$ such that $V_i=\partial_{s^i}$ for $1\leq i\leq k$. If $S\subseteq W$ is an embedded codimension-k submanifold and $p$ is a point of $S$ st $T_pS$ is complementary to $(V_1|_p,...,V_k|_p)$ the the chart can be chosen such that $S\cap U$ is the slice defined by $s^1=...=s^k=0$.

    The proof is rather involved as well, some parts are analogous to the flowout theorem (when a submanifold follows the flow of a non-tangent vector field), it relies on the fact that flows of commuting vector fields commute and on the local inversion theorem.

    How we obtain the coordinates:
    \begin{itemize}
        \item We choose a slice chart $(U,(x^i))$ for $S$ centered at $p$, st $S\cap U$ corresponds to $x^1=...=x^k=0$.
        \item Let $\theta_i$ be the flow of $V_i$ for $1\leq i\leq k$.
        \item The chart we choose is $\Phi(s^1,...,s^k)=(\theta_1)_{s^1}...(\theta_k)_{s^k}(0,...,0,s^{k+1},...,s^n)$

        Note: since the flows commute, for any $i$ we can put $\theta_i$ in front and see that $\partial_{s^i}\Phi=V_i$.
        \item By note above, commutativity of the flows makes it "easy" to compute the jacobian of $\Phi$ at $p$ (and the fact that $S$ is complementary to the $V_i$'s means that the jacobian is invertible), and we see that $\Phi$ is a diffeo. on a nbd of $p$ by the local inversion theorem.
    \end{itemize}
\end{itemize}

\section{Vector Bundles}

\subsection{Bundle morphisms}

Characterizing bundle morphisms
\begin{itemize}
    \item Given two vector bundles $E$,$F$, over $M$ there is an exact correspondance between bundle morphisms $E\to F$ (over $M$), and $C^{\infty}(M)$-linear maps $\Gamma(E)\to\Gamma(F)$.
\end{itemize}

\subsection{Subbundles}

We have several characterizations of vector subbundles, the most useful in general is the local frame criterion.

\begin{itemize}
    \item Given a vector bundle $\pi_E:E\to M$, a subbundle is a vector bundle $\pi_D:D\to M$ such that $D$ is an embedded submanifold of $E$ such that $\pi_D$ is the restriction of $\pi_E$ to $D$ and such that for each $p\in M$, $D_p$ is a vector subspace of $E_p$.
    \item (Local frame criterion)\label{LocalFrameCriterion} Given a vector bundle $\pi:E\to M$, and a choice of $k$-dimensional vector subspace $D_p\subseteq E_p$ for each $p\in M$, $\cup_{p\in M}D_p$ (equipped with the restriction of $\pi$) is a vector subbundle of $E$ iff for each $p\in M$ there exists a family of $k$ linearly independant smooth vector fields $(V_1,...,V_k)$ defined on an open nbd $W$ of $p$ st for each $q\in M$, $(V_1|_q,...,V_k|_q)$ span $D_q$.
\end{itemize}

Also useful is the fact that the kernel and image of constant rank bundle morphisms are subbundles.
\begin{itemize}
    \item Given a bundle morphism $F:E\to E'$, the rank of $F|_{E_p}:E_p\to E'_p$ is called the rank of $F$ at $p$. $F$ is said to have constant rank if $F$ has the same rank at all points.
    \item Given a bundle morphism $F:E\to E'$, $Ker(F)$ and $Im(F)$ are subbundles of $E$ and $E'$ respectively iff $F$ has constant rank.
\end{itemize}

\chapter{Distributions and Foliations}

[Lee]

\section{Distributions and Involutivity}

Main (and pretty much only) result of this section is Frobenius' theorem which gives us a local criterion for determining whether a distribution is a foliation.

\begin{itemize}
    \item A distribution on $M$ of rank $k$ is a rank $k$ subbundle of $TM$.

    By the local frame criterion (cf \ref{LocalFrameCriterion}), a rank k distribution on $M$ is equivalently defined as a choice of a k-dimensional subspace $D_p$ of $T_pM$ for each $p\in M$ such that for any $p\in M$ there exists an open nbd $W$ of $p$ and k linearly independent smooth vector fields $V_1,...,V_k$ defined on $W$ such that for each $q\in W$, $(V_1|_q,...,V_k|_q)$ span $D_q$.
\end{itemize}

\subsection{Integral Manifolds and Involutivity}

\begin{itemize}
    \item Given a distribution $D$ on $M$, a non-empty immersed submanifold $N$ in $M$ is an integral manifold of $D$ if $T_pN=D_p$ for all $p\in M$.

    Note: We don't require an integral submanifold to be embedded (ie we don't require existence of slice charts) => think of the Kronecker foliation of the Torus (a good example to keep in mind).
    Note: The reason why involutivity only allows us to derive local properties (so only "immersed", which is equivalent to "locally embedded" + injective) is probably bc involutivity is a local criterion.
    \item A distribution on $M$ is called integrable if each point of $M$ is contained in an integral manifold of $M$.
    \item Definitions of involutivity in terms of sections of $D$ (so vector fields):
    \begin{itemize}
        \item A distribution $D$ is called involutive if for any pair of local sections of $D$, their Lie bracket is also a section of $D$.
        \item A distribution is involutive if and only if $\Gamma(D)$ (the space of global sections of $D$) is a Lie subalgebra of $\mathcal{X}(M)$.

        Note: to show that this new definition implies the previous one, use bump functions (ie partitions of unity).
        \item (In terms of local frames) A distribution $D$ is involutive iff in a nbd of any point of $M$ there exists a local frame $(V_1,...,V_k)$ for $D$ such that $[V_i,V_j]$ is a section of $D$ for any $i,j$.

        Note: This 3rd characterization is obvious, the advantage it offers is that once we've chosen a basis, we only need to check the Lie bracket criterion for the element of the basis (and not for any pair of local sections).
    \end{itemize}
    \item An integrable distribution is involutive. This follows from the fact that if two vector fields are tangent to an immersed submanifold, so is their Lie bracket.
\end{itemize}

\subsection{Involutivity and Differential forms}

The differential ideal criterion, elegant and abstract.
\begin{itemize}
    \item We define $\Omega(M):=\Omega^0(M)\bigoplus...\bigoplus\Omega^n(M)$. An ideal in $\Omega(M)$ is a linear subspace $I\subseteq\Omega(M)$ that is closed under wedge product with any differential form, ie for $\alpha\in I$ and any $\beta\in\Omega(M)$, $\alpha\wedge\beta$ belongs to $I$.
    \item We say that a differential form $\alpha\in\Omega^r(M)$ annihilates the distribution $D$ if $\alpha(X_1,...,X_r)=0$ whenever $X_1,...,X_r$ are local sections of $D$.

    The set of forms annihilating $D$ forms an ideal $I(D)$ of $\Omega(M)$ (a direct sum of differential forms of different orders annihilated $D$ if each component annihilates $D$ in the previous sense).
    \item A differential ideal of $\Omega(M)$ is an ideal $I$ of $\Omega(M)$ such that for each $\alpha\in I$, $d\alpha$ belongs to $I$.
    \item A distribution $D$ is involutive iff the space $I(D)$ of forms annihilating $D$ forms a differential ideal.

    Note: We can see that this 'differential ideal' definition of involutivity follows from the Lie bracket one by using the intrinsic formula for the exterior derivative (the one that uses Lie brackets).

    To see that the Lie brackets one follows from the differential one, we also use the intrinsic formula for exterior derivatives, and we use the fact that a local vector field $X$ is a section of $D$ iff $\alpha(X)=0$ for every 1-form $\alpha$ that annihilates $D$. This can be deduced from the existence of local defining forms for $D$.
\end{itemize}

The existence of local defining forms.
\begin{itemize}
    \item Suppose $M$ is a smooth $n$-mfd and $D$ is a rank-$k$ distribution on $M$, then $D$ is smooth iff for each point $p$ of $M$ there exists $(n-k)$ smooth 1-forms $\omega^1,...,\omega^{(n-k)}$ defined on an open nbd $U$ of $p$ such that $D_q=Ker(\omega^1_q)\cap...\cap Ker(\omega^{(n-k)_q}$ for each $q\in U$.

    To see that the local defining forms exist for a smooth distribution $D$, choose a local frame for $D$, complete it to a local basis of $TM$, then take the local basis $T^*M$ dual to it.
    \item Note: the existence of local defining forms means that a 'local' vector field $X$ is a section of $D$ iff $\eta(X)=0$ for any 1-form annihilating $D$.
\end{itemize}

Once we've chosen local defining forms for $D$, local forms that annihilate $D$ admit a very particular expression.
\begin{itemize}
    \item Given local defining forms $\omega^1,...,\omega^{(n-k)}$ on an open $U$. An $r$-form $\eta$ defined on $U$ annihilates $D$ iff $\eta$ can be written as $\eta=\sum_{i=1}^{n-k}\omega^i\wedge\beta^i$ for some $(r-1)$-forms $\beta^1,...,\beta^{(n-k)}$.

    This can be shown once again by completing $\omega^1,...,\omega^{(n-k)}$ into a local basis of $T^*M$ and taking the dual basis.
\end{itemize}

From this we get two criteria for involutivity:
\begin{itemize}
    \item $D$ is involutive iff for any 1-form $\eta$ that annihilates $D$ on an open subset $U$, $d\eta$ also annihilates $D$ on $U$.

    This follows from the Lie bracket definition via the intrinsic formula for the exterior derivative $d\eta(X,Y)=X\eta(Y)-Y\eta(X)-\eta([X,Y])$.

    The Lie bracket definition follows from this by the intrinsic formula for the exterior derivative and the annihilating 1-form criterion for local sections of $D$.
    \item (Criterion using local defining 1-forms) Given local defining 1-foms $\omega^1,...,\omega^{(n-k)}$ defined on an open $U$ of $M$, TFAE
    \begin{itemize}
        \item D is involutive above $U$.
        \item $d\omega^1,...,d\omega^{(n-k)}$ annihilate D.
        \item There exist smooth 1-forms $\alpha^i_j$ st $d\omega^i=\sum_{j=1}^{(n-k)}\omega^j\wedge\alpha^i_j$ for any $i$.
    \end{itemize}

    It's not hard to see that the defining form criterion is equivalent to the 1-form criterion.
\end{itemize}

\section{Frobenius Theorem}

Still a local theorem. To state it we introduce flat charts for distributions and completely integrable distributions.
\begin{itemize}
    \item A chart $(U,\phi)$ is flat for a distribution $D$ if $\phi(U)$ is a cube and if each slice of $U$ the form $x^{k+1}=c^{k+1},...,x^n=c^n$ for constants $c^{k+1},...,c^n$ is an integral manifold of $D$. This is equivalent to requiring that $\partial_{x^1},...,\partial_{x^k}$ span $D$ at each point of $U$.
    \item A distribution is completely integrable if each point of $M$ is contained in a flat chart for $D$.
\end{itemize}

We already know completely integrable$\implies$integrable$\implies$involutive. Frobenius' theorem states that an involutive distribution is integrable, so that all these notions are equivalent.
\begin{itemize}
    \item (Frobenius) Involutive distributions are completely integrable.

    This follows from the following fact: a distribution is involutive iff for each point $p\in M$ there is a commutative local frame of $D$ defined on an open nbd of $p$. If we combine this with the standard form \ref{StandardFormCommutingFields} of commuting vector fields, we get the desired flat chart.

    To show that an involutive distribution admits commuting local frames, we do the following:
    \begin{itemize}
        \item An easy source of commuting vector fields is coordinates (coordinate vector fields commute). Since here we only need k vector fields, we project onto $\mathbb{R}^k$ and choose the projection $\pi$ so that $d\pi_q:D_q\to T_{\pi(q)}\mathbb{R}^k$ is an iso for each $q\in U$ ie $d\pi$ gives us an iso of vector bundles $D\simeq \pi^*T\mathbb{R}^k$.
        \item For $1\leq i\leq k$ we define $V_i$ as the unique section of $D$ that is $\pi$-related to $\partial_{x^i}$. Since $D$ is involutive, $[V_i,V_j]$ is a section of $D$, and by naturality of the Lie bracket, it is $\pi$-related to $[\partial_{x^i},\partial_{x^j}]=0$, so $[V_i,V_j]=0$.
    \end{itemize}
    \item Notice that if $S$ is a codimension k embedded submanifold of $M$ st $D_p$ is complementary to $T_pS$, we can find a chart $(U,(s^i))$ around $p$ that is flat for $D$ and such that $S\cap U$ corresponds to the slice $s^1=...=s^k=0$.
\end{itemize}

\subsection{Local Structure of Integral Manifolds}

First we show that the intersection of an integral manifold of an involutive distribution with a flat chart is a countable union of open subsets of slices, where each of these open subsets of slices are embedded.
\begin{itemize}
    \item Let $D$ be an involutive distribution on $M$, set $(U,(x^i))$ be a flat chart for $D$. If $H$ is an integral manifold of $D$, then $H\cap U$ is a union of a countable number of disjoint open subsets of $k$-slices of $U$, each of which is open in $H$ and embedded in $M$.

    (These open subsets of slices can be chosen to be the connected components of $H\cap U$).

    This decomposition is essentially the decomposition of $H\cap U$ into its connected components.
    \begin{itemize}
        \item $H\cap U$ is a manifold (so 2nd-countable and locally connected) so it can be partitioned into a countable number of open connected components.

        Note: $H\cap U$ is open in $H$ since the inclusion $H\to M$ I is continuous.
        \item Let $V$ be one of those connected components, we show that $V$ must lie in a single $k$-slice $S$ of $U$ because $dx^1,...,dx^k$ vanish on $V$ (since they are local defining forms for $D$) and $V$ is connected.
        \item Remains to show that $V$ is embedded in $M$, as an open subset of $S$. This follows from the fact that $V\to S$ is an injective immersion between manifolds (w/o boundaries) of the same dimension. So it is a local diffeo, an open map and a homeomorphism (in fact diffeo) onto its image. So $V$ is open in $S$, $V\to S$ is an embedding and $V\to M$ is an embedding as a composition of embeddings $V\to S\to M$.
    \end{itemize}
\end{itemize}

An important consequence: integral manifolds of involutive distributions are weakly embedded (though they are not necessarily embedded, think of the Kronecker foliation of the torus).
\begin{itemize}
    \item Every integral manifold of an involutive distribution is weakly embedded.

    Reminder: a submanifold $S\subseteq M$ is weakly embedded if for any smooth map $F:N\to M$ whose image lies in $S$, $F:N\to S$ is smooth.

    Sketch of the proof:
    \begin{itemize}
        \item Given a smooth map $F:N\to M$ whose image lies in $H$, an integral mfd of the involutive distribution $D$, the idea is to find for any $q\in N$ an open nbd $B$ of $q$ whose image lies in an open $V$ of a $k$-slice $S$ of flat chart $U$ of $D$. Then we can conclude using the fact that such a $V$ will be open in $H$ and embedded in $S$.

        \item To do this, we use the fact that $F(q)$ lies in a flat chart $(U,(x^i))$ for $D$, and that $F^{-1}(U)$ is open in $N$ which allows us to find a connected open nbd $B$ of $q$ whose image by $F$ lies in $U$.
        \item We will be able to show that $B$ lies in a single $k$-slice of $U$. We would like to immediately conclude that $F(B)$ must be a connected subset of $H$, and so must lie in a single connected subset of $H\cap U$, but we do not yet know that $F:N\to H$ is continuous.

        Instead we proceed like this: we define $F^i:=x^i\circ F$. Since $H$ meets $U$ in a countable number of slices, for $1\leq i\leq k$ $F^i$ takes a countable number of values on $B$, and since $B$ is connected, $F^i(B)$ must be an interval (intermediate value theorem) so $F^i$ is constant on $B$. So we see that $F(B)$ lies in a single connected component $V$ of $H\cap U$, with $V$ open in $H$, and embedded as an open subset of single $k$-slice $S$ of $U$.
        \item We then conclude: $F:B\to M$ is smooth and $F(B)$ lies in $V$ which is embedded in $M$, so $F:B\to V$ is smooth, and since $V$ is open in $H$, $F:B\to H$ is smooth.
    \end{itemize}
\end{itemize}

\section{Foliations}

\subsection{Definition of Foliation}

We now investigate the global structure of maximal integral manifolds of an involutive distribution. While doing this, we introduce the notion of foliation and show that foliations correspond exactly to involutive distributions.

\begin{itemize}
    \item Given a collection $\mathcal{F}$ of $k$-submanifolds of $M$ (and no longer a distribution), a chart $(U,\phi)$ on $M$ is said to be flat for $\mathcal{F}$ if $\phi(U)$ is a cube and if each element of $\mathcal{F}$ intersects $U$ in either the empty set or a countable number \emph{complete} $k$-slices of $U$ (and no longer just open subsets of slices).
    \item A foliation of dimension $k$ on $M$ is defined as a collection $\mathcal{F}$ of disjoint, connected, nonempty, immersed $k$-dimensional submanifolds of $M$ (called leaves of the foliation) whose union is $M$ and such that in a nbd of each point of $M$ there exists a flat chart for $\mathcal{F}$.
\end{itemize}

\subsection{Maximal integral manifolds}

To properly state the correspondence between foliations and involutive distributions, we introduce the notion of maximal integral manifolds. The notion of maximal integral manifolds makes sense because connected integral manifolds behave under union like connected/path-connected subsets.
\begin{itemize}
    \item Let $D$ be an involutive distribution and $(N_\alpha)_\alpha$ be a collection of connected integral manifolds of $D$ with a point in common (ie nonempty intersection). Then $N:=\bigcup_\alpha N_\alpha$ has a unique smooth manifold structure making it into a connected integral manifold of $D$.

    About the proof:
    \begin{itemize}
        \item The uniqueness of the smooth mfd structure making $N$ into an integral mfd of $D$ follows immediately from the fact that an integral mfd of an involutive distribution is weakly embedded (so if we denote by $N_1$, $N_2$ $N$ with 2 mfd structures making it into an integral mfd of $D$, the set identity maps $N_1\to N_2$, $N_2\to N_1$ would be smooth - since $N_1,N_2\to M$ are smooth, and mutually inverse, making them diffeos).

        \item More involved is the proof of the existence of such a smooth mfd structure. Most tedious is the proof of 2nd-countability.
    \end{itemize}
    \item As a consequence of this given an involutive distribution $D$ on $M$, $M$ partitions into the collection $\mathcal{F}$ of maximal connected integral mfds for $D$.
    \begin{itemize}
        \item A maximal connected integral mfd of $D$ is a connected integral mfd $H$ of $D$ such that if $L$ is any other connected integral mfd of $D$ and if $L\cap H$ is nonempty, then $L$ is included in $H$.

        In particular this means that if $H'$ is a connected integral manifold of $D$ that contains $H$, $H'=H$.
        \item The partition $\mathcal{F}$ corresponds to the equivalence relation on $M$ defined by: $x\sim y$ iff there exists a connected integral mfd of $D$ that contains both $x$ and $y$.
        \item Given a point $p$ in $M$, the equivalence class of $p$ ie the maximal connected integral manifold of $D$ containing $p$ can be constructed as the union of all connected integral mfds of $D$ that contain $p$.
    \end{itemize}
\end{itemize}

\subsection{Equivalence between foliations and involutive distrubutions}

\begin{itemize}
    \item (Global Frobenius) There is a 1:1 correspondence between foliations of $M$ and involutive distributions on $M$. More precisely:
    \begin{itemize}
        \item Given a foliation $\mathcal{F}$ of $M$, we define a distribution $D$ as follows: for $p\in M$, $D_p:=T_pH$ where $H$ is the unique element of $\mathcal{F}$ containing $p$.

        Then $D$ is an involutive distribution, and the leaves of $\mathcal{F}$ are the maximal integral manifolds of $D$.
        \item (The more difficult direction) Given an involutive distribution $D$ on $M$, its maximal integral mfds foliate $M$
    \end{itemize}
\end{itemize}

Also very useful is the local characterization of the diffeos of $M$ that preserve a foliation on $M$.
\begin{itemize}
    \item Let $\Phi:M\to M$ be a diffeo.
    \begin{itemize}
        \item A distribution $D$ on $M$ is $\Phi$-invariant if $D_{\Phi(x)}=d\Phi_xD_x$ for every $x\in M$.
        \item A foliation $\mathcal{F}$ is $\Phi$-invariant if for each leaf $L$ of $\mathcal{F}$, $\Phi(L)$ is also a leaf of $\mathcal{F}$.
    \end{itemize}
    \item The result is the following: Let $\mathcal{F}$ be the foliation corresponding to the involutive distribution $D$. Then $\mathcal{F}$ is $\Phi$-invariant iff $D$ is $\Phi$-invariant.
\end{itemize}



\chapter{Lie groups}

\section{Basic Definitions and Properties}

\subsection{Lie group morphisms}
[Lee]

Needed Reminder: Rank

\begin{itemize}
    \item Lie group morphisms have constant rank
    \item A Lie group morphism is an iso iff it is bijective. This follows from the global rank theorem and the previous point.
\end{itemize}

\subsubsection{Universal covering group}

\begin{itemize}
    \item Given connected lie group $G$, there exists a unique Lie group morphism $\pi:G'\rightarrow G$ with $G'$ simply connected [up to equivalence].

    $G'$ is constructed as follows: we take the universal cover $\pi: G'\rightarrow G$ of $G$, then choose $e'$ in the preimage of $e$, which will serve as the identity of $G'$. Then using $e'$. We also have to notice that $\pi\times\pi:G'\times G'\rightarrow G\times G$ is a covering. 
    
    Then, using $e'$ and $(e',e')$ as reference points, we can use the unique lifting property of covering spaces to lift the multiplication and inverse maps (we automatically get smoothness and (most of the) uniqueness). 
    
    Then we only have to show that the lifted operations verify the group axioms. This is done using the uniqueness part of the lifting property.
\end{itemize}

\subsection{Lie Subgroups}

[Lee]

Definition and proposition having to do with the subtle difference between an embedded submanifold and an immersed submanifold.

\begin{definition}[Lie Subgroup]
    A Lie subgroup of $G$ is a subgroup $H$ endowed with a smooth structure (so a topology as well) making it into a Lie group and an immersed submanifold.
\end{definition}

Needed Reminder: Embedded vs Immersed submanifolds

Embedded subgroups are automatically Lie subgroups.
\begin{itemize}
    \item A subgroup $H$ of $G$ that is also an embedded submanifold is a Lie subgroup.

    Note: An embedded submanifold has a unique topology + smooth structure making it into an immersed submanifold [check]. This would explain the omission of the topology+smooth structure in the previous statement.
\end{itemize}

An open subgroup is closed (so clopen).
\begin{itemize}
    \item Let $H$ be an open Lie subgroup of $G$ then $H$ is also closed.

    To see this, notice that every left coset of $H$ (so every element of $G/H$) is open.
\end{itemize}

\subsubsection{Identity Component}

\begin{definition}
    We call $G_0$ the connected component of $G$ containing the identity.
\end{definition}

The properties of $G_0$:
\begin{itemize}
    \item $G_0$ is an open subgroup generated by any connected neighbourhood of the identity.

    Seeing this requires a few steps. Let $W$ be a connected open nbd of the identity and let $H$ be the subgroup generated by $W$.
    \begin{itemize}
        \item $H$ is open, since for any $h$ in $H$, $H$ contains $hW$ which is an open nbd of $h$ (since left translation is a diffeo). In particular, H is clopen (since an open Lie subgroup is closed).
        \item $H$ is connected. To see this, we can proceed by induction [we will follow a path that is less systematic than Lee's but arguably more intuitive]. Let $\gamma$ be a path in $H$ joining $e$ to $h\in H$. Then we can obtain a path in $H$ joining $e$ and $wh$ for any $w\in W_1:=W\cup W^{-1}$ in the following way:
        \begin{itemize}
            \item Before anything, notice that $W_1$ is open and connected. Indeed $W^{-1}$ is connected and open as the image of $W$ by the inversion map (which is a diffeo). So $W_1$ is connected and open as the union of two connected opens which share the identity.
            \item Since $wH\subseteq H$, $w\gamma$ will be a path in $H$ joining $w$ and $wh$.
            \item Since $W_1$ is connected and open, so in particular path-connected (since a mfd is locally path-connected) there exists a path $\delta$ in $W_1$ joining $e$ and $w$. We see that the concatenation $\delta.w\gamma$ is a path in $H$ joining $e$ and $wh$
        \end{itemize}

        We thus see that $H$ is path-connected and in particular connected. Since $H$ contains $e$, $H$ must be a subset of $G_0$.
        \item Since $H$ is a nonempty clopen subset of $G_0$, which is connected, $H$ is equal to $G_0$, so $G_0$ really is a subgroup, generated by any connected nbd of the identity.
    \end{itemize}

    \item $G_0$ is a normal subgroup.

    Given an element $g$ of $G$, the map $c_g:G\rightarrow G,x\rightarrow gxg^{-1}$ is a diffeo (in fact it is a Lie group automorphism), so $gG_0 g^{-1}$ is connected and contains the identity, and must thus be a subset of $G_0$.

    \item G verifies the second countability requirement iff the quotient group $G/G_0$ is countable. For a Lie group $G$ (verifying that countability requirement), $G/G_0$ will be a discrete (so countable) Lie group [in fact $G\rightarrow G/G_0$ is the universal discretification of $G$].

    \item Any connected component of $G$ is diffeomorphic to $G_0$. In fact the connected components of $G$ are the cosets of $G_0$ ($G_0$ is normal so that its left cosets are its right cosets).
\end{itemize}

\subsubsection{The rank again}

Some consequences of the rank theorem and the fact that a Lie group morphism has constant rank.

\begin{itemize}
    \item The kernel of a Lie group morphism $F$ is a properly embedded Lie subgroup whose codimension is equal to the rank of $F$.

    Note: Once we understand quotients of Lie groups better, we'll see that a surjection of Lie groups is a principal fiber bundle.
    \item Given an injective Lie group morphism $F:G\rightarrow H$, the image $F(G)$ admits a unique topology and smooth structure st $F(G)$ is a Lie subgroup and $F:G\rightarrow F(G)$ is an iso of Lie groups.
\end{itemize}

A quirky example of Lie group
\begin{itemize}
    \item We can obtain a dense Lie subgroup of the torus (irrational stuff) through an injective Lie group morphism $\mathbb{R}\rightarrow \mathbb{T}^2$
\end{itemize}

\subsubsection{For Lie subgroups closed=embedded}

We know that some closed submanifolds are not embedded (the figure-eight) and that embedded submanifolds are not always closed (an open subset of $\mathbb{R}^n$). For Lie subgroups however the two notions are equivalent.
\begin{itemize}
    \item A Lie subgroup is closed iff it is embedded.

    The proof is rather involved. Go over it at some point in the future maybe.

    Note: This means that embedded Lie subgroups are always properly embedded.

    Note: Once we get the closed subgroup theorem, we'll see that a ('algebraic') subgroup of a Lie subgroup is a properly embedded Lie subgroup iff it is closed.
\end{itemize}

\section{Actions and equivariant maps}

\begin{itemize}
    \item The definition of smooth left and right actions.
    \item Basic notions of actions:
    \begin{itemize}
        \item Orbit.
        \item Stabilizer/Isotropy subgroup.
        \item Transitive action (a single orbit) and free action (stabilizers are trivial).
    \end{itemize}
    \item Definition of equivariant map when we have actions on two spaces for the same group.

    \item Rank properties of equivariant maps with transitive domain: Let $F:M\to N$ be an equivariant map between two mfds $M$, $N$ equipped with $G$-actions. If the action of $M$ is transitive, then $F$ has constant rank.

    In particular
    \begin{itemize}
        \item If $F$ is injective, $F$ is an immersion.
        \item If $F$ is surjective, $F$ is a submersion.
        \item If $F$ is bijective, $F$ is a diffeomorphism.
    \end{itemize}

    \item Properties of the orbit map: let $M$ be a smooth mfd admitting a $G$ action, and take $p\in M$. $\theta^p:G\to M, g\to g.p$ has constant rank (it is equivariant for the transitive action of $G$ on itself via left-translations) so 
    \begin{itemize}
        \item $G_p$, the stabilizer of $p$, is a properly embedded Lie subgroup.
        \item $G_p$ is trivial iff $\theta^p$ is injective, in which case $p$'s orbit $G.p$ is an immersed submanifold diffeo to $G$.
    \end{itemize}
\end{itemize}

Covering spaces as an important source of smooth actions of discrete Lie groups.

\begin{itemize}
    \item We are given a smooth covering map $\pi:E\to M$ between connected mfds (with or without boundary). An element of $Aut_\pi(E)$ is a diffeo $\phi:E\to E$ st $\pi\phi=\pi$ (imagine commuting triangle).

    Question: If we find a homeo $E\to E$ that preserves $\pi$, will it be a diffeo?
    \item It can be shown that $Aut_\pi(E)$ acts transitively on the fibers of $\pi$ (which are invariant for the action of $Aut_\pi(E)$ iff $\pi$ is a normal covering, meaning $\pi_*\pi_1(E,e_0)$ will be a normal subgroup of $\pi_1(M,p_0)$ (for $p_0=\pi(e_0)$).

    Idea for showing one direction:
    \begin{itemize}
        \item Take $p_0\in M$ and consider the fiber $\pi^{-1}(p_0)$ above $p_0$.
        \item Since $E$ is connected, the $\pi_*\pi_1(E,e_0)$ form a conjugacy class of subgroups of $\pi_1(M,p_0)$ as $e_0$ ranges over the fiber.
        \item Assume that there exists an element $\phi$ of $Aut_\pi(E)$ sending $e_0$ to $e_1$ (two elements of the fiber). Then $\pi_*\pi_1(E,e_0)=\pi_*\phi_*\pi_1(E,e_0)=\pi_*\pi_1(E,e_1)$ (where we used $\phi_*\pi_1(E,e_0)=\pi_1(E,e_1)$ since $\phi$ is a homeo).
        \item Thus we see that is $Aut_\pi(E)$ acts transitively on the fiber, then $\pi_*\pi_1(E,e_0)$ is the only element of its conjugacy class and is thus normal.
    \end{itemize}

    Note: the converse by using the lifting shenanigans, by looking at morphisms of fundamental groups?

    \item Main property is the following: with the discrete topology, $Aut_\pi(E)$ acts smoothly and freely on $E$

    Note: If $Aut_\pi(E)$ also acts transitively (normal covering), do we have $M=E/Aut_\pi(E)$?

    Some ideas in the proof:
    \begin{itemize}
        \item Since $Aut_\pi(E)$ is discrete, smoothness of the action simply follows from the fact that an element of $Aut_\pi(E)$ is a diffeo of $E$.
        \item The action being free follows from the uniqueness part of the lifting property for covering maps.
        \item To show that $Aut_\pi(E)$ is a Lie group, we need to show that it's countable. This will follow from second countability of $E$ and the action being free.
        \begin{itemize}
            \item Given $p\in M$, the fiber above $p$ is countable.

            Indeed, take $U\subseteq M$ be an evenly covered open nbd of $p$. Since $E$ is second-countable, $\pi^{-1}(U)$ admits a countable number of components, and each component contains exactly one element of the fiber above $p$.
            \item Since $Aut_\pi(E)$ acts freely on $E$ and preserves fibers, we have an injection of $Aut_\pi(E)$ into the fiber above $p$, and $Aut_\pi(E)$ must be countable.
        \end{itemize}
    \end{itemize}
\end{itemize}

\subsection{Semidirect products}

Construction of Lie groups through smooth actions.

\begin{itemize}
    \item Definition of semidirect product: Given an action $\theta$ of $H$ on $N$ (two Lie groups - in particular, $\theta_h:N\to N$ is a Lie group iso for any $h\in H$), then the product mfd $N\times H$ equipped with the multiplication $(n,h)(n',h')=(n\theta_h(n'),hh')$ is a Lie group. We denote it by $N\rtimes_{\theta}H$ and called a semidirect product of $N$ and $H$.

    \item Characterization of semi-direct product, internal semi-direct product (standard shape of a semi-direct product).

    \begin{itemize}
        \item Let $G:=N\rtimes_\theta H$ be a semi-direct product then
        \begin{itemize}
            \item $N':=N\times\{e\}$, $H':=\{e\}\times H$ are closed Lie subgroups of G, isomorphic to $N$ and $H$ respectively in the obvious way.
            \item $N'$ is a normal subgroup of $G$.
            \item $N'\cap H'=\{e\}$, and $N'H'=G$ (with $AB:=\{ab|a\in A,b\in B\}$.
            \item For $h\in H$, $n\in N$, $(e,h)(n,e)(e,h^{-1})=(\theta_h(n),e)$, so we see that the action of $H$ on $N$ becomes the conjugate action of $H'$ on $N'$ in $G'$
        \end{itemize}
        \item Given $G$ a Lie subgroup and $N$,$H$ two closed (so properly embedded) Lie subgroups of $G$ such that:
        \begin{itemize}
            \item ($N$, $H$ are a "decomposition" of $G$) $N\cap H=\{e\}$, $NH=G$.
            \item $N$ is normal subgroup of $G$.
        \end{itemize}

        Then the map $N\times H\to G,(n,h)\to nh$ is a Lie group iso between $G$ and the semidirect product $N\rtimes_\theta H$ where the action $\theta$ is given by conjugation ie $\theta_h(n)=hnh^{-1}$ for $h\in H$ and $n\in N$.

        The way to see this is the following:
        \begin{itemize}
            \item $(n,h)\to nh$ is surjective iff $NH=G$ and injective iff $N\cap H=\{e\}$ so under our hypotheses, it is a bijection.
            \item Thus we can make $N\times H$ into a group by pulling back the product of $G$ along the bijection. Using the fact that $N$ is normal, we can check that the resulting product is exactly that of the semidirect product given above.
            \item To wrap everything up, notice that we have constructed a smooth bijective group morphism between two Lie groups, so it must be an iso.
        \end{itemize}
        \item Note: We see that semidirect products are "what emerges naturally" when we attempt to break a group down into smaller components (in particular for non-abelian groups, the direct product is not very natural). We also see that any action of a Lie group another may be represented by the conjugation action.

        In category-theoretic language, this translates to the fact that semidirect products of (Lie) groups correspond to right-split short exact sequences (check)
    \end{itemize}
\end{itemize}

\section{Lie Algebra of a Lie Group}

\subsection{Definitions}

\begin{itemize}
    \item The space of left-invariant vector fields on a Lie Group $G$ equipped with the Lie bracket (of vector fields) is a finite-dimensional Lie algebra. We call it the Lie algebra of $G$, and denote it by $Lie(G)$ (sometimes $\mathfrak{g}$).

    \begin{itemize}
        \item We show that it is a Lie algebra by showing that the Lie bracket of two left-invariant vector fields is left-invariant (which follows by naturality of the Lie bracket).
        \item We show that it is finite-dimensional by showing that the evaluation map at the identity: $Lie(G)\to T_eG,X\to X_e$ is an isomorphism of vector spaces.

        Note: We see that any rough left-invariant vector field is smooth.
        Note: We can use that iso to transfer the Lie bracket onto $T_eG$.
        Note: We have an interpretation of $Lie(G)$ as an "infinitesimal version" of $G$.
    \end{itemize}
\end{itemize}

An important facts:
\begin{itemize}
    \item Any Lie group admits a left-invariant smooth global frame and is therefore parallelizable.
    \item Any left-invariant vector field on a Lie group is complete. This follows from the uniform time lemma cf \ref{LeftInvariantComplete}.
\end{itemize}

\subsection{Induced Lie algebra morphism}

A morphism of Lie groups induces a morphism in their Lie algebras, which can be understood as the differential at the identity.
\begin{itemize}
    \item Let $F:G\to H$ be a morphism of Lie groups. For $X\in Lie(G)$, there is a unique element of $Lie(H)$ which is $F$-related to $X$, and we denote this element by $F_*X$.
    \begin{itemize}
        \item The map $F_*:Lie(G)\to Lie(H)$ is a morphism of Lie algebras.
        \item The relation $F\to F_*$ is functorial (so we get a functor from the category of Lie groups to the category of Lie algebras). In particular isomorphic Lie groups have isomorphic Lie algebras.
    \end{itemize}

    $F_*X$ is characterised by $(F_*X)_e=dF_eX_e$ - then we only have to show that the element of $Lie(G)$ corresponding to $X_e$ is $F$-related to the element of $Lie(H)$ corresponding to $dF_eX_e$, we use left-invariance and the fact that $F$ is a morphism of Lie groups - in that sense, $F_*$ can be understood as the differential of $F$ at the identity.
\end{itemize}

\subsection{Lie algebra of a Lie subgroup}

\begin{itemize}
    \item Let $H$ be a Lie subgroup of $G$ with inclusion map $\iota:H\to G$. Then there is a subalgebra $\mathfrak(h)$ of $Lie(G)$ which is canonically isomorphic to $Lie(H)$ and which is characterized in either of the 2 following ways
    \begin{equation*}
        \mathfrak{h}=\{X\in Lie(G)|X_e\in T_eH\}=\iota_*Lie(H)
    \end{equation*}

    Note: this works because of the fact that for two vector fields $X,Y$ tangent to an immersed submanifold $S$, $[X,Y]$ is also tangent to $S$.
\end{itemize}

We also have Ado's theorem, telling us that any finite-dimensional real algebra is isomorphic to a subalgebra of matrices (of some finite rank).
\begin{itemize}
    \item Any finite-dimensional real Lie algebra admits a faithful finite-dimensional representation. 
\end{itemize}

In fact Ado proved this theorem for finite-dimensional algebras over any field of characteristic zero, and Kenkichi Iwasawa removed the restriction on the characteristic.

\subsection{Foliations, Lie subalgebra and Lie subgroups}

The core is the following: a Lie subalgebra of $Lie(G)$ induces a left-invariant involutive distribution on $G$, and a Lie subgroup $H$ is an integral manifold of the distribution induced by its Lie algebra $Lie(H)$ (seen as a subalgebra of $Lie(G)$). The two main consequences this has for us, is that a Lie subgroup will always be weakly embedded and the fact that any Lie subalgebra of $Lie(G)$ induces a connected Lie subgroup by the global Frobenius theorem.
\begin{itemize}
    \item A distribution $D$ on a Lie group $G$ is left-invariant if it is invariant under left translations, so $L_g$-invariant for each $g\in G$.
    \item Involutive left-invariant distributions on $G$ correspond exactly to subalgebras of $Lie(G)$ in the following way: a Lie subalgebra $\mathfrak{h}$ if $Lie(G)$ corresponds to the distribution $D$ defined by $D_g:=\{X_g|X\in \mathfrak{h}\}$.
    \item A Lie subgroup $H$ of $G$ is an integral manifold of the left-invariant distribution induced by the subalgebra $\mathfrak{h}=\iota_*Lie(H)$ of $Lie(G)$.

    Consequently $H$ is weakly embedded in $G$.
    \item Given a Lie subalgebra $\mathfrak{h}$ of $Lie(G)$, there exists a unique connected Lie subgroup of $G$ whose Lie algebra is $\mathfrak{h}$, which is the leaf of the distribution $D$ induced by $\mathfrak{h}$ containing the identity $e$.

    \begin{itemize}
        \item We show that the leaf $H$ going through $e$ is an algebraic subgroup using the fact that $D$ is left-invariant:
        \begin{itemize}
            \item We use $\mathcal{H}_g$ to denote the leaf going through $g\in G$. Since $D$ is left-invariant, we have $L_g\mathcal{H}_{g'}=\mathcal{H}_{gg'}$.

            And we have defined $H:=\mathcal{H}_e$.
            \item For $h\in H$ we have $H=\mathcal{H}_h=L_h\mathcal{H}_e=L_hH$ so that $H$ is closed under the multiplication.
            \item For $h\in H$ we have $L_{h^{-1}}H=L_{h^{-1}}\mathcal{H}_{h}=\mathcal{H}_e=H$, so $L_{h^{-1}}$ sends $H$ to $H$ which implies that $h^{-1}\in L_{h^{-1}}\mathcal{H}_e=H$.

            So $H$ is closed under inversion.
        \end{itemize}
        \item The fact that $H$ is a Lie subgroup follows from the fact that it is weakly embedded. The multiplication $m:H\times H\to G$ and the inversion $i:H\to G$ are smooth as composition of smooth maps, and their images lie in $H$, since $H$ is weakly embedded in $G$, they are smooth as maps into $H$.
        \item On the other hand, if $H'$ is another connected Lie subgroup whose Lie algebra is $\mathfrak{h}$ then $H'$ is a connected integral mfd of $D$, so $H'$ is an open subgroup of $H$. As we've already seen, an open Lie subgroup is closed, so $H'$ is both open and closed in $H$ (also nonempty) so $H'=H$, and we obtain uniqueness.
    \end{itemize}
\end{itemize}

\section{Exponential Map}

\subsection{One-Parameter Subgroups and Exponential map}

\begin{itemize}
    \item A one-parameter subgroup of a Lie group $G$ is a Lie group homomorphism $\gamma:\mathbb{R}\to G$.
    \item The one parameter subgroups of $G$ are exactly the maximal integral curves of left-invariant vector fields starting at the identity.

    In that way there is a 1:1 correspondence between one-parameter subgroups of $G$ and elements of $Lie(G)$.

    Note: we already showed that left-invariant vector fields are complete (cf \ref{LeftInvariantComplete}).
    \begin{itemize}
        \item Given a left-invariant vector field $X$ and $\gamma:\mathbb{R}\to G$ the maximal integral curve of $X$ starting at the identity, for $s\in\mathbb{R}$, since $X$ is left-invariant, $t\to L_{\gamma(s)}\gamma(t)=\gamma(s)\gamma(t)$ is an integral curve of $X$ starting at $\gamma(s)$, but so is $t\to \gamma(s+t)$ by uniqueness of integral curves we get $\gamma(s)\gamma(t)=\gamma(s+t)$ for all $s,t\in\mathbb{R}$ so $\gamma$ is indeed a one-parameter subgroup.
        \item Given a one parameter subgroup $\gamma:\mathbb{R}\to G$, one can check that $\partial_t$ is $\gamma$-related to $X:=\gamma_*\partial_t$ (the Lie algebra morphism $\gamma_*:\mathbb{R}\to Lie(G)$, with $\partial_t$ the canonical left-invariant vector field on $\mathbb{R}$), so that $\gamma$ is the integral curve of $X$.
    \end{itemize}
    \item One-parameter Lie subgroups of a Lie subgroup: given a Lie subgroup $H$ of $G$, the one-parameter subgroups of $H$ are exactly the one-parameter subgroups of $G$ whose initial velocities lie in $T_eH$.
    \begin{itemize}
        \item Given a one-parameter subgroup of $H$, $\gamma:\mathbb{R}\to H$, since $H\to G$ is a Lie group morphism, $\gamma:\mathbb{R}\to G$ is also a one-parameter subgroup of $G$, and clearly its initial velocity lies in $T_eH$.
        \item Conversely, given a one-parameter subgroup $\gamma:\mathbb{R}\to G$ whose initial velocity lies in $T_eH$. Let $\tilde{\gamma}$ be the one-parameter subgroup of $H$ with the same initial velocity as $\gamma$. As seen in the previous point, $\tilde{\gamma}:\mathbb{R}\to G$ is a one-parameter subgroup with the same initial velocity as $\gamma$, so they must be equal.
    \end{itemize}
\end{itemize}

Exponential map:
\begin{itemize}
    \item We define $\exp:Lie(G)\to G$ by $\exp(X):=\gamma_X(1)$ where $\gamma_X$ is the one-parameter subgroup of $G$ induced by $X$.
    \item $\mathbb{R}\to G,s\to \exp(sX)$ is the one-parameter subgroup generated by $X$.
    \item Properties of the exponential map
    \begin{itemize}
        \item $\exp$ is smooth.
        \item $\exp((s+t)X)=\exp(sX)\exp(tX)$ but in general $\exp(X+Y)\neq\exp(X)\exp(Y)$. As a consequence:
        \begin{itemize}
            \item $\exp(-X)=\exp(X)^{-1}$.
            \item $\exp(nX)=\exp(X)^n$.
        \end{itemize}
        \item The differential $(d\exp)_0:T_0\mathfrak{g}\simeq\mathfrak{g}\to\mathfrak{g}\simeq T_eG$ is the identity map. Consequently, by the local inversion theorem:
        \item $\exp$ restricts to a diffeomorphism from some neighbourhood of $0$ in $\mathfrak{g}$ to some neighbourhood of the identity in $G$.
        \item Given a morphism of Lie groups, $\Phi:G\to H$ we have $\Phi(\exp(X))=\exp(\Phi_*X)$ for any $X\in Lie(G)$. This follows from the fact that $X$ is $\Phi$-related to $\Phi_*X$ and the integral curve characterization of one-parameter subgroups.
        \item The flow $\theta$ of the left-invariant vector field $X$ is given by $\theta_t=R_{\exp(tX)}$.

        Note: It's because the flow of a left-invariant vector field is expressed via right-translation that the actions inducing a morphism of Lie algebras (and not an antimorphism) are the RIGHT actions.

        In fact, by our conventions (ie choosing left-invariant vector fields to construct the Lie algebra), we should think of a Lie group $G$ as a homogenous $G$ space for the natural right action of $G$ on itself.

        Then, for a right action $\theta$ of $G$ on $M$, and $p\in M$, we should $\theta^{(p)}:G\to M,g\to pg$ as an equivariant map (coherent with the fact that for $X\in Lie(G)$, $X$ will be $\Phi$-related to $\hat{X}$).
    \end{itemize}
\end{itemize}

Characterization of the Lie subalgebra of a Lie subgroup via the exponential map:
\begin{itemize}
    \item Given a Lie subgroup $H$ of $G$, if we consider $Lie(H)$ as a subalgebra of $Lie(G)$, then the exponential map of $H$ is simply the exponential map of $G$ restricted to $H$ and we have $Lie(H)={X\in Lie(G)|\exp(tX)\in H \forall t\in\mathbb{R}\}$.
    \begin{itemize}
        \item The fact that the exponential map of $H$ is simply the restriction of the exponential map of $G$ to $Lie(H)$ follows from the fact that the one-parameter subgroups of $H$ are the one-parameter subgroups of $G$ with initial velocity lying in $T_eH$.
        \item It follows that for $X\in Lie(H)$, $\exp(tX)$ will belong to $H$ for any $t\in\mathbb{R}$.
        \item Conversely, assume that for $X\in Lie(G)$, the one-parameter subgroup $\gamma_X:\mathbb{R}\to G,t\to\exp(tX)$ takes value in $H$, then since $H$ is weakly embedded, $\gamma_X:\mathbb{R}\to H$ is smooth and $X_e=\dot{\gamma}_X(0)$ lies in $T_eH$.
    \end{itemize}
\end{itemize}

\subsection{Closed subgroup theorem}

Pretty involved, proved using the exponential map.

\subsection{Infinitesimal Generators of Group Actions (Foliations)}

Because of our chosen conventions, ie defining $Lie(G)$ as the space of left-invariant vector fields, infinitesimal generators are most naturally defined for right-actions - essentially because the flow of a left-invariant vector field on $G$ is expressed via right-translation.

\begin{itemize}
    \item (Fundamental vector fields and the infinitesimal generator of our action) Given a right action $\theta$ of a Lie group $G$ on a mfd $M$, the fundamental vector field $\hat{X}$ on $M$ induced by $X\in Lie(G)$ is the field associated to the flow $(t,p)\to p\exp(tX)$ on $M$.

    We use $\hat{\theta}$ to denote the map $Lie(G)\to\mathfrak{X}(M),X\to\hat{X}$, and call this map the infinitesimal generator of $\theta$.

    Note: clearly fundamental vector fields are complete.
    \item The fundamental vector field $\hat{X}$ associated to $X\in Lie(G)$ can also be characterized as follows: $\hat{X}$ is the unique vector field on $M$ that is $\theta^{(p)}$-related to $X$ for each $p\in M$, with $\theta^{(p)}:G\to M,g\to pg$.

    This follows from the fact that $\theta^{(p)}$ maps the flow of $X$ (given by $(t,g)\to g\exp(tX)$ onto the flow of $\hat{X}$. The uniqueness follows from the fact that any point of $M$ lies in the image of a $\theta^{(p)}$.
    \item The infinitesimal generator $\hat{\theta}:Lie(G)\to\mathfrak{X}(M)$ is Lie algebra morphism.

    This follows from the naturality of the Lie bracket and the preceding characterization of fundamental vector fields. Indeed, given $X,Y\in Lie(G)$, for any $p\in M$, since $\hat{X},\hat{Y}$ are $\theta^{(p)}$-related to $X$ and $Y$ resp., by naturality of the Lie bracket, $[\hat{X},\hat{Y}]$ is $\theta^{(p)}$-related to $[X,Y]$. Since this is valid for any $p\in M$, we conclude $\hat{[X,Y]}=[\hat{X},\hat{Y}]$.
\end{itemize}

Now the most important theorem of this section, which we deduce from Frobenius' theorem. It tells us that the datum of an infinitesimal Lie action of $G$ on $M$ - a Lie algebra morphism $Lie(G)\to\mathfrak{X}(M)$ with a completeness requirement - will induce a right action of $G$ on $M$.

\begin{itemize}
    \item Given a Lie algebra $\mathfrak{g}$ and a mfd $M$. A right action of $\mathfrak{g}$ on $M$ is a Lie algebra morphism $\hat{\theta}:Lie(G)\to\mathfrak{X}(M)$. 
    
    It is said to be complete if $\hat{\theta}(X)$ is complete for any $X\in Lie(G)$.
    \item (Fundamental theorem on Lie algebra actions) Let $M$ be a smooth mfd and $G$ be a \emph{simply-connected} Lie group. Given a complete right action $\hat{\theta}$ of $Lie(G)$ on $M$, there exists a unique right action $\theta$ of $G$ on $M$ whose infinitesimal generator is $\hat{\theta}$.

    \begin{itemize}
        \item At the core of the proof given in the Lee, we construct $\theta$ by first constructing its orbits as leaves of the involutive distribution given by the image of $\hat{\theta}$.

        The simple-connectedness of $G$ intervenes via the theory of coverings.
        \item A more handsy approach, with an idea as to why simple-connectedness of $G$ matters: 
        \begin{itemize} 
            \item To find the action of $g\in G$ on $M$, we write $g=\exp(X_1)...\exp(X_n)$ for $X_1,...,X_n\in Lie(G)$ which is always possible since $G$ is connected. 
            
            THEN if $\hat{\theta}$ is to be the infinitesimal generator of $\theta$, we must have for $p\in M$, $pg=p\exp(\hat{X}_1)...\exp(\hat{X}_n)$. 
            
            Here we use a particular notation for the flow of $\hat{X}$ (for $X\in Lie(G)$), which is assumed to be complete, we denote it by $(t,p)\to p\exp(t\hat{X})$, in particular, we use $p\exp(\hat{X})$ to denote the image of $(t=1,p)$ by the flow of $\hat{X}$.

            This already gives us uniqueness of $\theta$.
            \item The simple-connectedness of $G$ allows us to show that the action of $g$ we just computed does not depend on the choice of decomposition $g=\exp(X_1)...\exp(X_n)$.

            The idea is to see the decomposition $\exp(X_1)...\exp(X_n)$ as a path joining the identity $e$ to $g$, and to use the fact that any two path joining $e$ to $g$ are homotopic.
            \item Once we have independence of the choice of decomposition, we can show that the resulting $\theta$ is indeed a right-action by concatenating decompositions.
        \end{itemize}       
    \end{itemize}
\end{itemize}

\subsection{Lie correspondence}

The fundamental theorem on Lie algebra actions is very fruitful when applied to morphism of Lie algebras.

\begin{itemize}
    \item Let $G$, $H$ be Lie groups with $G$ simply connected. Then for any Lie algebra morphism $\phi:Lie(G)\to Lie(H)$ there is a unique Lie group morphism $\Phi:G\to H$ such that $\phi=\Phi_*$.

    The idea is that $\phi:Lie(G)\to Lie(H)$ is a complete right action of $Lie(G)$ on $H$ which induces a right action $\theta$ of $G$ on $H$, and we set $\Phi(g)=\theta(e_H,g)=e_Hg$. The fact the infinitesimal generator $\hat{\theta}=\phi$ lands in left-invariant vector fields will allow us to show that $\Phi$ is a morphism of Lie groups.
    \item A corollary: if $G$ and $H$ are simply connected Lie groups with isomorphic Lie algebras, then they are isomorphic.

    In fact, as we'll see we have a true equivalence of categories between the category of Lie algebras and the category of simply connected Lie groups.

    \item (Lie Correspondence) There is a 1:1 correspondence between isomorphism classes of finite-dimensional Lie algebras and isomorphism classes of simply connected Lie groups.

    The only thing left to show is the fact that any finite-dimensional Lie algebra $\mathfrak{g}$ there exists a simply connected Lie group $G$ whose Lie algebra is isomorphic to $\mathfrak{g}$. This is shown using Ado's theorem, the correspondence between connected Lie subgroups and Lie subalgebras and the theory of coverings for Lie groups.

    \begin{itemize}
        \item By Ado's theorem, we can find a natural $n$ and an embedding (of Lie algebras) of $\mathfrak{g}$ into $gl(n,\mathbb{R})$ (allowing us to identify $\mathfrak{g}$ with a subalgebra of $gl(n,\mathbb{R})$).
        \item Then, there exists a unique connected Lie subgroup $G'$ of $Gl(n,\mathbb{R})$ whose Lie algebra is $\mathfrak{g}$.
        \item Finally we obtain $G$ as the universal cover of $G'$.
    \end{itemize}
\end{itemize}

\section{The adjoint representation}

There is a natural action of a Lie group $G$ on its Lie algebra $Lie(G)$, the adjoint action, which induces the natural action of $Lie(G)$ on itself, which is also called the adjoint representation.

\begin{itemize}
    \item For $g\in G$, we define $Ad(g):=(C_g)_*:Lie(G)\to Lie(G)$, with $C_g$ the group morphism $G\to G,h\to ghg^{-1}$ and $(C_g)_*$ the induced Lie algebra morphism $Lie(G)\to Lie(G)$.

    The resulting map $Ad:G\to Gl(\mathfrak{g})$ is a smooth linear representation of $G$.
    \begin{itemize}
        \item The smoothness can be shown as follows: define $C:G\times G\to G, (g,h)\to C_gh=ghg^{-1}$, then $(Ad(g)X)_e=(dC)_{(g,e)}(0,X_e)$ and since $dC:T(G\times G)\to TG$ is smooth, we conclude that $Ad$ is smooth.
    \end{itemize}
    \item For a Lie algebra $\mathfrak{g}$ we define the adjoint action $ad:\mathfrak{g}\to gl(\mathfrak{g})$ by $ad(X)Y:=[X,Y]$. The Jacobi identity implies that $ad$ is a morphism of Lie algebras.
    \item We can show that for a Lie group G, $ad=Ad_*$ with $Ad_*:\mathfrak{g}\to gl(\mathfrak{g})$ the Lie algebra morphism induced by the algebra morphism $Ad:G\to Gl(\mathfrak{g})$.

    One way of showing this:
    \begin{itemize}
        \item For $X,Y\in Lie(G)$, $(Ad_*(X)Y)_e=\frac{d}{dt}|_{t=0}dC_{\exp(tX)}Y_e$.
        \item Since we have $C_g=R_{g^{-1}}L_g$ we obtain $(Ad_*(X)Y)_e=\frac{d}{dt}|_{t=0}R_{\exp(-tX)}L_{\exp(tX)}Y_e=\frac{d}{dt}|_{t=0}R_{\exp(-tX)}Y_{\exp(tX)}$ since $Y$ is left invariant, so $(Ad_*(X)Y)_e=\frac{d}{dt}|_{t=0}d(\theta_{-t})Y_{\theta_te}$ where $\theta$ is the flow of $X$.
        \item Using the flow characterization of the Lie bracket of vector fields, we conclude $(Ad_*(X)Y)_e=[X,Y]_e$
    \end{itemize}
\end{itemize}

Another way of seeing that the adjoint action is the natural action of $G$ on its Lie algebra, is the fact that it corresponds to the natural action of $G$ on $Lie(G)$ by right translation:
\begin{itemize}
    \item Given $g\in G$ and $X\in Lie(G)$, $(R_g)_*X$ (the pushforward of $X$ by the diffeo $R_g:G\to G$) is a left-invariant vector field. 
    
    This is a consequence of the fact that right translations commute with left translations. Indeed for $a,b,c\in G$ we have $L_aR_bc=acb=R_bL_ac$ so that $L_aR_b=R_bL_a$.

    And so for $h\in G$ we find $(L_h)_*(R_g)_*X=(R_g)_*(L_h)_*X=(R_g)_*X$.
    \item Remarkable is the fact that $(R_g)_*X=Ad(g^{-1})X$.

    Note: $g^{-1}$ appears instead of $g$ in $Ad$ because both the LHS and RHS must describe right actions of $G$ on $Lie(G)$.

    Indeed $((R_g)_*X)_e=d(R_g)X_{g^{-1}}=d(R_g)d(L_{g^{-1}})X_e=d(C_{g^{-1}})X_e=(Ad(g^{-1})X)_e$.
\end{itemize}

Following our heuristics of thinking of a Lie group $G$ as acting transitively on itself on the right, and of right $G$-actions as equivariant maps, we find that when $G$ acts on the right on a mfd $M$, $G$ acts on the rights on the fundamental vector fields on $M$ and that the infinitesimal generator interlaces this right action with the adjoint action of $G$ on $Lie(G)$.

Note: This is useful when manipulating spaces with $G$-actions, in particular principal $G$-bundles.

\begin{itemize}
    \item\label{AdjointRepresentationFundamentalVectorField} Let $G$ be a Lie group acting on the right on a smooth manifold $M$. Then for $X\in Lie(G)$, $(R_g)_*\hat{X}=\widehat{Ad(g^{-1})X}$.

    \begin{itemize}
        \item This boils down to realizing that for $p\in M$, $\theta$ interlaces the right action of $G$ on $G$ and $M$ ie $\theta^{(p)}R_g=R_g\theta^{(p)}$.
        \item Consequently for any $p\in M$ and $X\in Lie(G)$, we find that $(R_g)_*\hat{X}$ is $\theta^{(p)}$-related to $(R_g)_*X=Ad(g^{-1})X$ which allows us to conclude.
    \end{itemize}
\end{itemize}

\subsection{Ideals and Normal Subgroups}

The main observation here is the fact that connected normal Lie subgroups correspond exactly to ideals of the Lie algebra.

Note: normal Lie subgroups and ideals both correspond to kernels of the appropriate morphisms.

\begin{itemize}
    \item Let $H$ be a connected Lie subgroup of a Lie connected Lie group $G$. Then $H$ is normal iff $\exp(X)\exp(Y)\exp(-X)$ belongs to $H$ for all $X\in Lie(G)$, $Y\in Lie(H)$.

    This follows from the fact that the connectedness of $H$ means that one of its elements, say $h$ can be written as a product $h=\exp(Y_1)...\exp(Y_n)$ for $Y_1,...,Y_n\in Lie(H)$ and similarly for $G$ (for a sequence of elements in $Lie(G)$).
    \item In fact for $H$, $G$ connected Lie groups with $H$ a Lie subgroup of $G$, $H$ is normal in $G$ iff $Lie(H)$ is an ideal of $Lie(G)$.
\end{itemize}

\section{Quotient Manifolds}

Openess of the quotient map:
\begin{itemize}
    \item Let a topological group $G$ act continuously on a topological space $X$. Then the quotient map $\pi:M\to M/G$ is open.

    Note: we always equip $M/G$ with the quotient topology.
    \begin{itemize}
        \item This is proven as follows: for $U$ open in $M$, $\pi^{-1}(\pi(U))=\bigcup_{g\in G}gU$ is open since as a union of opens, indeed: for $g\in G$, $gU$ is open as the image of an open by a homeomorphism.
    \end{itemize}
\end{itemize}

To guarantee that the quotient of a smooth mfd by the action of a Lie group is well-behaved wrt smoothness, we require properness of the action (tied to the notion of a proper map).

\begin{itemize}
    \item A continuous left action of a topological group $G$ on a topological space $M$ is proper if the map $G\times M\to M\times M, (g,p)\to (gp,p)$ is proper.

    Note: This is weaker than requiring that $G\times M\to M,(g,p)\to gp$ be proper.
    \item If a Lie group acts continuously and properly on a mfd, then the orbit space is Hausdorff.

    Note: we need the properties of a topological mfd (hence why we ask for a Lie group and a smooth mfd).
    \item Characterizations of proper actions. Let $M$ be a mfd, and let $G$ be a Lie group acting continuously on $M$. TFAE
    \begin{itemize}
        \item The action is proper.
        \item If $(p_i)$ is a sequence in $M$ and $(g_i)$ a sequence in $G$ such that both $(p_i)$ and $(g_ip_i)$ converge, then $(g_i)$ converges as well.
        \item For every compact subset $K$ of $M$, the set $G_K:=\{g\in G|(gK)\cap K\neq\emptyset\}$ is compact.
    \end{itemize}
    \item Corollary: any continuous action by a compact Lie group on a smooth manifold is proper.
    \item (Orbits of Proper Actions) Suppose $\theta$ is a proper smooth action of a Lie group $G$ on a smooth mfd $M$. Then for any point $p\in M$, $\theta^{(p)}:G\to M$ is proper. In particular:
    \begin{itemize}
        \item The stabilizer/isotropy group $G_p=\theta^{(p)}^{-1}(p)$ is compact.
        \item The orbit $Gp=\theta^{(p)}(G)$ is closed in $M$ (since a smooth proper map is closed).
        \item If $G_p=\{e\}$ then $\theta^{(p)}$ is a smooth embedding and $Gp$ is a properly embedded submanifold.

        Note: The steps: $\theta^{(p)}$ is an injective immersion as an injection of constant rank (bc of equivariance) and it is proper.
    \end{itemize}
    \item Example: the two necessary conditions on proper actions - compact isotropy groups and closed orbits - can help us determine when an action is non proper. For example we can see that the left action of $\mathbb{R}^+$ on $\mathbb{R}^n$ given by $(t,x)\to tx$ is non proper in 2 ways:
    \begin{enumerate}
        \item Because the isotropy group of $0$ is $\mathbb{R}^+$ which is non compact.
        \item Because the orbits of non zero elements are open rays which are not closed.
    \end{enumerate}

    We see that all the non properness of the action seems to be tied to the origin of $\mathbb{R}^n$, and indeed the same action but defined on $\mathbb{R}^n-\{0\}$ IS proper (and free) and the quotient mfd, $S^{n-1}$ is a smooth mfd.
\end{itemize}

\subsection{Quotient Mfd Theorem}

The big one (with a long proof).
\begin{itemize}
    \item Let $G$ be a Lie group acting smoothly, properly and freely on a smooth mfd $M$, then the orbit space $M/G$ is a topological mfd with dimension $\dim M-\dim G$ and admits a unique smooth structure for which $M\to M/G$ is a smooth submersion.
\end{itemize}

\subsection{Homogenous spaces}

First: when is the coset space of a Lie subgroup a well-behaved smooth mfd? Answer when the Lie subgroup is closed.
\begin{itemize}
    \item Let $H$ be a closed Lie subgroup of $G$. Then the left coset space $G/H$ is a topological mfd of dimension $\dim G-\dim H$ and admits a unique smooth structure for which $G\to G/H$ is a smooth submersion. For this smooth structure, the usual left action of $G$ on $G/H$ makes $G/H$ into a smooth homogenous $G$-space.

    \begin{itemize}
        \item Boils down to showing that the action of $H$ on $G$ is proper. Then use submersion shenanigans to show that the action of $G$ on $G/H$ is smooth.
        \item Actually, $G$ is a principal $H$ bundle over $G$.
        \item $H$ being closed is necessary: if $G\to G/H$ is smooth, then $H$ must be closed as the preimage of a (closed) singleton by a continuous map.
    \end{itemize}
\end{itemize}

All homogenous spaces are of this form.
\begin{itemize}
    \item Let $M$ be a homogenous $G$-space and $p$ be any point of $M$. Then the isotropy subgroup $G_p$ at $p$ is a closed subgroup of $G$ and the map $F:G/G_p\to M, gG_p\to gp$ is an equivariant diffeomorphism. 
\end{itemize}

Sets with transitive group actions of Lie groups (and with closed isotropy subgroups) have a canonical smooth structure.
\begin{itemize}
    \item Let $X$ be a set, and assume a Lie group $G$ acts transitively on $X$ and that for some point $p\in X$ the isotropy subgroup $G_p$ is closed. Then $X$ admits a unique smooth structure for which the action of $G$ is smooth. For this structure $X$ is canonically diffeomorphic to $G/G_p$ and so has dimension $\dim G-\dim G_p$.
\end{itemize}

\subsection{Isomorphism theorem for Lie groups}

Quotient theorem for Lie groups (closed normal Lie subgroups are kernels of Lie morphisms).
\begin{itemize}
    \item Let $K$ be a closed normal Lie subgroup of $G$. Then $G/K$ is a Lie group and the quotient map $G\to G/K$ is a subjective Lie group morphism.
\end{itemize}

Isomorphism theorem for Lie groups.
\begin{itemize}
    \item Let $F:G\to H$ be a Lie group morphism. Then $\ker F$ is a normal closed subgroup of $G$, $Im F$ is a Lie subgroup of $H$ and $F$ descends to a Lie group isomorphism $\tilde{F}:G/\ker F\to Im F$.

    Steps:
    \begin{itemize}
        \item Straightforward to show that $\ker F$ is a closed normal subgroup.
        \item Then $G/\ker F$ is a Lie group and since $G\to G/\ker F$ is a submersion, $F$ descends to a Lie group morphism $\tilde{F}:G/\ker F\to H$.
        \item Since $\tilde{F}$ is an injective Lie group morphism, it is a smooth embedding and $Im F$ is a Lie subgroup w/ $\tilde{F}:G/\ker F\to Im F$ a Lie group iso.

        Note: since Lie group are weakly embedded their smooth structure is always unique.
    \end{itemize}
\end{itemize}

\section{From KN}

\subsection{Left invariant forms}

Definition and Mauer-Cartan equation.

\begin{itemize}
    \item A differential form $\alpha$ on $G$ is left invariant if $(L_a)^*\alpha=\alpha$ for all $a\in G$.
    \item The space of left invariant 1-forms $\mathfrak{g}^*$ is the dual of the Lie algebra of $G$, $\mathfrak{g}$. This is because for $\omega\in\mathfrak{g}^*$ and $X\in\mathfrak{g}$ $\omega(X)$ will be constant on $G$.

    More generally a left invariant $r$-form on $G$ is an element of $\Lambda^r\mathfrak{g}$ (and the space of left invariant r forms is finite-dimensional).
    \item (Equation of Mauer-Cartan) For $\omega$ left invariant 1-form, $d\omega$ is a left invariant form (by naturality of the exterior derivative) and for $A,B\in\mathfrak{g}$ we find $d\omega(A,B)=-\omega([A,B])$.

    This follows immediately from the formula for the exterior derivative in terms of Lie brackets and the fact that a left invariant form evaluated on left invariant vector fields gives a constant function.

    Notes
    \begin{itemize}
        \item Extending this reasoning a bit, we obtain a formula relating the exterior derivative of a left invariant forms to the Lie bracket for any order.
        \item In KN a $1/2$ appears in the formula above, because they use a different convention for the wedge product.
        \begin{itemize}
            \item In KN the wedge is defined by $\omega\wedge\eta:=Alt(\omega\otimes\eta)$, and interior multiplication is defined by $\iota_X\omega=r\omega(X,-,...,-)$ for an $r$-form $\omega$.
            \item Our conventions are those used in Lee's book, so $\omega\wedge\eta=\frac{(k+l)!}{k!l!}Alt(\omega\otimes\eta)$ (which gives $\alpha\wedge\beta(X,Y)=\alpha(X)\beta(Y)-\alpha(Y)\beta(X)$ for two 1-forms $\alpha$, $\beta$), and $\iota_X\omega=\omega(X,-,...,-)$.

            Note: this must change the convention on the exterior derivative, but where? And why does Stokes' theorem still hold then?
        \end{itemize}
        \item We already saw that the spaces of left invariant forms could be seen as tensor spaces constructed on $\mathfrak{g}$, and now we see that the exterior derivative can be expressed via the Lie bracket of $\mathfrak{g}$, so the de Rham complex of left-invariant forms can be completely constructed using $\mathfrak{g}$ and its Lie bracket.

        This is the motivation for the definition of the cohomology of a Lie algebra.
    \end{itemize}
\end{itemize}

The canonical left invariant $\mathfrak{g}$-valued 1-form, and coordinate expression of Mauer-Cartan.
\begin{itemize}
    \item The canonical left invariant $\mathfrak{g}$-valued 1-form $\theta$ is defined by $\theta(A)=A$ for any $A\in\mathfrak{g}$.
    \item We can choose a basis $E_i$ of $\mathfrak{g}$.
    \begin{itemize}
        \item We can encode the Lie bracket in the structure constants $c^i_{jk}$ defined by $[E_j,E_k]=c^i_{jk}E_i$.
        \item We have a unique expression $\theta=\theta^iE_i$ and $(\theta^i)$ is the basis of $\mathfrak{g}^*$ dual to $(E_i)$.
        \item Mauer-Cartan is then equivalent to $d\theta^i=-c^i_{jk}\theta^j\wedge\theta^k$

        Note: We see, these times through the structure constants that the information contained in the Lie bracket of left invariant vector fields is exactly the information encoded in the exterior derivative of left invariant forms.
    \end{itemize}
\end{itemize}

\part{Theory of Connections [KN]}

\chapter{Fiber Bundles}

Definition.
\begin{itemize}
    \item Definition as $G$-space: Given a Lie group $G$ and a smooth mfd $M$, a principal fiber bundle over $M$ with structure group $G$ is a smooth manifold $P$ with a right $G$-action satisfying the following conditions:
    \begin{itemize}
        \item $G$ acts freely on $P$.
        \item $M$ is the quotient space $P/G$ and the quotient map $\pi:P\to M$ is differentiable.
        \item Local triviality.
    \end{itemize}

    We denote such a principal fiber bundle by $P(M,G,\pi)$ and $P(M,G)$.
    \item Definition by local trivialization (cocycle).
    \item Note that a trivialization of $P$ over an open $U$ of $M$ is equivalent to a local section $\sigma:U\to P$ (the section is to be thought of as a choice of identities).
\end{itemize}

Fundamental vector field.
\begin{itemize}
    \item For $A\in\mathfrak{g}$, we use $A^*$ to denote the fundamental vector field induced on $P$ by $A$.
    \item As we saw in (\ref{AdjointRepresentationFundamentalVectorField}) for $a\in G$, $A\in\mathfrak{g}$, $(R_a)_*A^*$ is the fundamental vector field corresponding to $Ad(a^{-1})A$.
\end{itemize}

Morphism of principal fiber bundles.
\begin{itemize}
    \item A morphism of $P'(M',G')$ into $P(M,G)$ is comprised of a smooth map $f:P'\to P$ and a Lie group morphism $f:G'\to G$ st $f(ua)=f(u)f(a)$ for $u\in P'$ and $a\in G'$.

    $f:P'\to P$ preserves fibers and induces a smooth map $f:M'\to M$
\end{itemize}

A whole section on the reduction of the structure group of a fiber bundle. Two equivalent definitions of reducibility.
\begin{itemize}
    \item Given a Lie subgroup $H$ of $G$, and a principal $G$-bundle $P(M,G)$ over $M$, a reduction of the structure group of $P$ to $H$ is a subbundle $P'(M,H)\to P(M,G)$ (over $M$ - ie the identity of $M$). The subbundle $P'(M,H)$ is called the reduced bundle.

    Note: According to the heuristics of thinking of the fibers of $P$ as "copies of $G$ without identity", a reduction of $G$ to $H$ should be thought of as a "choice of $H$" in each fiber - which would generalize a trivialization as a "choice of identity in each fiber", and indeed, a trivialization is a special case of reduction, where $H=\{e\}$.
    \item A reduction of the structure group of $P(M,G)$ to $H$ can equivalently be defined as a choice of trivializing atlas whose transition maps factor through $H$.
\end{itemize}

Associated fiber bundle.
\begin{itemize}
    \item Given a smooth left action of $G$ on a mfd $F$ and a principal $G$-bundle $P(M,G)$, we can construct a fiber bundle $E(M,F,G,P)$ with standard fiber $F$, which we call the fiber bundle associated to $P$.
    \begin{itemize}
        \item As a set $E:=P\times_GF$ where $P\times_GF$ is the orbit space of $P\times F$ for the right $G$-action $(u,x)a:=(ua,a^{-1}x)$ for $(u,x)\in P\times F$, $a\in G$.
        \item The projection $P\times F\to M,(u,x)\to\pi(u)$ descends to a projection $\pi_E:E\to M$ (since $G$ preserves the fibers of $P$), which is the one we assign to $E$.
        \item Once we choose a base point $u\in P$ above $p\in M$, we obtain the following bijection btw the fiber of $E$ above $p$ and $F$: $F\to E_p,x\to (u,x)$.

        So we see that a trivialization of $P$ above $U$, so a section $U\to P$ will induce a bijection $\pi_E^{-1}(U)\simeq U\times F$. We endow $E$ with a smooth structure by requiring that this bijection be a diffeomorphism.

        The transition maps between the trivialization are given by the smooth action of $G$ on $F$ (and the transition maps of $P$), and so are diffeos.
    \end{itemize}

    Note: Elements of the principal fiber bundle should be thought of as "generalized bases" of the associated fiber bundle.
    \item (Associated fiber bundle for closed subgroup) Given a closed subgroup $H$ of $G$, we have the natural left action of $G$ on $G/H$ which allows to construct the associated bundle $P\times_G G/H$ from a principal bundle $P(M,G)$. This associated bundle is canonically isomorphic to $P/H$, the quotient of $P$ by the right action of $H$.

    To see this: clear that once we choose a base point $u\in P_p$ above $p\in M$, the fiber of any of those two bundles above $p$ is naturally iso. to $G/H$. More explicitly, after choosing $u\in P$ in the fiber of $P$ above $p$ we have iso $E_p\to (P/H)_p,(u,gH)\to ugH$.
\end{itemize}

For a closed subgroup $H$ of $G$, reductions to $H$ correspond to cross sections of the associated bundle $E(M,G/H,G,P)$. 
\begin{itemize}
    \item A reduction of $P(M,G)$ to $H$ corresponds to a global section $M\to E(M,G/H,G,P)$.

    Note: The intuitive explanation is that a global section of $E=P/G$ corresponds exactly to a "smooth choice" of $H$ in each fiber of $G$, so a reduction of $P$ to $H$

    Note: This will become very useful when $G/H$ is iso to a Euclidean space, since the existence of a global section will then be guaranteed.

    Note: We can check coherence in extreme cases, so when $H=G$ the associated bundle is $P/G=M$, which trivially admits a unique global section, corresponding to the unique reduction of $P$ to $G$. When $G=\{e\}$, the associated bundle is $P/e=P$ and so a reduction of $P$ to $\{e\}$, that is, a global trivialization, is indeed equivalent to a global section of $P$.
\end{itemize}

Existence of sections under certain circumstances when the fibers are diffeo to a Euclidean space => Combined with reduction/sections of associated fiber bundles, it can guarantee that structure group be reducible under certain circumstances (see some examples)
\begin{itemize}
    \item Let $E$ be a fiber bundle (in KN this means a bundle associated to some fiber bundle - basically this means that we ask the transition maps to factor through some finite dimensional Lie group) st $M$ is paracompact and the standard fiber $F$ is diffeo to a euclidean space $\mathbb{R}^m$.

    Then for a closed subset $A$ (possibly empty) of $M$, any section $\sigma:A\to E$ (check def of smoothness on a nonopen subset) on $A$ can be extended to a global section defined on $M$.

    In particular, choosing $A=\emptyset$, we see that any fiber bundle $E$ with standard fiber diffeo to a Euclidean space over a paracompact mfd admits a global section.

    Note: Some partition of unity stuff.
\end{itemize}

Some examples.
\begin{itemize}
    \item For $H$ a closed Lie subgroup of $G$, the quotient map $G\to G/H$ together with the natural right $H$-action on $G$ makes $G$ into a principal $H$-bundle over $G/H$, $G(G/H,H)$.
    \item (Bundle of Linear Frames) An element of the fiber of $L(M)$ above $p\in M$ is a linear iso. $u:\mathbb{R}^n\to T_pM$ (so a choice of basis on $T_pM$). $Gl(n,\mathbb{R})$ acts on $L(M)$ on the right by precomposition.

    A chart on $M$ provides a trivialization of $TM$ and of $L(M)$.
    \item Tangent bundle $TM$ is associated to $L(M)$ through the standard action of $Gl(n,\mathbb{R})$ on $\mathbb{R}^n$.
    \item More generally any tensor bundle $T^r_s(M)$ is associated to $L(M)$ through the standard action of $Gl(n,\mathbb{R})$ on $T^r_s(\mathbb{R})$.
\end{itemize}

Examples of reductions.
\begin{itemize}
    \item (Riemannian structures) There is a 1:1 correspondence btw reductions of $L(M)$ to $O(n)$ and riemannian structures on $M$.

    Since $O(n)$ is a closed subgroup of $Gl(n)$ with $Gl(n)/O(n)=Sym(n)$ ($Sym(n)$ the space of symmetric matrices), we know that such a reduction exists <=> any paracompact mfd admits a riemannian structure.
    \item Iwasawa: any Lie group is diffeomorphic to a product of a maximal compact subgroup and a Euclidean space. This means that over a paracompact mfd, we can always reduce the structure group to a compact one.
\end{itemize}

[Examples given by normal coverings]

Pullback of a principal fiber bundle.
\begin{itemize}
    \item Given a principal bundle $P(M,G)$ and a mapping $f:N\to M$, there exists a unique (up to iso) principal bundle $Q(N,G)$ with a morphism $f:Q\to P$ that descends to $f:N\to M$ and corresponds to the identity morphism of $G$.
\end{itemize}

\chapter{Principal Connections}

\section{Starting definitions}

Equivalent ways of defining a principal connection on a principal bundle $P(M,G)$.
\begin{itemize}
    \item First definition as a $G$-invariant horizontal distribution on $P$.
    \begin{itemize}
        \item For $u\in P$ we define the vertical subspace at $u$ as the subspace of $T_uP$ comprised of the vectors tangent to the fiber through $u$.

        \item Together the vertical subspaces form the vertical bundle $VP$ (a subbundle of $TP$).

        On the structure of the vertical bundle:
        \begin{itemize}
            \item $P\times\mathfrak{g}\to VP,(u,A)\to A^*_u$ is an isomorphism of vector bundles (so $VP$ is trivial).
            \item Above the right action of $G$ on $P$ we have a right $G$-action on $VP$ and a right $G$-action on $P\times\mathfrak{g}$: for $a\in G$ we have $Xa:=d(R_a)X$ for $X\in VP$ and $(u,A)a:=(ua,ad(a^{-1})A)$ for $(u,a)\in P\times\mathfrak{g}$.
            \times Since $(R_a)_*A^*$ corresponds to $ad(a^{-1})A$, we see that the isomorphism between $VP$ and $P\times\mathfrak{g}$ is $G$-equivariant.
        \end{itemize}
        \item A principal connection on $P$ is then a distribution $Q$ on $P$, called the horizontal distribution, such that
        \begin{itemize}
            \item For each $u\in P$, $T_uP=G_u\oplus Q_u$.
            \item $Q$ is $G$-invariant, ie for $u\in P$, $a\in G$, $d(R_a)Q_u=Q_{ua}$.
        \end{itemize}
    \end{itemize}
    \item As an equivariant vertical projection. Specifying such a horizontal distribution is equivalent to providing a $G$-equivariant projection $v:TP\to VP$, that is a morphism of vector bundles $v:TP\to VP$ such that:
    \begin{itemize}
        \item $v^2=v$ (projection).
        \item $(R_a)_*v=v(R_a)_*$ for $a\in G$.
    \end{itemize}

    The link to the previous definition is the following: $Q=\ker v$.

    \item Using the equivariant iso btw $VP$ and $P\times\mathfrak{g}$, an equivariant projection $TP\to VP$ corresponds to a connection 1-form, that is a $\mathfrak{g}$-valued 1-form $\omega$ on $P$ st
    \begin{itemize}
        \item $\omega(A^*)=A$ for any $A\in \mathfrak{g}$ (this translates the "projection" aspect).
        \item $(R_a)^*\omega=ad(a^{-1})\omega$ for all $a\in G$ (this in the "$G$-invariant" part).
    \end{itemize}
\end{itemize}

\subsection{Horizontal Lift}

Horizontal lift of a vector field: vector fields on $M$ correspond to $G$-invariant horizontal vector fields on $P$.
\begin{itemize}
    \item A horizontal vector field on $P$ is a section of the horizontal distribution $Q$.
    \item Since $VP=\ker \pi_*$ with $\pi:P\to M$ the projection, and since $TP=Q\bigoplus VP$, we see that $\pi_*$ induces an iso. btw $Q$ and $\pi^*TM$.
    \item The horizontal lift of a vector field $X$ on $M$ is the unique horizontal vector field $X^*$ on $P$ that projects onto $X$. $X^*$ is $G$-invariant and conversely, any $G$-invariant horizontal field on $P$ is the lift of a unique field on $M$.
\end{itemize}

Basic properties of horizontal lift of vector fields:
\begin{itemize}
    \item Assume $X^*,Y^*$ are horizontal lifts of $X,Y$ then
    \begin{itemize}
        \item $X^*+Y^*$ is the horizontal lift of $X+Y$.
        \item $f^*X^*$ will be the horizontal lift of $fX$ for any $f\in C^{\infty}(M)$, with $f^*:=f\circ\pi$.
        \item By naturality of the Lie bracket, $[X^*,Y^*]$ projects onto $[X,Y]$, but is not necessarily horizontal (involutivity of the horizontal distribution). 
        
        Note: We see that the failure of $Q$ to be involutive is "vertical", ie $[X^*,Y^*]-[X,Y]^*$ is vertical. This is our first indication that the curvature (measuring the failure of $Q$ to be involutive) can be defined as a $\mathfrak{g}$-valued 2-form (see Vakar).
    \end{itemize}
\end{itemize}

[Existence and Extension of Principal connections - once again, any principal bundle over a paracompact mfd admits a principal connection]

\subsection{Local form of the connection form}

Representation of the connection form in a local trivialization and transformation of the representation under change of trivialization.
\begin{itemize}
    \item We can choose a trivializing atlas of $P(M,G)$, that is an open cover $(U_\alpha)$ of $M$ with a section $\sigma_\alpha:U_\alpha\to P$.

    The transition maps $\psi_{\alpha\beta}:U_\alpha\cap U_\beta\to G$ are defined by $\sigma_\beta=\sigma_\alpha\psi_{\alpha\beta}$.
    \item If we have a connection 1-form $\omega$ on $P$ we can represent $\omega$ above $U_\alpha$ by the pullback $\omega_\alpha:=\sigma^*_\alpha\omega$, a $\mathfrak{g}$-valued 1-form on $U_\alpha$.
    \item We use $\sigma_\alpha,\sigma_\beta,\psi_{\alpha\beta}$ to denote the differentials of the respective maps. For a tangent vector $X\in T_x(U_\alpha\cap U_\beta)$, by applying the formula for the differential for a map defined on a product to the action map $P\times G\to P$ we get $\sigma_\beta(X)=\sigma(X)\psi_{\alpha\beta}(x)+\sigma(x)\psi_{\alpha\beta}(X)$.

    Since $\omega_\alpha(X)=\omega\sigma_\alpha(X)$ and $\omega_\beta(X)=\omega\sigma_\beta(X)$ we get $\omega_\beta(X)=ad(\psi_{\alpha\beta}(x))\omega_\alpha(X)+\psi_{\alpha\beta}(X)$ (the 2nd term appears because $\sigma_\alpha(x)\psi_{\alpha\beta}(X)=\psi_{\alpha\beta}(X)^*_{\sigma_\alpha(x)}$).

    From this we conclude $\omega_\beta=ad(\psi_{\alpha\beta}^{-1})\omega_\alpha+\theta_{\alpha\beta}$. With $\theta_{\alpha\beta}=\psi_{\alpha\beta}^*\theta$.

    Note: The appearance of the $\theta_{\alpha\beta}$ term is related to the fact that $\omega$ is not horizontal, and so not equivalent to a $\mathfrak{g}$-valued on $M$ - pseudotensorial, not tensorial.
\end{itemize}

\section{Parallelism - Parallel Transport}

Definition of horizontal lift of a path, and existence+uniqueness.

\begin{itemize}
    \item Given a path $\tau=x_t$, $a\leq t\leq b$, a horizontal lift of $\tau$ is a horizontal curve $\tau^*=u_t$, $a\leq t\leq b$ in $P$ such that $\pi(u_t)=x_t$.

    Note: A horizontal curve is a curve in $P$ whose tangent vectors are all horizontal.

    Note: A possible interpretation of the connection form: given a path $\tau'$ in $P$, $\tau'$ is horizontal iff $\omega(\dot{\tau'})=0$, otherwise $\omega(\dot{\tau'})$ evaluated at a point $u$ in $\tau'$ evaluates the instantenous rate of divergence between $\tau'$ and a horizontal path starting at $u$ which projects onto the same path onto $M$ as $\tau'$. 
    \item Given a path $\tau=x_t$, $0\leq t\leq 1$ in $M$. For an arbitrary point $u_0$ in $P$ with $\pi(u_0)=x_0$ there exists a unique horizontal lift $\tau^*=u_t$ of $\tau$ starting at $u_0$.

    Elements of the proof [Very geometric in flavour]
    \begin{itemize}
        \item Using the local triviality of $P$, we know that there exists a lift $v_t$ (not necessarily horizontal) of $\tau$ starting at $u_0$. A horizontal lift $u_t$ of $\tau$ would then have to take the form $u_t=v_ta_t$ for some curve $a_t$ in $G$ starting at the identity.
        \item Now we derive the condition that $a_t$ must satisfy in order for $u_t$ to be horizontal.
        \begin{itemize}
            \item Using the formula for the differential of a smooth map defined on a product, we get $\dot{u_t}=\dot{v_t}a_t+v_t\dot{a_t}$. So we have $\omega(\dot{u_t})=ad(a^{-1}_t)\dot{v_t}+a^{-1}_t\dot{a_t}$ (here $a^{-1}_t\dot{a_t}$ is the left translation of $\dot{a_t}$ which is tangent to $a_t\in G$, and brings it back to the identity).

            Note: why do we have $\omega(v_t\dot{a_t})=a^{-1}_t\dot{a_t}$? Because $a^{-1}_t\dot{a_t}$ corresponds to the left-invariant vector field taking value $\dot{a_t}$ at $a_t$ so that $v_t\dot{a_t}=(a^{-1}_t\dot{a_t})^*_{v_t}$.
            \item So $u_t$ is horizontal if and only if $-\omega(\dot{v_t})=ad(a_t)a^{-1}_t\dot{a_t}=\dot{a_t}a^{-1}_t$.
            \item Then the we only need the following lemma: Given a curve $Y_t$, $0\leq t\leq 1$ in $T_eG$ (identified with $\mathfrak{g}$), there exists a unique curve $a_t$ $0\leq t\leq 1$ such that $a_0=e$ and $Y_t=\dot{a_t}a^{-1}_t$.
        \end{itemize}
    \end{itemize}
\end{itemize}

Once we have uniqueness and existence, we can define parallel displacement.
\begin{itemize}
    \item Given a path $\tau=x_t$, $0\leq t\leq 1$ in $M$ going from $x_0$ to $x_1$, we obtain a map (also called $\tau$) from the fiber $\pi^{-1}(x_0)$ onto the fiber $\pi^{-1}(x_1)$ in the following way:
    \begin{itemize}
        \item $u_0$ in $\pi^{-1}(x_0)$ is sent to the endpoint of the horizontal lift $\tau^*$ of $\tau$ starting at $u_0$.
    \end{itemize}
    \item Because the action of $G$ sends a horizontal lift of $\tau$ to another horizontal lift of $\tau$, parallel displacement along $\tau$ is a $G$-equivariant map, ie $R_a\tau=\tau R_a$ for any $a\in G$.

    From this we deduce that $\tau:\pi^{-1}(x_0)\to\pi^{-1}(x_1)$ is smooth (in fact a diffeo).
    \item Since the concatenation of horizontal lifts is a horizontal lift of the concatenation, we can show that if $\tau$ and $\mu$ are concatenable paths in $M$, then the parallel displacement along $\mu.\tau$ is the composition of the parallel displacements along $\mu$ and $\tau$.

    Combining this with the fact that the inversion of a horizontal lift will be a horizontal lift of the inversion, we can also show that the inverse of $\tau$ is given by $\bar{\tau}$, parallel displacement along the path inverse to $\tau$. (Another way of showing that $\tau$ is a diffeo).
    \item Note also that parallel displacement along a path $\tau$ in $M$ doesn't vary if we reparametrize $\tau$. Indeed, if $\tau^*$ is a horizontal lift of $\tau$ and if our parametrization takes the form $\tau\circ\gamma$, then $\tau^*\circ\gamma$ will be a horizontal lift of $\tau\circ\gamma$.
\end{itemize}

Note: In KN, some care is afforded to the relaxation of regularity requirements on the paths involved. We just assume everything is smooth - by parts maybe - (we don't require analycity though apparently we could).

\section{Holonomy Group}

Different definitions of the holonomy group.
\begin{itemize}
    \item Given a point $x\in M$ we use $C(x)$ to denote the set of loops in $M$ based at $x$. We use $C^0(x)$ to denote the set of loops based at $x$ which are homotopic to zero.

    Given a principal bundle $P(M,G)$ equipped with a principal connection $\Gamma$, the holonomy group $\Phi(x)$ of $\Gamma$ with reference point $x$ is the group of $G$-equivariant automorphisms of $\pi^{-1}(x)$ that correspond to parallel displacements along elements of $C(x)$.

    The restricted holonomy group $\Phi^0(x)$ of $\Gamma$ with reference point $x$ has an analogous definition, except we only allow elements of $C^0(x)$ (instead of $C(x)$).
    \item It is often convenient to see the holonomy group as a subgroup of $G$, which can be done once we choose a base point $u$ in $P$.

    So let us choose a point $u\in P$ above $x\in M$. Given $\tau\in C(x)$, since $\tau:\pi^{-1}(x)\to\pi^{-1}(x)$ is $G$-equivariant, the map $\tau$ is completely determined by $\tau u$. And since $\pi^{-1}(x)$ is equipped with a free transitive $G$-action, there exists a unique $a_\tau\in G$ such that $\tau u=u a_{\tau}$.

    We define $\Phi(u)$, the holonomy group of $\Gamma$ with reference point $u$ as the subset $\{a_\tau|\tau\in C(x)\}$ of elements of $G$.

    We define $\Phi^0(u)$ similarly, with $C^0(x)$ playing the role of $C(x)$.

    The behaviour of the horizontal lift wrt to concatenation and inversion, and the existence of the zero loop allows us to show that $\Phi(x)$, $\Phi^0(x)$ are subgroups.
    \item We can also construct $\Phi(u)$ via an equivalence relation, which in any case is useful otherwise.

    We define the following relation on $P$: $u\sim v$ if $u$ and $v$ can be connected by a horizontal path (note: given $\tau$ in $M$ connecting $x_0$ and $x_1$, for $u_0\in\pi^{-1}(x_0)$, $u_1:=\tau u_0$ is the unique element of $\pi^{-1}(x_1)$ such that $u_0\sim u_1$).

    We can show that $\sim$ is an equivalence relation:
    \begin{itemize}
        \item Identity follows from the fact that the zero path is horizontal.
        \item Symmetry follows from the fact that the inversion of a horizontal path is horizontal.
        \item Transitivity follows from the fact that the concatenation of horizontal paths is horizontal.
    \end{itemize}

    Given the fact that the action of $G$ sends horizontal paths to horizontal paths, we can also show that $\sim$ is compatible with the $G$-action, ie for $u,v\in P$, $u\sim v$ implies $ua\sim va$.

    We define $\Phi(u)$ as the subset of elements $a$ of $G$ such that $u\sim ua$. $\Phi^0(u)$ can be defined as the subset of elements of $G$ such that there exists a horizontal path in $P$ connecting $u$ and $ua$ whose projection onto $M$ can be contracted to zero.

    Once again we can show that $\Phi(u)$ and $\Phi^0(u)$ are subgroups.
    \begin{itemize}
        \item They contain the identity bc of the identity of $\sim$ (so the zero path).
        \item Then we use the fact that $u\sim v\implies ua\sim va$, so if $ua\sim u$ then $u=uaa^{-1}\sim ua^{-1}$ so that $\Phi(u)$ is stable by inversion, and if $u\sim ua$ and $u\sim ub$ we find $u\sim ub\sim uab$ so that $\Phi(u)$ is stable by multiplication.
    \end{itemize}
\end{itemize}

Some basic properties of the holonomy groups.
\begin{itemize}
    \item For $u\in P$ and $a\in G$ we have $\Phi(ua)=ad(a^{-1})\Phi(u)$, and similarly $\Phi^{0}(ua)=ad(a^{-1})\Phi^0(u)$.
    
    To see this: if $u\sim ub$, then $ua\sim uba=(ua)a^{-1}ba=(ua)ad(a^{-1})b$ so $b\in\Phi(u)$ implies $ad(a^{-1})b\in\Phi(ua)$ and we get the converse from $u=(ua)a^{-1}$. 
    \item If $u\sim v$ then $\Phi(u)=\Phi(v)$ and $\Phi^0(u)=\Phi^0(v)$.

    To see this: if $v\sim va$ then since we know $u\sim v$ which implies $ua\sim va$ we find $ua\sim va\sim v\sim u$. So we see that $a\in\Phi(v)$ implies $a\in\Phi(u)$ and the converse follows immediately from the symmetry of $\sim$.
\end{itemize}

The cool thing is that the holonomy group is a Lie subgroup (above paracompact mfd).
\begin{itemize}
    \item Let $P(M,G)$ be a principal fiber bundle over a paracompact manifold $M$ let $\Phi(u)$ and $\Phi^0(u)$ be the holonomy and restricted holonomy groups of a principal connection $\Gamma$ on $P$ with reference point $u$. Then $\Phi_0(u)$ is a connected Lie subgroup of $G$ and $\Phi(u)/\Phi^0(u)$ is countable.

    Note: indicated the fact that $\Phi(u)$ is a Lie subgroup whose identity component is $\Phi^0(u)$.
\end{itemize}

\section{Curvature form and Structure Equation}

\subsection{(Pseudo)tensorial forms}

Definition and characterization of tensorial forms.
\begin{itemize}
    \item Given a linear representation of $G$ on a finite dimensional vector space $V$, a pseudotensorial form of degree $r$ of type $(\rho,V)$ on $P$ is a $V$-valued r-form $\phi$ on $P$ such that $(R_a)^*\phi=\rho(a^{-1})\phi$ for any $a\in G$.

    Such a form $\phi$ is tensorial if it is horizontal in the sense that $\phi(X_1,...,X_r)$ if at least one of the tangent vectors $X_1,...,X_r$ is vertical (ie tangent to the fiber).
    \item Using $\rho$ we can construct the associated vector bundle $E$ with standard fibre $V$.

    Choosing $u\in P$ above $x\in M$ the map $V\to\pi_E^{-1}(x),v\to(u,v)=:uv$ is a linear iso which we also denote by $u$.

    Note: In fact we endow $\pi_E^{-1}(x)$ by requiring that this bijection be a linear iso, and the resulting structure does not depend on the choice of $u$ because we the action of $G$ on $V$ is linear.
    \item Tensorial $r$-forms on $P$ of type $(\rho,V)$ are in 1:1 correspondence with $E$-valued $r$-forms on $M$.
    \begin{itemize}
        \item Given a tensorial $r$-form on $P$, we construct $\phi'\in\Omega^r(E)$ in the following way: for $X_1,...,X_r\in T_xM$, $\phi'(X_1,...,X_r):=u\phi(X^*_1,...,X^*_r)$ where $u$ is any element of $P$ in the fiber above $x$ and $X^*_1,...,X^*_r$ are vectors tangent to $u$ that project onto $X_1,...,X_r$ (note we could make a canonical choice by requiring $X^*_1,...,X^*_r$ to be horizontal).

        Because $\phi$ is horizontal, the choice of $X^*_1,...,X^*_r$ doesn't matter. Because $\phi$ is pseudotensorial, the choice of $u$ doesn't matter.

        \item Conversely if we have an $E$-valued form $\phi'$ on $M$, we can construct a tensorial form $\phi$ in the following way: for $X_1,...,X_r$ tangent to $u\in P$, $\phi(X_1,...,X_r):=u^{-1}\phi'(\pi(X_1),...,\pi(X_r))$.

        $\phi$ will be horizontal because $VP=\ker\pi$, $\phi$ will be pseudotensorial of type $(\rho,V)$ because we use $u^{-1}$.

        Note: From a pseudotensorial form $\phi$ we can always construct a $E$-valued form $\phi'$ (by lifting to horizontal vectors), this corresponds to the fact we can always use the horizontal projection $h:TP\to Q$ to construct a tensorial form $h\phi$ from $\phi$. And indeed, $h\phi$ will be the tensorial form recovered from $\phi'$ (and not $\phi$) - forms obtained on $P$ in this way must always be horizontal.
    \end{itemize}
\end{itemize}

Covariant exterior derivative and horizontal projection.
\begin{itemize}
    \item Given a pseudotensorial form $\phi$ of type $(\rho,V)$ on $P$:
    \begin{itemize}
        \item $h\phi$ defined by $(h\phi)(X_1,...,X_r)=\phi(hX_1,...,hX_r)$ is a  tensorial form of type $(\rho,V)$. This works because $hR_a=R_ah$ for any $a\in G$.
        \item $d\phi$ is pseudotensorial of type $(\rho,V)$. This works because $d$ commutes with $R_a$ and with $\rho(a^{-1})$ (since it is linear).
        \item Thus we can define the tensorial form $D\phi:=(d\phi)h$. This operation is called exterior covariant differentiation.

        Note: Exterior covariant differentiation manifests on the level of $E$-valued forms through the covariant derivative induced by $\Gamma$ on $E$.
    \end{itemize}
\end{itemize}

\subsection{Curvature form and Identities}

\begin{itemize}
    \item Given a principal connection $\Gamma$ on $P(M,G)$, the connection form $\omega$ is a pseudotensorial form of type $adG$, meaning it is of type $(ad,\mathfrak{g})$ with $ad$ the adjoint representation of $G$ on $\mathfrak{g}$.

    The curvature form of $\Gamma$ is the exterior covariant derivative of $\omega$, ie $\Omega:=D\omega$, in particular it is a tensorial form.

    Note: The curvature form also appears through $D^2$ (at least on the level of $E$-valued forms) => tied to the fact that the chain of tensorial forms is not a complex => Geometric relevance?
    \item Structure equation, we have $\Omega(X,Y)=d\omega(X,Y)+[\omega(X),\omega(Y)]$ (a factor $\frac 12$ appears in front of $[,]$ if we use KN's conventions for the wedge product).

    Equivalently we can write $\Omega=d\omega+\frac 12 [\omega,\omega]$.

    $[\omega,\omega]$ is sometimes written $[\omega\wedge\omega]$, and the operation $[,]$ is defined as the unique bilinear operation $\Omega(P,\mathfrak{g})$ verifying $[g\otimes\alpha,h\otimes\beta]=[g,h]\otimes\alpha\wedge\beta$ for $g,h\in\mathfrak{g}$ and $\alpha,\beta\in\Omega(P,\mathbb{R})$. The important fact is that from $\alpha\wedge\beta(X,Y)=\alpha(X)\beta(Y)-\alpha(Y)\beta(X)$ we can deduce $[\omega,\omega](X,Y)=2[\omega(X),\omega(Y)]$.

    Proving the structure equation (we assume that $X,Y$ are tangent to a point in $P$):
    \begin{itemize}
        \item The equation holds when $X,Y$ are horizontal, in that case $\Omega(X,Y)=d\omega(X,Y)$ by def and $[\omega(X),\omega(Y)]=0$ since $\omega$ vanishes on horizontal vectors.
        \item The equation holds when $X,Y$ are vertical. We assume $X,Y$ are pointwise evaluations of $A^*$ and $B^*$ for $A,B\in\mathfrak{g}$. Then $\Omega(A^*,B^*)=0$ since $\Omega$ is horizontal, and by the intrinsic formula for the exterior derivative we get $d\omega(A^*,B^*)=A^*\omega(B^*)-B^*\omega(A^*)-\omega([A^*,B^*])=-\omega([A,B]^*)=-[A,B]$ since $\omega(A^*)=A$, $\omega(B^*)=B$ are constant. And finally, since $[\omega(A^*),\omega(B^*)]=[A,B]$, the RHS vanishes as well.
        \item The equation holds when $X$ is horizontal and $Y$ is vertical. We extend $X$ into a horizontal vector field $X$ on $P$ and $Y$ into a fundamental vector field $A^*$. Then once again the LHS vanishes since $\Omega$ is horizontal, and on the RHS $[\omega(X),\omega(A^*)]$ vanishes since $\omega(X)=0$ and we're left with $d\omega(X,A^*)=X\omega(A^*)-A^*\omega(X)-\omega([X,A^*])=-\omega([X,A^*])$.

        We conclude by showing that $[X,A^*]$ is horizontal. Intuitively this is because the action of $G$ on $\mathfrak{X}(P)$ preserves horizontal fields, and $[-,A^*]$ is an infinitesimal element of that action. More formally, $[X,A^*]=\lim_{t\to 0}\frac 1t (R_{a_t}X-X)$ is horizontal because $R_{a_t}X-X$ is horizontal for all $t$.
    \end{itemize}
    \item A corollary: in the case when $X,Y$ are horizontal we have $\Omega(X,Y)=-\omega([X,Y])$.

    Note: this shows that $\Omega$ measures the failure of $Q$ (horizontal distribution) to be involutive, in particular $Q$ is involutive iff $\Omega=0$.
\end{itemize}

Structure equation in basis.
\begin{itemize}
    \item Once we choose basis $(e_i)$ of $\mathfrak{g}$ we can write $\omega=\omega^ie_i$, $\Omega=\Omega^ie_i$ for unique 1-forms $\omega^i$ and 2-forms $\Omega^i$.

    We also introduce the structure constants defined by $[e_j,e_k]=c^i_{jk}e_i$.
    \item The structure equation then reads $\Omega^i=d\omega^i+\frac 12 c^i_{jk}\omega^j\wedge\omega^k$
\end{itemize}

Bianchi's identity
\begin{itemize}
    \item Using the structure equation, we can show that $D\Omega=0$.

    Sketch of proof:
    \begin{itemize}
        \item We need only show that $d\Omega(X,Y,Z)=0$ for $X,Y,Z$ horizontal.
        \item By considering $\mathfrak{g}$-valued forms of the form $A\otimes\alpha$ with $A\in\mathfrak{g}$ and $\alpha\in\Omega(P)$, we can show $d[\alpha,\beta]=[d\alpha,\beta]-[\alpha,d\beta]$ for $\alpha,\beta\in\Omega(P,\mathfrak{g})$.
        \item Thus using the structure equation, we see that $d\Omega=\frac12([d\omega,\omega]-[\omega,d\omega])$ (since $d^2\omega=0)$, and using the fact that $\omega$ vanishes on horizontal fields, we deduce $d\Omega(X,Y,Z)=0$ for $X,Y,Z$ horizontal.
    \end{itemize}
\end{itemize}

Exterior covariant derivative of tensorial forms of type $adG$.
\begin{itemize}
    \item Let $\phi$ be a tensorial 1-form of type $adG$, then $D\phi(X,Y)=d\phi(X,Y)+[\phi(X),\omega(Y)]+[\omega(X),\phi(Y)]$.

    This can be shown in a case by case fashion similar to the one we used in the proof of the structure equation.
\end{itemize}

\section{Mappings of Connections}

\chapter{Linear Connections}

\section{Geodesics}

\begin{itemize}
    \item A curve $\tau=x_t$ on a manifold $M$ with an affine connection if the velocity $X=\dot{x}_t$ is parallel along $\tau$. That is if $\nabla_XX=0$.
    \begin{itemize}
        \item It is good to remember that a geodesic is a parametrized curve and therefore shouldn't be strictly identified with its image. But using the fact that parallel transport is not affected by changes of parametrization, we can show that the parametrization making a curve into a geodesic is unique up to affine transformations.
    \end{itemize}
    \item A useful fact is that given any affine connection, there exists a unique torisonless connection with the same geodesics. This is because what appears in the differential equation defining geodesics is the symmetrization of the Christoffel symbols.
\end{itemize}


\subsection{Geodesics and Integral curves}

\begin{itemize}
    \item With standard horizontal vector fields: The projection onto M of an integral curve of a standard horizontal vector field on $L(M)$ is a geodesic, and conversely any geodesic is obtained this way.

    The correspondence is not 1:1 though, as a given geodesic can be "lifted" as an integral curve of a standard vector field in many different ways, which is an effect of the "redundancy" within the standard horizontal vector fields one can already notice via the $G$-action.
    \item We can arguably obtain a more satisfying correspondence by introducing the geodesic spray on $TM$. The geodesic spray $W$ is a vector field on $TM$ defined as the unique horizontal vector field st $\pi(W_v)=v$ for any $v\in TM$.

    One can show that there is a 1:1 correspondence between geodesics and integral curves of $W$: a geodesic $x_t$ is sent to $\dot{x}_t$, a curve in $TM$ which can be shown to be an integral curve of $W$, and conversely an integral curve $X_t$ of $W$ is projected onto $x_t:=\pi(X_t)$ which can be shown to be a geodesic.
\end{itemize}

Additional facts on the geodesic spray:
\begin{itemize}
    \item After choosing $\xi\in\mathbb{R}^n$, we get a (in general non-surjective) smooth map $q_{\xi}:L(M)\to TM,u\to u\xi$. $B(\xi)$, the standard horizontal vector field associated to $\xi$ is the unique vector field on $L(M)$ that is $q_{\xi}$-related to $W$ (do we need to require that $B(\xi)$ be horizontal?).

    \item It is interesting to investigate the relationship between geodesic sprays and affine connections. The data encoded in the geodesic spray is the symmetrization of the Christoffel symbols. In that way, two torsionless connection with the same geodesic spray will be equal, and for any affine connection, there is a unique torsionless connection with the same geodesic spray.

    \item We also obtain that $(M,\nabla)$ is geodesically complete iff the geodesic spray $W$ is complete as a vector field on $TM$.
    
\end{itemize}

\subsection{Normal coordinates}

The easy part is showing the existence of normal coordinates and the expression that the connection takes in such coordinates. We also get the useful formula for the exterior derivative in terms of a torsionless affine connection.
\begin{itemize}
    \item Using the geodesic spray, we can show that for $X\in T_xM$ there is a unique maximal geodesic $\gamma_X$ such that $\dot{\gamma}_X(0)=X$ (any geodesic starting at $x$ with the same initial velocity will be a restriction of $\gamma_X$).
    \item We can define the exponential map $\exp:TM\to M$ which sends $X\in T_xM$ to $\gamma_X(1)$.
    \begin{itemize}
        \item $\exp$ is the projection of $\theta_1:TM\to TM$, with $\theta$ the flow of the geodesic spray $W$ on $TM$, so in general (unless the connection is geodesically complete), $\exp$ will not be defined on $TM$.
        \item Through the zero section, we can see $M$ as a submanifold of $TM$ in a natural way. Since $W$ vanishes on $M\subseteq TM$, $\mathbb{R}\times M$ is included in the domain of $\theta_1$, and thus $\exp$ must be defined on a nbd $N$ of $M$ (should work).
        \item Via the inversion theorem we can also straightforwardly show that for $x\in M$ there will exist a nbd $U_x$ of $x$ in $T_xM$ mapped diffeomorphically onto a nbd $N_x$ of $x$ in $M$ by $\exp$. We say that such a diffeo. gives us normal coordinates (they are indeed coordinates once we choose a basis of $T_xM$), and that $N_x$ is a normal coordinate nbd.
    \end{itemize}
    \item The simple expression of geodesics starting from $x$ in normal coordinates centered at $x$.
    \item For a torsionless connection, the Christoffel symbols vanish at the origin.

    From this we get that $d\omega=A(\nabla\omega)$ for a differential form $\omega$ and a torsionless connection $\nabla$, where $A$ denotes the antisymmetrization.
\end{itemize}

The following statement is of fundamental importance but its proof is pretty analytical (the existence of convex nbd's).
\begin{itemize}
    \item Let $(x^1,...,x^n)$ be a normal coordinate system with origin $x_0$. Let $U(x_0,\rho)$ be the nbd of $x_0$ defined by $\sum_i (x^i)^2<\rho^2$. There exists $a>0$ st for any $0<\rho<a$,
    \begin{itemize}
        \item $U(x_0,\rho)$ is (geodesically) convex in the sense that any two points of $U(x_0,\rho)$ can be joined by a geodesic lying in $U(x_0,\rho)$.
        \item Each point of $U(x_0,\rho)$ has a coordinate nbd containing $U(x_0,\rho)$.
    \end{itemize}
\end{itemize}

\section{Equivalence Problem}

Main ref: KN, and the main objective is proving the characterization of affine locally symmetric spaces.

In this section we show how one can retrieve the affine connection on a manifold from different data sets. Each of these methods give us sufficient conditions for the existence of a local affine iso. 

If we combine the existence of a local affine iso. with certain results on the extension of such iso's (which are valid most of the time if the affine manifold is simply connected and geodesically complete), we can extend these local iso's into unique global affine iso's. One moral that seems to emerge is that simply connected, complete affine manifolds are very rigid, especially if some "parallism" conditions are added (on the torsion and curvature typically).

\begin{itemize}
    \item A recurring notion throughout this section is that of an adapted vector field. Given a normal coordinate nbd $U$ centered at $x\in M$ (the corresponding nbd of $0$ in $T_xM$ must be starshaped) and $X\in T_xM$, for $y\in U$ we define $X^*_y\in T_yM$ as the tangent vector obtained by parallel transporting $X$ along the unique geodesic going from $x$ to $y$.

    The resulting map $X^*$ is a smooth vector field on $U$ which we call the vector field adapted to $X$.

    Once we choose a basis $X_1,...,X_n$ of $T_xM$, which gives us normal coordinates $x^1,...,x^n$, the family $X^*_1,...,X^*_n$ is a local frame over $U$, so it is a local section $\sigma:U\to L(M)$. We call it the cross section adapted to the normal coordinate system $(x^1,...,x^n)$.

    Notice that $\sigma$ is not the usual section of $L(M)$ induced by any coordinate chart.

    To avoid problems of star-shapedness, we will assume all our normal coord. nbd's to be balls, that is to be of the form $|x^i|<\delta$ for some $\delta>0$ and some normal coordinate system $x^1,...,x^n$.
    \item (The 1st statement)

    In terms of construction: Given a normal coordinate system $x^1,...,x^n$ on $U$ centered at $x_0\in M$, with adapted frame $\sigma$, the connection on $U$ is determined by $\sigma$, $R|_U$ and $T|_U$ (the torsion and curvature.

    In terms of diffeo: We now have two affine manifolds $(M,\nabla)$, $(M',\nabla')$, two normal coordinates systems $(x^1,...,x^n)$, $(y^1,...,y^n)$ on $U\subseteq M$ and $V\subseteq M'$ respectively, centered at $x_0$ and $y_0$, with adapted frames $\sigma$, $\sigma'$. A diffeo $f:U\to V$ is an affine iso iff it preserves the torsion, curvature and adapted frames (ie $\sigma'(f(x))=f(\sigma(x))$.

    Question: shouldn't we have $y^i\circ f=x^i$ or at least $f(x_0)=y_0$?

    Note: Important part is the construction, that the connection $\omega$ can be recovered from $R$, $T$ and $\sigma$. Adding $\sigma$ does add information since $\sigma$ is constructed using the connection (via parallel transport).

    Sketch of proof. We only need to prove the construction statement, and this is done essentially by exhibiting a differential equation.
    \begin{itemize}
        \item We introduce $\theta=(\theta^i)$, $\omega=(\omega^i_j)$ the canonical and connection forms, as well as $\Theta=(\Theta^i)$, $\Omega=(\Omega^i_j)$ the torsion and curvature forms.

        We pull them back and define:
        \begin{itemize}
            \item $\bar{\theta}^i=\sigma^*\theta^i=A^i_jdx^j$.
            \item $\bar{\omega}^i_j=\sigma^*\omega^i_j=B^i_{jk}dx^k$.
            \item $\bar{\Theta}^i=\sigma^*\Theta^i=\bar{T}^i_{jk}\bar{\theta}^j\wedge\bar{\theta}^k$.
            \item $\bar{\Omega}^i_j=\sigma^*\Omega^i_j=\bar{R}^i_{jkl}\bar{\theta}^k\wedge\bar{\theta}^l$
        \end{itemize}

        Note 1: $\bar{T}^i_{jk}=\sigma^*\tilde{T}^i_{jk}$, $\bar{R}^i_{jkl}=\sigma^*\tilde{R}^i_{jkl}$ (cf section "expression in local coordinates", yet to do).

        Note 2: We can express $\Theta$ and $\Omega$ in terms of $\theta$ because the $n^2+n$ forms $\omega^i_j$,  $\theta^i$ define an absolute parallelism on $L(M)$, and because $\Omega$, $\Theta$ are tensorial.

        \item We fix an arbitrary $a=(a^1,...,a^n)$ and set
        \begin{itemize}
            \item $\hat{A}^i_j(t):=tA^i_j(ta)$, $\hat{B}^i_{jk}(t):=tB^i_{jk}(ta)$.
            \item $\hat{T}^i_{jk}(t):=\bar{T}^i_{jk}(ta)$, $\hat{R}^i_{jkl}(t):=\bar{R}^i_{jkl}(ta)$
        \end{itemize}
        \item We show that the functions $\hat{A}^i_j(t)$ and $\hat{B}^i_{jk}(t)$ verify the system of ODE's:
        \begin{itemize}
            \item $d\hat{A}^i_j(t)/dt=\delta^i_j+\hat{B}^i_{jk}(t)a^k+\hat{T}^i_{kl}(t)\hat{A}(t)^k_j a^l$.
            \item $d\hat{B}^i_{jk}(t)/dt=\hat{R}^i_{jlm}(t)\hat{A}^l_{k}(t)a^m$
        \end{itemize}
        with initial condition.
        \begin{itemize}
            \item $\hat{A}^i_j(0)=0$, $\hat{B}^i_{jk}(0)=0$.
        \end{itemize}
        \item We conclude with Cauchy's theorem: $R$ and $T$ determine the coefficients of the system of ODE ($\hat{R}$, $\hat{T}$), and the initial conditions are simply $0$, so once $R$ and $T$ are given, we have a unique solution to the system of ODE.
    \end{itemize}
    \item (The 2nd statement)

    In terms of construction: Let $M$ be an analytic manifold with analytic linear connection $\nabla$, we can recover the connection on a nbd of $x_0\in M$ from $\nabla^m T_{x_0}$, $\nabla^m R_{x_0}$, $m=0,1,2,...$.

    In terms of diffeo: Let $M$, $M'$ be analytic manifolds with analytic linear connections. If a linear iso. $F:T_{x_0}M\to T_{y_0}M'$ maps the tensors $\nabla^m T_{x_0}$, $\nabla^m R_{x_0}$ into the tensors $\nabla^mT'_{y_0}$, $\nabla^mR'_{y_0}$ for $m=0,1,2,...$. Then there is an affine iso of a nbd $U$ of $x_0$ onto a nbd $V$ of $y_0$ such that $f(x_0)=y_0$, and st that the differential of $f$ at $x_0$ is $F$.

    Sketch of proof: Once again the core thing is proving the construction part. We still describe how the affine iso is constructed: we choose normal coordinates $(x^1,...,x^n)$ on $U$ centered at $x_0$ then normal coordinates $(y^1,...,y^n)$ on $V$ centered at $y_0$ and determined by $\partial_{y^i}|_{y_0}=F(\partial_{x^i}|_{x_0})$. Then the diffeo $f:U\to V$ is defined by $y^i\circ f=x^i$.
    \begin{itemize}
        \item We choose a normal coordinate system $x^1,...,x^n$.
        \item The 1st core observation is that, using the notation of the previous proposition, the $\nabla^m T_{x_0}$, $\nabla^m R_{x_0}$ will determine all the derivatives of $\hat{T}^i_{jk}(t)$ and $\hat{R}^i_{jkl}(t)$ at $t=0$.

        Since $R$ and $T$ are analytic, this determines $R$ and $T$.
        \item To exhibit the relationship btw the $\nabla^m T$, $\nabla^m R$ and the derivatives of the $\hat{T}$, $\hat{R}$, we use the following property:

        \begin{itemize}
            \item Let $u_t$ be the horizontal lift of a curve $x_t$ on $M$, to $L(M)$. To a tensor $K$ of type $(r,s)$ we associate the $T^r_s$-valued function $\tilde{K}$ function defined along $u_t$ by $\tilde{K}(t):=u_t^{-1}K_{x_t}$ (note this is simply the $T^r_s$-valued equivariant function on $L(M)$ corresponding to $K$ evaluated along $u_t$).

            Then $d\tilde{K}(t)/dt=u_t^{-1}(\nabla_{\dot{x}_t}K)$.
        \end{itemize}
        \item Then, using the system of ODE, we see that knowing $\hat{R}$ and $\hat{T}$ allows us to recover $\bat{\theta}^i$ and $\bar{\omega}^i_j$. So to recover the connection, the only missing ingredient is the adapted frame $\sigma$.
        \item And precisely, the second core observation is that the $\theta^i$ determine $\sigma$. More generally, given two sections $\sigma,\sigma':U\to L(M)$, if $\sigma^*\theta=\sigma'^*\theta$, then $\sigma=\sigma'$.

        To see how $\sigma^*\theta$ determines $\sigma$:
        \begin{itemize}
            \item For $X\in T_xM$ ($x\in u$) we have $(\sigma^*\theta)(X)=\theta(\sigma(X))=\sigma(x)^{-1}\pi(\sigma(X))=\sigma(x)^{-1}X$, so $\sigma(x)^{-1}X=(\sigma^*\theta)(X)$.
        \end{itemize}
    \end{itemize}

    \item (Finally the statement we need for locally symmetric spaces)

    In terms of construction: Let $M$ be a smooth mfd with linear connection $\nabla$. If $\nabla T=0$ and $\nabla R=0$, then for $x_0\in M$, knowing $T_{x_0}$ and $R_{x_0}$ allows us to recover the connection in a nbd of $x_0$.

    In terms of diffeo: Let $M$, $M'$ be smooth mfds with linear connections $\nabla$, $\nabla'$. Assume $\nabla T$, $\nabla R$, $\nabla' T'$, $\nabla' R'$ all vanish. Then given a linear iso $F:T_{x_0}M\to T_{y_0}M'$ sending the tensors $T_{x_0}$, $R_{x_0}$ into the tensors $T'_{x_0}$, $R'_{x_0}$, there exists an affine iso $f:U\to V$ of a nbd $U$ of $x_0$ onto a nbd $V$ of $y_0$ st $f(x_0)=y_0$ and the differential of $f$ at $x_0$ is equal to $F$.

    Sketch of proof of the construction:
    \begin{itemize}
        \item $\nabla R=0$ and $\nabla T=0$ imply that $\hat{T}$ and $\hat{R}$ are constant and thus determined by $T_{x_0}$, $R_{x_0}$ we then conclude as in the previous statements, that is:
        \begin{enumerate}
            \item Via the differential equation, $\hat{R}$ and $\hat{T}$ determine $\bar{\theta}$ and $\bar{\omega}$.
            \item $\bar{\theta}=\sigma^*\theta$ determines $\sigma$.
            \item $\bar{\omega}=\sigma^*\omega$ and $\sigma$ determine $\omega$.
        \end{enumerate}
    \end{itemize}
\end{itemize}

Other stuff not covered for now:
\begin{itemize}
    \item Linear connection invariant by parallelism:
    \begin{itemize}
        \item Characterized by $\nabla R=0$, $\nabla T=0$.
        \item Analytic, and connection is analytic using normal coordinate systems.
    \end{itemize}
    \item Additional results for simply connected complete affine manifolds.
\end{itemize}

\part{Sekigawa Vanhecke}

\chapter{Intro}

The objective of this section is to give a complete overview of Sekigawa and Vanhecke's theorem, which states that an almost hermitian manifold whose geodesic symmetries are symplectic (ie preserve the Kahler form) is a locally symmetric Kahler manifold, showing in particular that a Kahler manifold is locally symmetric iff its geodesic symmetries are symplectic.

That last statement follows from S and V's theorem, as well as the fact that if a Kahler mfd is locally symmetric as an affine mfd, its geodesic symmetries are automatically isometric, symplectic and holomorphic (so unitary).

\chapter{Complex Manifolds and Symmetric Spaces}

Here we mainly wish to show that if a Kahler mfd is locally symmetric as an affine mfd, then its geodesic symmetries are symplectic and holomorphic.

Our reference is once again KN. It contains far too many beautiful results, and a large portion of them won't be mentioned here (for now at least).

\section{Complex Manifolds}

\subsection{Complex and Almost Complex Structures}

\begin{itemize}
    \item An almost complex structure on a manifold $M$ is a bundle morphism $J:TM\to TM$ (ie a $(1,1)$-tensor field) such that $J^2=-1$.

    A mfd together with an almost complex structure is an almost cx mfd.
    \item An almost cx mf is always even-dimensional and orientable. The almost cx structure induces a canonical orientation.

    This follows from the fact that a real vector space $V$ with a cx structure $J$ (ie an endomorphism of $V$ st $J^2=-1$) is always even-dimensional AND all elements of $Gl(V,J)$ (ie elements $A$ of $Gl(V)$ st $AJA^{-1}=J$ have positive determinant).

    As for the canonical orientation of $(V,J)$, if $V$ is of real dimension $2n$, we can always find a cx iso $(V,J)\sim(\mathbb{C}^n,J_0)$ with $J_0$ the stdrd cx structure of $\mathbb{C}^n$. We pull the standard orientation of $\mathbb{C}^n$ (corresponding to its standard basis $(e_1,...,e_n,ie_1,...,ie_n)$) along this iso. As shown before, the resulting orientation on $V$ won't depend on the choice of cx iso $(V,J)\sim(\mathbb{C}^n,J_0)$.
    \item A complex manifold $M$ admits a canonical almost complex structure $J$ defined by $J\partial_{x^j}=\partial_{y^j}$ and $J\partial_{y^j}=-\partial_{x^j}$ for holomorphic coordinates $z^j=x^j+iy^j$.

    Not every almost cx structure is induced by an underlying cx mfd. If that is the case, we say that the almost cx structure is integrable.

    If the almost cx structure is integrable, then the underlying cx structure is unique. This follows from the fact that a map $f:M\to N$ between cx mfds is holomorphic iff it is almost cx (ie if $df\circ J=J\circ df$).
    \item The torsion, or Nijenhuis tensor of an almost cx structure $J$ is defined as $N(X,Y)=[JX,JY]-J[JX,Y]-J[X,JY]-[X,Y]$ (also commonly defined as the opposite of that).

    An almost cx structure is integrable iff its torsion vanishes.
\end{itemize}

\subsection{Connections on almost complex manifolds}

The result most interesting to us here is the fact that if an almost cx structure $J$ is parallel for a torsionless connection, then $J$ is integrable.
\begin{itemize}
    \item A linear connection $\nabla$ is almost cx if $\nabla J=0$ which is equivalent to $\nabla_X(JY)=J\nabla_XY$ for any $X,Y\in\mathfrak{X}(M)$.
    \item Any almost cx mfd admits an almost cx connection st its torsion $T$ is given by $N=8T$.
    \item Given an almost cx connection $\nabla$ on $(M,J)$ with torsion $T$, we have the identity $N(X,Y)=T(X,Y)+JT(JX,Y)+JT(X,JY)-T(JX,JY)$.

    In particular, if an almost complex structure $J$ is parallel for a torsionless connection $\nabla$ (ie $\nabla J=0$), then $J$ is integrable.

    Note: the formula is straightforwardly proven using $T(X,Y)=\nabla_XY-\nabla_YX-[X,Y]$.
    \item The curvature of an almost cx connection also verifies $J\circ R(X,Y)=R(X,Y)\circ J$.
\end{itemize}

\subsection{Hermitian metrics and Kaehler metrics}

Hermitian metric on an almost cx mfd, and basic identity.
\begin{itemize}
    \item An hermitian metric on an almost cx mfd $(M,J)$ is a riemannian metric $g$ on $M$ st $g(JX,JY)$. An almost cx mfd together with a hermitian metric is called an almost hermitian mfd.

    Note: any almost cx mfd admits a hermitian metric. This is straightforwardly shown once one notices that given any metric $g$, $h(X,Y):=g(X,Y)+g(JX,JY)$ will be a hermitian metric.

    Note: The existence of an almost cx structure for a given riemannian metric is more complicated though, obstructions occur (the basic one being even dimension and orientation), this can be compared to the difficulty of obtaining a symplectic structure on a manifold.

    Though, once we have a symplectic structure (ie obstructions for that have been overcome) we have a plethora of compatible almost cx structures.

    Note: any hermitian metric comes from a "cx hermitian metric".

    \item Given a hermitian metric $g$ on $(M,J)$ we define the fundamental or Kaehler 2-form $\Phi(X,Y):=g(X,JY)$.

    Some basic properties:
    \begin{itemize}
        \item $\Phi$ is a nondegenerate 2-form.
        \item $\Phi(JX,JY)=\Phi(X,Y)$.
    \end{itemize}

    \item We have the equation $d\Phi(X,JY,JZ)-d\Phi(X,Y,Z)=2g(Y,(\nabla_XJ)Z)+g(X,JN(Y,Z))$.

    Where $\nabla$ is the unique torsionless connection induced by $g$.

    Note: From this equation and the nondegeneracy of $g$, we see that $d\Phi=0$ and $N=0$ imply $\nabla J=0$ (which in turn implies $\nabla \Phi=0$).

    (see details).
\end{itemize}

Kaehler metrics.
\begin{itemize}
    \item An almost hermitian manifold is Kaehler if $J$ and $\Phi$ are integrable (we say that $\Phi$ is integrable if $\Phi$ is closed).
    \item An almost hermitian manifold is Kaehler iff its Levi-Civita connection is almost complex.

    In that case its Levi-Civita connection is also symplectic.

    Sketch of proof:
    \begin{itemize}
        \item If $J$ is parallel for the Levi-Civita connection $\nabla$, then $J$ is integrable since $\nabla$ is torsionless.
        \item Conversely, the almost hermitian manifold is Kaehler, then $d\Phi=0$ and $N=0$, which, by the equation given in the previous paragraph, implies $\nabla J=0$.
        \item Finally, if $J$ is parallel for the Levi-Civita connection, then $\Phi$ is as well since $\Phi$ is obtained from $g$ and $J$ (and $g$ is parallel for the Levi-Civita connection by definition).
    \end{itemize}
\end{itemize}

\section{Symmetric Spaces}

\subsection{Different kinds of locally symmetric spaces and their characterizations}

The most basic one, which underlies all the others, is the affine locally symmetric space.

\begin{itemize}
    \item An affine manifold $(M,\nabla)$ is affine locally symmetric if for every $x\in M$, there exists an open nbd $N_x$ of $x$ for which $s_x:N_x\to N_x$ is an affine iso, where $s_x$ is the local geodesic symmetry about $x$.
    \item An affine manifold is locally symmetric iff $T=0$, $\nabla R=0$.

    Sketch of proof:
    \begin{itemize}
        \item If $(M,\nabla)$ is affine locally symmetric, then $T$ and $\nabla R$ are both preserved by $s_x$. Since $d(s_x)_x=-I_x$, and since $T$ and $\nabla R$ are of odd degree, this implies $T=0$ and $\nabla R=0$.
        \item Conversely, if $T=0$ and $\nabla R=0$, then since for $x\in M$, $-I_x$ sends $R_x$ to $R_x$ (since $R_x$ is of even degree), there must exist a local affine iso $f$ st $f(x)=x$ and $df_x=-I_x$ (see section on equivalence problem).

        Since $f$ is an affine tf, $f$ must commute with $\exp$ and so we obtain $f\exp(X)=\exp(-I_xX)$ so $f=s_x$.
    \end{itemize}
\end{itemize}

The spicy kinds of symmetric spaces.
\begin{itemize}
    \item A riemannian manifold is locally symmetric if the geodesic symmetries of its Levi-Civita connection are isometric. This immediately implies that the geodesic symmetries are affine (since the metric determines the affine connection).
    \item A symmetric symplectic mfd is a mfd $M$ together with an affine connection $\nabla$ and a symplectic form $\omega$ whose geodesic symmetries are affine and symplectic.
    \item An almost hermitian locally symmetric space is an almost hermitian mfd $(M,J,g)$ st the geodesic symmetries of its Levi-Civita connection are holomorphic and isometric (note: we have to require 'holomorphic' since the link btw $g$ and $J$ is tenuous).
    \item The more popular definition:
    
    A hermitian locally symmetric space is a hermitian mfd $(M,J,g)$ (so here $J$ has to be integrable), st the geodesic symmetries its Levi-Civita are isometric and holomorphic.

    We'll see that almost hermitian locally symmetric mfd are exactly hermitian locally symmetric space, which are exactly Kaehler mfds whose geodesic symmetries are isometric.
\end{itemize}

Basic results to characterize them. In particular we'll show that a Kaehler mfd whose locally symmetric as a Riemannian mfd is locally symmetric as a hermitian mfd (so its geodesic symmetries are holomorphic and symplectic).
\begin{itemize}
    \item The basic result, let $M$, $M'$ be 2 connected mfd's, let $\nabla$, $\nabla'$ be affine connections on $M$ and $M'$ resp, and $W$, $W'$ be tensors of the same type on $M$ and $M'$. Assume $\nabla W=0$, $\nabla' W'=0$, let $f:M\to M'$ be an affine iso. and assume that there exists $x_0\in M$ st $df_{x_0}$ sends $W_{x_0}$ on $W'_{y_0}$ (with $f(x_0)=y_0$), then $f$ sends $W$ on $W'$.

    Proof:
    \begin{itemize}
        \item The core observation is that $fW$ is parallel for $\nabla'$ ($f$ is the pushforward of $W$ along $f$, so a tensor on $M'$), since $f\nabla=\nabla'$ and $\nabla W=0$.
        \item We then use $(fW)_{y_0}=W'_{y_0}$ and conclude using the fact that on a connected affine mfd, two parallel tensors which coincide at one point must be equal.

        Indeed for $y\in M'$, take a path $\tau$ joining $y_0$ to $y$. Since $fW$ and $W'$ are parallel, $(fW)_y$, $W'_y$ are both obtained by parallel transporting $(fW)_{y_0}=W'_{y_0}$ along $\tau$, and so must be equal.
    \end{itemize}
    \item As a corollary, we see that for an affine mfd $(M,\nabla)$ with a parallel tensor $W$ of even degree, for $x\in M$, if $s_x$ is an affine tf of a nbd of $x$, then $s_x$ also preserves $W$. This follows from the previous point and the fact that $d(s_x)_x=-I_x$ sends $W_x$ on $W_x$ (since $W$ is of even degree).
    \item From this we immediately obtain:
    \begin{itemize}
        \item A riemannian mfd is locally symmetric as a riemannian mfd iff it is locally symmetric as an affine mfd.
        \item An affine symplectic mfd (so a symplectic mfd together with a symplectic connection) is locally symmetric as an affine symplectic mfd iff it is locally symmetric as an affine mfd.
        \item In particular a Riemannian mfd (resp affine symplectic) is locally symmetric iff $\nabla R=0$ (by def the connection is torsionless in both cases).
    \end{itemize}
    \item The second core observation is the following: let $W$ be a tensor of type $(r,s)$ on an affine mfd $(M,\nabla)$, we say that $s_x$ is preserved by the geodesic symmetries if $s_xW=(-1)^{r+s}W$ on a nbd of $x$ for all $x\in M$. Then $W$ is parallel.

    Note: For tensors of even degree, this coincides with the usual meaning of "being preserved". For tensors of odd degree, being preserved in the usual sense would force $W=0$. This and the previous proposition motivate the definition (maybe we should find another name to avoid confusion).

    Sketch of proof:
    \begin{itemize}
        \item We can use Pierre's formula for the affine connection.
        \item We can also use normal coordinates (we can actually use normal coordinates to prove Pierre's formula). 

        \begin{itemize}
            \item Choose normal coordinates $(x^1,...,x^n)$ ($|x^i|<\delta$ for some $\delta>0$) centered at $x_0$. We'll denote the geodesic symmetry about $x_0$ by $s_0$.  We use $\bar{W}:U\to T^r_s$ to denote the components of $W$ in these coordinates. Using the fact that in normal coordinates $s_0(x^1,...,x^n)=(-x^1,...,-x^n)$, which implies $s_0 W=(-1)^{r+s}(\bar{W}^I_J\circ{s_0})\partial_{\otimes,x^I}dx^{\otimes,J}$, we see that $s_0 W=(-1)^{r+s}W$ is equivalent to $\bar{W}=\bar{W}\circ s_0$.
            \item Once again, using the peculiar form of $s_0$ in these coordinates, we see that $\bar{W}\circ s_0=\bar{W}$ implies that $\partial_{x^i}\bar{W}(0)=0$ for all $i$. Since the Christoffel symbols vanish at $0$, this implies $\nabla W_{x_0}=0$.
        \end{itemize}
    \end{itemize}
    \item Using this, we see that if the geodesic symmetries of an almost hermitian mfd $(M,J,\nabla)$ are holomorphic (preserve $J$, which is of even degree, so the 'old' and 'new' definitions of "being preserved" coincide), then $J$ is parallel for the Levi-Civita connection.

    From this we deduce that $(M,J,\nabla)$ is Kaehler (so in particular hermitian).

    Note: a faster way to see that a locally symmetric almost hermitian mfd must have parallel $J$: for such a mfd, $\nabla J$ is a tensor of odd degree which is preserved by the geodesic symmetries in the 'old' sense, hence it must vanish (here we take advantage of the additional 'affine locally symmetric' hypothesis).
    \item Let $(M,J,g)$ be a Kaehler mfd and assume that $(M,g)$ is locally symmetric as a riemannian mfd. Then $(M,J,g)$ is locally symmetric as a hermitian mfd, meaning its geodesic symmetries are holomorphic (and in particular symplectic).

    With this we concule that the class of locally symmetric almost hermitian mfds, is exactly the class of locally symmetric hermitian mfds, which is exactly the class of Kaehler mfds which are locally symmetric as Riemannian mfds.

    All of these manifolds are thus kaehler, and their geodesic symmetries preserve all the present structures: they are affine, isometric, holomorphic and symplectic.

    Proof:
    \begin{itemize}
        \item All of it follows from the fact that the almost cx (and symplectic) structures of a Kaehler mfd are parallel (a fact proven in an earlier section, using the link between $d\Phi,g,J$) and of even degree.
        \item Thus if $(M,J,g)$ is locally symmetric as an affine mfd (equivalently, as a riemannian mfd), then the geodesic symmetries must preserve $J$ and $\omega$.
    \end{itemize}
\end{itemize}

\chapter{Sekigawa Vanhecke}

\end{document}