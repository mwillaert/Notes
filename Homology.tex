\documentclass[a4paper,12pt,parskip=half*,chapterprefix=true,numbers=noendperiod]{scrreprt}

% LANGUAGES AND SYMBOLS
\usepackage[utf8]{inputenc}
\usepackage[T1]{fontenc}
\usepackage[english]{babel}
\usepackage{lmodern}
\usepackage{xparse}

% EVERYDAY PACKAGES
\usepackage{lipsum}
\usepackage[shortlabels]{enumitem}
\usepackage{verbatim}
\usepackage[normalem]{ulem}
\usepackage[dvipsnames]{xcolor}
\usepackage{csquotes}
\usepackage{pdfpages}
\usepackage{easy-todo}
\usepackage{setspace}
%\usepackage[all]{nowidow}

\usepackage{graphicx}
\usepackage{caption}
\usepackage{subcaption}

% MATHEMATICS
\usepackage{mathtools}
\usepackage{stmaryrd}
\usepackage{amsthm, thmtools}
\usepackage{framed}
\usepackage{nameref}
\usepackage[colorlinks,menucolor=blue,linkcolor=blue, citecolor=blue, urlcolor=blue]{hyperref} 
\usepackage[capitalise]{cleveref}% if problems, load cleveref last
\crefname{subsection}{Subsection}{Subsections}
\Crefname{subsection}{Subsection}{Subsections}
%\usepackage[retainorgcmds]{IEEEtrantools}
\usepackage{amssymb}
\usepackage[mathscr]{euscript}
\usepackage{esint}
\usepackage{esvect}
\usepackage{relsize}

\usepackage{tikz}
\usepackage{tikz-qtree}
\usepackage[framemethod=tikz]{mdframed}
\usetikzlibrary{cd}
\usetikzlibrary{matrix}
\usetikzlibrary{backgrounds}

%Packages for references, Indices

\usepackage{imakeidx}[intoc]
\usepackage{biblatex}
\addbibresource{refs.bib}

%Additionnal Packages

\usepackage{float}
\usepackage{cancel}

%Macors

\newcommand{\opname}{\operatorname}

%Theorems

\newtheorem{theorem}{Theorem}[section]
\newtheorem{proposition}{Proposition}[section]
\newtheorem{lemma}{Lemma}[section]
\newtheorem{corollary}{Corollary}[section]

\theoremstyle{definition}
\newtheorem{definition}{Definition}[section]
\newtheorem{example}{Example}[section]
\newtheorem{notation}{Notation}[section]

\theoremstyle{remark}
\newtheorem*{remark}{Remark}

\title{Homology}
\author{Maxime Willaert}
\begin{document}

\maketitle

\tableofcontents

\part{Homological Algebra, Lectures of Tim Van der Linden \cite{vanderLinden:Homology}}

\chapter{Modules over a ring, exact sequences and homology}

Based on lectures given by Tim Van der Linden in 2022-2023 \cite{vanderLinden:Homology} (might turn out to be a copy).

\section{Modules over a ring}

The rings are not assumed to be commutative, only to possess a unit 1.
\begin{definition}[Ring]
	A \textbf{ring} is an abelian group $R$ (written additively) together with a binary operation
	\begin{equation*}
		\cdot:R\times R\to R,(r,a)\to r\cdot a\overset{n}{=} ra
	\end{equation*}
	such that $(R,\cdot)$ is a monoid with unit $1$, and such that $\cdot$ is distributive over $+$, meaning that for all $a,a',r,r'\in R$
	\begin{enumerate}[(i)]
		\item (Left-distributivity) $r(a+a')=ra+ra'$;
		\item (Right-distributivity) $(r+r')a=ra+ra'$.
	\end{enumerate}
\end{definition}

\begin{example}
	By definition, a monoid is always with unit (or identity). A nonzero ring with commutative multiplication and multiplicative inverses for every non-zero element is called a \textbf{field}. Note that in any ring $R$ we have $0a=(0+0)a=0a+0a$ which implies $0a=0$ (and similarly we have $a0=0$), so unless $0=1$, $0$ cannot admit a multiplicative inverse, and $1=0$ if and only if $R$ is the \textbf{zero ring} (the ring with one element, denoted $0$).
\end{example}

Modules are to rings what vector spaces are to fields.

\begin{definition}[Module]
	Let $R$ be a ring. A \textbf{left $R$-module} is an abelian group $A$ (written additively) together with an operation (scalar multiplication)
	\begin{equation*}
		\cdot:R\times A\to A,(r,a)\to r\cdot a\overset{n}{=}ra
	\end{equation*}
	such that for all $r,r'\in R$, $a,a'\in A$
	\begin{enumerate}[(i)]
		\item (Distributivity) $(r+r')a=ra+r'a$ and $r(a+a')=ra+ra'$;
		\item (Multiplicative compatibility) $(rr')a=r(r'a)$;
		\item (Identity) $1a=a$.
	\end{enumerate}
	
	Given two left $R$-modules, $A$ and $B$, an \textbf{$R$-linear map} (or morphism of $R$-modules) $A\to B$ is a group morphism (for the abelian group structure of $A$ and $B$) $\phi:A\to B$ such that $\phi(ra)=r\phi(a)$ for $r\in R$, $a\in A$. \textbf{Isomorphisms} are define in the usual way (as invertible $R$-linear maps with $R$-linear inverses), it turns out that the inverse of a bijective $R$-linear map is automatically $R$-linear, so that isomorphisms of $R$-modules are exactly the bijective $R$-linear maps.
\end{definition}
\begin{remark}
	We can define \textbf{right $R$-modules} in the obvious way. Given a ring $R$ we can define the \textbf{opposite ring $R^{op}$} with the same underlying set, the same addition and reversed multplication, i.e. $r\cdot_{op}r':=r'r$ for $r,r'\in R$. We can then see that right $R$-modules are exaclty left $R^{op}$-modules. For a commutative ring $R=R^{op}$ and there is no distinction between left and right $R$-modules. In what follows we'll often use "$R$-modules" to refer to \emph{left} $R$-modules.
\end{remark}
\begin{example}
	For a field $\mathbb{F}$, a (left or right) $\mathbb{F}$-module is a \textbf{vector space} over $\mathbb{F}$.
\end{example}
\begin{example}
	For an $R$-module $A$, we have $0a=(0+0)a=0a+0a$ implying $0a=0$. Combining this with the identity ($1a=a$) we see that the only module over $0$ (the zero ring) is the \textbf{zero module} (the module with one element, also denoted $0$).
\end{example}
\begin{example}
	By definition any $\mathbb{Z}$-module comes with an abelian group structure. Conversely, an abelian group $A$ admits a unique scalar product making $A$ into a $\mathbb{Z}$-module. So we see that the $\mathbb{Z}$-modules are the abelian group (there is an isomorphism of category $\text{Ab}\simeq\mathbb{Z}\text{-Mod}$).
\end{example}

\begin{definition}[Submodules]
	Let $A$ be an $R$-module. Given a subset $S$ of $A$, we say that $S$ is a \textbf{submodule} (or $R$-submodule) of $A$ if $S$ admits an $R$-module structure for which the injection $\iota:S\hookrightarrow A$ is $R$-linear. We can show that $S$ is a submodule of $A$ if and only if
	\begin{enumerate}[(i)]
		\item $S$ is a subgroup of $A$ (automatically abelian);
		\item For all $r\in R$, $s\in S$, $rs\in S$.
	\end{enumerate}
	In that case the $R$-module structure for which $\iota:S\hookrightarrow A$ is $R$-linear is unique, and obtained by restricting the operations of $A$ to $S$.
\end{definition}

\begin{proposition}
	Submodules are stable by intersection, meaning that for an $R$-module $A$ and a family $(S_i)_{i\in I}$ of submodules of $A$, $\bigcap_{i\in I}S_i$ is a submodule of $A$.
\end{proposition}

\begin{definition}[Quotient by a submodule]
	Let $A$ be an $R$-module, and $S$ be a submodule of $A$. For $a,a'\in A$ we write $a\sim_S a'$ if $(a-a')\in S$. $\sim_S$ is then an equivalence relation on $A$, and we can define the quotient $q:A\to A/S:=A/\sim_S$. The \textbf{quotient} of $A$ by $S$ is then defined to be $A/S$ equipped with the unique $R$-module structure for which $q:A\to A/S$ is $R$-linear. The image of $a\in A$ by $q:A\to A/S$ (so the equivalence class of $a$ for $\sim_S$) is often denoted by $a+S$.
\end{definition}
\begin{definition}[Kernel, image and cokernel]
	Given an $R$-linear map $\phi:A\to B$. The \textbf{image} $\opname{im}(\phi)$ of $\phi$ is a submodule of $B$, while the \textbf{kernel} of $\phi$ is defined to be submodule $\ker(\phi):=\{a\in A|\phi(a)=0\}$. The \textbf{cokernel} of $\phi$ is the quotient $q:B\to B/\opname{im}(\phi)$ of $B$ by the image of $\phi$.
\end{definition}

\begin{proposition}
	Given an $R$-linear map $\phi:A\to B$
	\begin{enumerate}[(i)]
		\item $\phi$ is injective if and only if $\ker(\phi)=0$;
		\item $\phi$ is surjective if and only if $\opname{coker}(\phi)=0$.
	\end{enumerate}
	In particular $\phi$ is an isomorphism if and only if both $\ker(\phi)$ and $\opname{coker}(\phi)$ are zero.
\end{proposition}

\begin{definition}[Span of a subset]
	Let $X$ be a subset of an $R$-module $A$. The span $\langle X\rangle$ is defined to be the smallest submodule of $A$ containing $X$, so
	\begin{equation*}
		\langle X\rangle:=\bigcap\{S\text{ submodule of }A|X\subseteq S\}.
	\end{equation*}
	$\langle X\rangle$ consists of the (finite) linear combinations of elements of $X$ (another possible definition of $\langle X\rangle$)
	\begin{equation*}
		\langle X\rangle:=\{\sum_{i=1}^kr_ix_i|0\leq k <\infty, r_j\in R, x_j\in X\}.
	\end{equation*}
\end{definition}


\subsection{Free modules}

\begin{definition}[Free over a set]
	Let $A$ be an $R$-module and let $\delta:X\to R$ be a map from a set $X$ to the underlying set of $R$. We say that $R$ is \textbf{free over $X$} (or $\delta$ to be more precise) if for any map $\xi:X\to B$ there exists a unique $R$-linear map $\alpha:A\to B$ such that the following diagram commutes
	\begin{figure}[H]
		\centering
		\begin{tikzcd}
            & A \arrow[d, "\alpha", dotted] \\
			X \arrow[ru, "\delta"] \arrow[r, "\xi"'] & B                            
		\end{tikzcd}
	\end{figure}
\end{definition}

\begin{definition}[The free module over a set]
	Given a set $X$ there exists an $R$-module $R[X]$ together with an set map $\delta:X\to R[X]$ such that $R[X]$ is free over $\delta$. The pair $(R[X],\delta)$ is unique up to isomorphism (as for any other object defined by means of a universal property) and $\delta$ is injective.
\end{definition}
\begin{proof}
We'll construct a \textbf{standard version} of $(R[X],\delta)$. $R[X]$ consists of the \textbf{almost zero} functions $\phi:X\to R$, meaning that the support $\operatorname{supp}\phi:=\{x\in X|\phi(x)\neq 0\}$ is finite, equipped with pointwise addition and scalar multiplication. The injection $\delta:X\to R[X]$ sends $x\in X$ to the indicator function of $x$
\begin{equation*}
	\delta(x)\overset{n}{=}\delta_x\overset{n}{=}x:X\to R,y\to\begin{cases} 1,\text{ if }x=y\\
	0,\text{ if }x\neq y
	\end{cases}
\end{equation*}

Any nonzero element $\phi$ of $R[X]$ is written uniquely $\phi=\sum_{i=1}^kr_ix_i$ for $x_1,...,x_k\in X$ distinct and $r_i=\phi(x_i)\in R-\{0\}$. Given a map $\xi:X\to B$ from $X$ to another $R$-module $B$, the unique factoring map $\alpha:R[X]\to B$ sends $\phi=\sum_{i=1}^kr_ix_i$ to $\sum_{i=1}^kr_i\xi(x_i)$.
\end{proof}

\begin{remark}
	From now on we'll use $(R[X],\delta)$ to refer to the standard free module over $X$.
\end{remark}
\begin{remark}
	The universal property of the free module over $X$ can be stated as follows: for any $R$-module $A$, we have a canonical bijection
	\begin{equation*}
		\text{Set}(X,UA)\simeq\opname{Hom}(R[X],A)
	\end{equation*}
	where $U$ denotes the forgetful functor $U:R\text{-Mod}\to\text{Set}$. So we see that the existence of free modules is equivalent to the existence of a left-adjoint to the forgetful functor.
\end{remark}

\begin{proposition}	
	Given an $R$-module $A$, a set $X$ and a map $\xi:X\to A$. By the universal property the free module, there exists a unique $R$-linear map $\alpha:R[X]\to A$ such that
	\begin{figure}[H]
	\centering
	\begin{tikzcd}
    & {R[X]} \arrow[d, "\alpha", dotted] \\
	X \arrow[ru, "\delta"] \arrow[r, "\xi"'] & A                                 
	\end{tikzcd}
	\end{figure}
	commutes. $A$ is free over $\xi$ if and only if $\alpha$ is an isomorphism. In particular $\xi$ must be injective (for $A$ to be free over $\xi$).
\end{proposition}

\begin{definition}[Basis of a module]
	Let $A$ be an $R$-module. A subset $X\subseteq A$ is a \textbf{basis} of $A$ if $A$ is free over the injection $\iota:X\hookrightarrow A$. In other words, $X$ is a basis of $A$ if and only if for any $R$-module $B$, any map $\phi:X\to B$ extends uniquely to an $R$-linear map $\bar{\phi}:A\to B$. 
\end{definition}

\begin{proposition}
	Given an $R$-module $A$, a set $X$ and a map $\xi:X\to A$, $A$ is free over $\xi$ if and only if $\xi$ is injective and $\opname{im}(\xi)\subseteq A$ is a basis of $A$.
\end{proposition}
\begin{remark}
	Given a set $X$, $X$ is a basis of $R[X]$ (when identified with its image by $\delta$).
\end{remark}

\begin{proposition}
	Let $X$ be a subset of the $R$-module $A$. By the universal property of the free module, there exists a unique map $\alpha:R[X]\to A$ such that
	\begin{figure}[H]
	\centering
	\begin{tikzcd}
    & {R[X]} \arrow[d, "\alpha", dotted] \\
	X \arrow[ru, "\delta"] \arrow[r, "\iota"', harpoon, hook] & A                                 
	\end{tikzcd}
	\end{figure}
	commutes. $X$ is a basis of $X$ if and only if $\alpha$ is an isomorphism.
\end{proposition}

\begin{corollary}
	A subset $X$ of an $R$-module $A$ is a basis of $A$ if and only if
	\begin{enumerate}[(i)]
		\item $X$ is \textbf{linearly independent}, meaning that for $1\leq k$ and $x_j\in X$, $r_j\in R$
		\begin{equation*}
			\sum_{i=1}^kr_ix_i=0
		\end{equation*}
		if and only if $r_1=...=r_k=0$. This is equivalent to requiring that the unique factoring map $\alpha:R[X]\to A$ be injective.
		\item $X$ \textbf{spans} $A$, meaning that $\langle X\rangle=A$ (i.e. that any element of $A$ is a finite linear combination of elements of $X$). This is equivalent to requiring that the unique factoring map $\alpha:R[X]\to A$ be surjective. 
	\end{enumerate}
\end{corollary}

\begin{definition}[Free module]
	An $R$-module $A$ is said to be free if and only if $A$
\end{definition}




\printbibliography

\end{document}