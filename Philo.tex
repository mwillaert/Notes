\documentclass{report}
\usepackage{graphicx} % Required for inserting images
\usepackage[pdfusetitle]{hyperref}
\usepackage{amssymb,amsmath,amsthm}

\theoremstyle{definition}
\newtheorem{definition}{Definition}
\newtheorem{notation}{Notation}

\title{Philo}
\author{Maxime Willaert}
\date{July 2024}

\begin{document}

\maketitle

\tableofcontents

\part{Philo Contemporaine ULB}

\chapter{Enregistrements audios}

\section{Cours 8 février}

Introduction à l'approche généalogique de Nietzsche.
\begin{itemize}
    \item Nietzsche propose une nouvelle position vis-à-vis des concepts philosophiques (et avec cette position il prétend trancher avec l'approche qui dominait jusqu'alors).

    Les concepts étudiés par les philosophes ne doivent plus être vus comme des entités "nécessaires" qui existent hors du temps (dans une sorte de monde parallèle idéal) et que nous découvririons.

    Il faut prendre conscience que les concepts ont été élaborés, créés par nos prédécesseurs et qu'ils ont ensuite évolué au cours du temps, et qu'ils sont contingents (accidentels, ils auraient pu ne pas exister, ne pas être créés ou ne pas se perpétuer).

    Un exemple est fourni: selon le travail du sinologue helléniste François Jullien - "Le sage est sans idée", la notion d'eidos développée par les grecques (qui aurait engendré notre notion d'idée) n'apparait pas dans la philosophie chinoise classique, qui ne s'en porte pas plus mal.
    \item Le temps est donc indissociable du sens (phrase choc), il n'y a pas de concept hors temps.

    Questions:
    \item Une fois qu'on a pris conscience de l'historicité des concepts, Nietzsche nous exhorte à étudier l'histoire de ces concepts, en explorant 3 questions:
    \begin{itemize}
        \item Qui a créé ces concepts?
        \item Comment les a-t-il créé?
        \item Pourquoi les a-t-il créé, animé par quelle force? Comment ces concepts lui ont-il été utiles?
    \end{itemize}

    Une note que j'ai moins comprise: généalogie serait étude de l'histoire d'un concept "à partir du présent". En adoptant un regard contemporain? Ou alors, avec l'objectif principal de mieux comprendre le présent?
    \item Avec cette approche, Nietzche prône donc une certaine forme de scepticisme (v-à-v des concepts philosophiques ici). Il étend ce scepticisme sur le plan axologique (l'étude des valeurs) et nous enjoint à nous \emph{méfier} des concepts.

    Exemple donné: la vérité. Il faut percevoir plus loin que l'aspect innocent (qui se prétend vertueux) du concept, et chercher les motivations cachées.
    \begin{itemize}
        \item Si une personne affirme qu'elle compte tout abandonner pour se consacrer à la recherche de la vérité, on éprouvera un certain respect pour cette personne. On verra dans sa décision une forme de noblesse et de détachement de certaines considérations "plus basses".
        \item Mais Nietzsche nous dit de questionner. Pourquoi cette personne recherche-t-elle la vérité? Est-ce pour dominer, exercer un pouvoir sur les autres ou sur elle-même, pour gagner une forme de maitrise? Est-ce pour pouvoir juger les autres, les condamner? Est-ce un besoin de condamner, de pouvoir prôner une supériorité morale?
    \end{itemize}

    On a vu avant que le temps est indissociable du concept, on retire donc au concept son caractère éternel, ici il semble que l'on retire aussi au concept sa noblesse. Le concept est élaboré dans le temps par l'homme, dans le but de servir l'homme, et de cette manière le concept est indissociable des forces qui animent l'homme, force parfois sombres, souterraines, inconscientes et difficiles à avouer.
    \item Philosophes influencés par Nietzsche:
    \begin{itemize}
        \item Heidegger, Bergson.
        \item Français après les années 60: Foucault.
    \end{itemize}
\end{itemize}

\section{Cours 9 février}

\begin{itemize}
    \item Nietzsche nous exhorte à sortir du point de vue anthropologique.
    \begin{itemize}
        \item Point de vue anthropologique est moderne. Le constat de base et que l'homme ne peut échapper à son point de vue, et qu'il convient donc de ne jamais prétendre sortir de la perspective humaine au cours de nos investigations.
        \item Nietzsche nous enjoint au contraire à abandonner l'étalon humain (étalon moral, cognitif,...).

        Nous devons revenir au sens de la Terre selon lui. À l'immanence de la Terre, qui est, simplement (caractéristique immanente=caractéristique qui dépend seulement de la nature de l'entité).

        Il s'agit de cesser l'anthropomorphisation de la Terre.

        \item Selon Nietzche, la philosophie a été une longue tentative de s'arracher à la Terre, faisant appel à des mondes "supérieurs" d'absolu (platon, christianisme,...), mais nous devons au contraire y revenir.

        De nombreux concepts développés par la philosophie (tous?) sont des tentatives d'humaniser le monde. Exemple: concepts d'identité, de causalité.

        \item La question morale est importante pour Nietzsche. Dans l'anthropomorphisation du monde est inclue la moralisation du monde. L'humain impose Bien et Mal sur la Terre, mais la Terre EST simplement.
        \item Pour Nietzsche le surhomme est un homme qui est revenu au sens de la Terre. Le concept a souvent été mal compris car le surhomme Nietzschéen est l'opposé d'un surhomme transhumain (un homme qui décuple sa puissance pour achever l'anthropomorphisation du monde).

        Pour Nietzsche cet appel au surhomme est un appel à la guérison. Une grande partie de son oeuvre se donne pour objectif d'éliminer un poison qui affecte l'homme moderne.
    \end{itemize}
\end{itemize}

\end{document}