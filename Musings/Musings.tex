\documentclass[a4paper,12pt,parskip=half*,chapterprefix=true,numbers=noendperiod]{scrreprt}

% LANGUAGES AND SYMBOLS
\usepackage[utf8]{inputenc}
\usepackage[T1]{fontenc}
\usepackage[english]{babel}
\usepackage{lmodern}
\usepackage{xparse}

% EVERYDAY PACKAGES
\usepackage{lipsum}
\usepackage[shortlabels]{enumitem}
\usepackage{verbatim}
\usepackage[normalem]{ulem}
\usepackage[dvipsnames]{xcolor}
\usepackage{csquotes}
\usepackage{pdfpages}
\usepackage{easy-todo}
\usepackage{setspace}
%\usepackage[all]{nowidow}

\usepackage{graphicx}
\usepackage{caption}
\usepackage{subcaption}

% MATHEMATICS
\usepackage{mathtools}
\usepackage{stmaryrd}
\usepackage{amsthm, thmtools}
\usepackage{framed}
\usepackage{nameref}
\usepackage[colorlinks,menucolor=blue,linkcolor=blue, citecolor=blue, urlcolor=blue]{hyperref} 
\usepackage[capitalise]{cleveref}% if problems, load cleveref last
\crefname{subsection}{Subsection}{Subsections}
\Crefname{subsection}{Subsection}{Subsections}
%\usepackage[retainorgcmds]{IEEEtrantools}
\usepackage{amssymb}
\usepackage[mathscr]{euscript}
\usepackage{esint}
\usepackage{esvect}
\usepackage{relsize}

\usepackage{tikz}
\usepackage{tikz-qtree}
\usepackage[framemethod=tikz]{mdframed}
\usetikzlibrary{cd}
\usetikzlibrary{matrix}
\usetikzlibrary{backgrounds}

%Packages for references, Indices

\usepackage{imakeidx}[intoc]
\usepackage[backend=biber]{biblatex}
\addbibresource{../refs.bib}

%Additionnal Packages

\usepackage{float}
\usepackage{cancel}

%Macors

\newcommand{\opname}{\operatorname}

%Theorems

\newtheorem{theorem}{Theorem}[section]
\newtheorem{proposition}{Proposition}[section]
\newtheorem{lemma}{Lemma}[section]
\newtheorem{corollary}{Corollary}[section]

\theoremstyle{definition}
\newtheorem{definition}{Definition}[section]
\newtheorem{example}{Example}[section]
\newtheorem{notation}{Notation}[section]

\theoremstyle{remark}
\newtheorem*{remark}{Remark}

\title{Musings}
\author{Maxime Willaert}
\begin{document}

\maketitle

\tableofcontents

\section{Wednesday 28/08/24}

A list of keywords, in the event that I fail to cover the subjects I intend to, due to my slowness.
\begin{itemize}
	\item Normality, wrongness.
	\item No saviour.
	\item The structure of branches in a forest. The fascination for the non-human and the dullness of the artificial.
	\item Relationship to reading and the intellect.
	\item Letting go of this perceived negative outside negative judgment.
\end{itemize}

I hesitated for a while before attempting to write this. The process of recording the thoughts I collect throughout the day is one I usually find tedious (and seldom manage to complete), no matter the value I may assign to those thoughts. On those occasions I burden myself with no task other than that of contemplating the flow of experience, I enter a state of peace, freedom and clarity, wherein I let threads of fragmented sentences mould an intuitive (but never obscure) understanding. In these moments (and perhaps no other), the weight of this mysterious existential dread leaves me, and I feel genuinely, and naturally connected to \textbf{Life}. The act of writing forces me to leave this state, and confronts me with the rambling and chaotic nature of my thought process, as well as with the impossible task of imbuing dead symbols with this ever-moving, ever-branching meaning.

The first hurdle (which has been the last I faced on more than one occasion) is choosing a starting point, knowing that the game of presenting concepts and experiences in a linear manner is an artificial distortion of reality. So let us cut down this obstacle with brutal arbitrariness.

At this stage, it seems to me that my understanding was most enhanced and my suffering most alleviated by letting go of certain fundamental notions that had percolated to the base of my mind and had since then served as the bedrock of my experience, unquestioned (I do not claim to have unearthed all of these notions, not even a substantial portion of them). The two shackles I had to perceive and free myself of (at least partially) were \textbf{normality} and \textbf{wrongness}\footnote{I often wonder if these considerations result from a superficial understanding of Nietzsche.}. Perhaps I should try to provide a definition for these two terms.
\begin{definition}[Normality]
	\textbf{Normality} is what arises when one loses sight of the fundamentally contigent nature of things (maybe most people need not worry about this particular pitfall, but I fall into it recurrently). We stop seeing events, ideas, structures for what they are: products of a long, complex and accidental history. It seems to us that it was \emph{necessary} for the world to be as it is, and contingency is then erroneously replaced by a set of attributes (which usually come hand-in-hand): things are thought to be \emph{eternal} (they have always been as they are, and will remain so for the rest of time), things are thought to be somehow \emph{good} and \emph{natural}, and lastly an alternative to the current state of affair ceases to be conceivable (the question is not even raised).
\end{definition}
\begin{definition}[Wrongness]
	\textbf{Wrongness} is the dread that results from believing that the universe operates under a fixed \emph{teleology} (a notion that ties naturally to that of \emph{normality} as defined above), thus, in a way, it is the negative manifestation of a larger phenomenon (and maybe not the best starting point for an explanation). If one starts with the assumption that, underneath reality, there exists a deep and true \emph{purpose} to be uncovered, from which derive such things as 'meaning' or 'the path a good life must follow', then simultaneously (at least in my personal experience), the fear of standing \emph{outside} of this purpose, of being fundamentally \emph{wrong}, is born.
\end{definition}

In truth purpose is \emph{not} a fundamental property of reality. It is something that living beings such as humans experience and project onto the world. Those living beings that experience purpose experience it because it enhances their ability to persist in existence. And so events are neither purposeful nor purposeless, a life is neither well-lived nor wasted. They simply are (and are \emph{to be observed} in my view). They \emph{are}, in the immense tapestry that is existence.


\end{document}