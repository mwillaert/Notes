\documentclass[a4paper,12pt,parskip=half*,chapterprefix=true,numbers=noendperiod]{scrreprt}

% LANGUAGES AND SYMBOLS
\usepackage[utf8]{inputenc}
\usepackage[T1]{fontenc}
\usepackage[english]{babel}
\usepackage{lmodern}
\usepackage{xparse}

% EVERYDAY PACKAGES
\usepackage{lipsum}
\usepackage[shortlabels]{enumitem}
\usepackage{verbatim}
\usepackage[normalem]{ulem}
\usepackage[dvipsnames]{xcolor}
\usepackage{csquotes}
\usepackage{pdfpages}
\usepackage{easy-todo}
\usepackage{setspace}
%\usepackage[all]{nowidow}

\usepackage{graphicx}
\usepackage{caption}
\usepackage{subcaption}

% MATHEMATICS
\usepackage{mathtools}
\usepackage{stmaryrd}
\usepackage{amsthm, thmtools}
\usepackage{framed}
\usepackage{nameref}
\usepackage[colorlinks,menucolor=blue,linkcolor=blue, citecolor=blue, urlcolor=blue]{hyperref} 
\usepackage[capitalise]{cleveref}% if problems, load cleveref last
\crefname{subsection}{Subsection}{Subsections}
\Crefname{subsection}{Subsection}{Subsections}
%\usepackage[retainorgcmds]{IEEEtrantools}
\usepackage{amssymb}
\usepackage[mathscr]{euscript}
\usepackage{esint}
\usepackage{esvect}
\usepackage{relsize}

\usepackage{tikz}
\usepackage{tikz-qtree}
\usepackage[framemethod=tikz]{mdframed}
\usetikzlibrary{cd}
\usetikzlibrary{matrix}
\usetikzlibrary{backgrounds}

%Packages for references, Indices

\usepackage{imakeidx}[intoc]
\usepackage{biblatex}
\addbibresource{../refs.bib}
\addbibresource{dailyrefs.bib}

%Additionnal Packages

\usepackage{float}
\usepackage{cancel}

%Macors

\newcommand{\opname}{\operatorname}

%Theorems

\newtheorem{theorem}{Theorem}[section]
\newtheorem{proposition}{Proposition}[section]
\newtheorem{lemma}{Lemma}[section]
\newtheorem{corollary}{Corollary}[section]

\theoremstyle{definition}
\newtheorem{definition}{Definition}[section]
\newtheorem{example}{Example}[section]
\newtheorem{notation}{Notation}[section]

\theoremstyle{remark}
\newtheorem*{remark}{Remark}

\title{Log}
\author{Maxime Willaert}
\begin{document}

\maketitle

\tableofcontents

\section{Personal, Sunday 25/08/24}

The foundations of differential geometry.

\subsection{Encounters}

$C^0$ manifolds may not have $C^k$ structures for $k\geq 1$. If such structures exist, several inequivalent structures may arise (e.g. uncountably many inequivalent smooth structures on $\mathbb{R}^4$, or several smooth structures on $\mathbb{S}^7$). For $k\geq 1$, a $C^k$ manifold will have a unique (up to iso.) smooth structure, and even a unique (up to iso) analytic structure (based on the work of Whitney) \cite{Overflow:ManifoldAnalyticStructure,Stack:WhitneyAnalyticEmbedding}.

Properties of analytic manifolds. Smooth geometry relies on a handful of theorems from analysis, mainly the inverse mapping theorem and the standard existence uniqueness and regularity results from analysis. So it is useful to have their analogues in (real or complex) analytic geometry \cite{Narasimhan:AnalysisRealComplexMan,Bourbaki:VarietesDiffAnalytic}.

Defining tangent spaces as quotients of the ring of germs (an approach closer to algebraic geometry?) \cite{Warner:DiffMan}.

Categorical approach to manifolds (fibered manifolds, fiber bundles, transversal map and the pullback,...) \cite{KMS:NatDiffGeo}.

General topology \cite{Munkres:Top}. Different axioms, filters and nets (also related to logic and Stone spaces), convergence spaces. \cite{Wiki:AxiomTop,Wiki:FiltersTop,Wiki:ConvergenceSpace}. Categorical approach to topology (quotients, products, subspaces,...) \cite{Lee:IntTopMan}.

\subsection{Writing}

In \cite{personal:BasicsDiffGeo}, topology: basic definitions, maps between topological spaces ((local) homeos, open and closed maps, characterizing continuous maps), generating topologies (link with lattices \cite{personal:Lattices}, metrizable spaces and convergence.





\printbibliography

\end{document}