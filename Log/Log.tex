\documentclass[a4paper,12pt,parskip=half*,chapterprefix=true,numbers=noendperiod]{scrreprt}

% LANGUAGES AND SYMBOLS
\usepackage[utf8]{inputenc}
\usepackage[T1]{fontenc}
\usepackage[english]{babel}
\usepackage{lmodern}
\usepackage{xparse}

% EVERYDAY PACKAGES
\usepackage{lipsum}
\usepackage[shortlabels]{enumitem}
\usepackage{verbatim}
\usepackage[normalem]{ulem}
\usepackage[dvipsnames]{xcolor}
\usepackage{csquotes}
\usepackage{pdfpages}
\usepackage{easy-todo}
\usepackage{setspace}
%\usepackage[all]{nowidow}

\usepackage{graphicx}
\usepackage{caption}
\usepackage{subcaption}

% MATHEMATICS
\usepackage{mathtools}
\usepackage{stmaryrd}
\usepackage{amsthm, thmtools}
\usepackage{framed}
\usepackage{nameref}
\usepackage[colorlinks,menucolor=blue,linkcolor=blue, citecolor=blue, urlcolor=blue]{hyperref} 
\usepackage[capitalise]{cleveref}% if problems, load cleveref last
\crefname{subsection}{Subsection}{Subsections}
\Crefname{subsection}{Subsection}{Subsections}
%\usepackage[retainorgcmds]{IEEEtrantools}
\usepackage{amssymb}
\usepackage[mathscr]{euscript}
\usepackage{esint}
\usepackage{esvect}
\usepackage{relsize}

\usepackage{tikz}
\usepackage{tikz-qtree}
\usepackage[framemethod=tikz]{mdframed}
\usetikzlibrary{cd}
\usetikzlibrary{matrix}
\usetikzlibrary{backgrounds}

%Packages for references, Indices

\usepackage{imakeidx}[intoc]
\usepackage[backend=biber]{biblatex}
\addbibresource{../refs.bib}
\addbibresource{dailyrefs.bib}

%Additionnal Packages

\usepackage{float}
\usepackage{cancel}

%Macors

\newcommand{\opname}{\operatorname}

%Theorems

\newtheorem{theorem}{Theorem}[section]
\newtheorem{proposition}{Proposition}[section]
\newtheorem{lemma}{Lemma}[section]
\newtheorem{corollary}{Corollary}[section]

\theoremstyle{definition}
\newtheorem{definition}{Definition}[section]
\newtheorem{example}{Example}[section]
\newtheorem{notation}{Notation}[section]

\theoremstyle{remark}
\newtheorem*{remark}{Remark}

\title{Log}
\author{Maxime Willaert}
\begin{document}

\maketitle

\tableofcontents

\section{Personal, Sunday 25/08/24}

The foundations of differential geometry.

\subsection{Encounters}

$C^0$ manifolds may not have $C^k$ structures for $k\geq 1$. If such structures exist, several inequivalent structures may arise (e.g. uncountably many inequivalent smooth structures on $\mathbb{R}^4$, or several smooth structures on $\mathbb{S}^7$). For $k\geq 1$, a $C^k$ manifold will have a unique (up to iso.) smooth structure, and even a unique (up to iso) analytic structure (based on the work of Whitney) \cite{Overflow:ManifoldAnalyticStructure,Stack:WhitneyAnalyticEmbedding}.

Properties of analytic manifolds. Smooth geometry relies on a handful of theorems from analysis, mainly the inverse mapping theorem and the standard existence uniqueness and regularity results from analysis. So it is useful to have their analogues in (real or complex) analytic geometry \cite{Narasimhan:AnalysisRealComplexMan,Bourbaki:VarietesDiffAnalytic}.

Defining tangent spaces as quotients of the ring of germs (an approach closer to algebraic geometry?) \cite{Warner:DiffMan}.

Categorical approach to manifolds (fibered manifolds, fiber bundles, transversal map and the pullback,...) \cite{KMS:NatDiffGeo}.

Infinite dimensional manifolds in the online supplement of \cite{Lee:ManDiffGeo}.

General topology \cite{Munkres:Top}. Different axioms, filters and nets (also related to logic and Stone spaces), convergence spaces. \cite{Wiki:AxiomTop,Wiki:FiltersTop,Wiki:ConvergenceSpace}. Categorical approach to topology (quotients, products, subspaces,...) \cite{Lee:IntTopMan}.

\subsection{Writing}

In \cite{personal:BasicsDiffGeo}, topology: basic definitions, maps between topological spaces ((local) homeos, open and closed maps, characterizing continuous maps), generating topologies (link with lattices \cite{personal:Lattices}, metrizable spaces and convergence.

\section{Work, Monday 26/08/24}

Planning of the defence of my master's thesis \cite{personalHand:MasterThesis}.

\subsection{Musings}

Elucidating the bundles associated to a principal $G$-bundle $P$, also called \textbf{$G$-structures}. They are those fiber bundles whose structure group can be reduced to $G$, i.e. they admit a trivializing atlas/cover with transition maps taking value in $G$ (automatically verifies cocycle condition). $P$ is then the principal bundle induced by this cocycle. The usual interpretation: $G$ is usually the Lie subgroup preserving a structure $\xi$ on $F$, a $G$-structure $E$ is then fiber bundle locally isomorphic to $U\times F$ with transition maps preserving the structure $\xi$ on $F$. So a $G$-structure is a bundle with standard fiber bundle equipped with a structure in the family of $\xi$ ($\xi$ serves as the standard model for this family).

Let $G$ act on $F$, so that we can construct the associated bundle $E:=P\times_GF=P\times F/G$. The connections on $E$ compatible with the $G$-structure of $E$ are those induced by principal connections on $P$ (what structure do they preserve exactly? -> usually the structure modelled by $\xi$, which will be required to be preserved by parallel transport).

The pullback seems to invtervene in this discussion in the following way: the trivial bundle $P\times F$ over $P$ equipped with the quotient map $q:P\times G\to P\times F/G=E$ is a pullback of $E\to M$ along the projection $\pi:P\to M$, i.e. $P\times F\simeq \pi^*E$. Conversely, given a local section $\sigma :U\to P$ of $P$, $E|_U\simeq \sigma^*(P\times F)$, the restriction of $E$ to $U$ is the pullback of $P\times F$ along $\sigma$. Given a principal connection on $P$, can the connection induced on $E$ be obtained via a connection on $P\times F$.

To make this discussion precise, I could maybe consult \cite{KMS:NatDiffGeo} and \cite{vakar:BundlesGauges}.

\subsection{Encounters}

In \cite{KMS:NatDiffGeo} the characterization of \textbf{initial submanifolds} (called \textbf{weakly embedded submanifolds} in \cite{Lee:IntSmMan}), the universal property of \textbf{submersions} (also called \textbf{fibered manifolds}).

\section{Work, Tuesday 27/08/24}

Taking inspiration on the chapter of \cite{KN:FundDiffGeo} dedicated to invariant affine connections, we try to give a description of invariant affine connections on homogeneous spaces that does not invoke principal connections. This is handwritten \cite{personalHand:Connections} (in the "probing" notebook).



\printbibliography

\end{document}