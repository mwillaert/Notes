\documentclass[a4paper,12pt,parskip=half*,chapterprefix=true,numbers=noendperiod]{scrreprt}

% LANGUAGES AND SYMBOLS
\usepackage[utf8]{inputenc}
\usepackage[T1]{fontenc}
\usepackage[english]{babel}
\usepackage{lmodern}
\usepackage{xparse}

% EVERYDAY PACKAGES
\usepackage{lipsum}
\usepackage[shortlabels]{enumitem}
\usepackage{verbatim}
\usepackage[normalem]{ulem}
\usepackage[dvipsnames]{xcolor}
\usepackage{csquotes}
\usepackage{pdfpages}
\usepackage{easy-todo}
\usepackage{setspace}
%\usepackage[all]{nowidow}

\usepackage{graphicx}
\usepackage{caption}
\usepackage{subcaption}

% MATHEMATICS
\usepackage{mathtools}
\usepackage{stmaryrd}
\usepackage{amsthm, thmtools}
\usepackage{framed}
\usepackage{nameref}
\usepackage[colorlinks,menucolor=blue,linkcolor=blue, citecolor=blue, urlcolor=blue]{hyperref} 
\usepackage[capitalise]{cleveref}% if problems, load cleveref last
\crefname{subsection}{Subsection}{Subsections}
\Crefname{subsection}{Subsection}{Subsections}
%\usepackage[retainorgcmds]{IEEEtrantools}
\usepackage{amssymb}
\usepackage[mathscr]{euscript}
\usepackage{esint}
\usepackage{esvect}
\usepackage{relsize}

\usepackage{tikz}
\usepackage{tikz-qtree}
\usepackage[framemethod=tikz]{mdframed}
\usetikzlibrary{cd}
\usetikzlibrary{matrix}
\usetikzlibrary{backgrounds}

%Packages for references, Indices

\usepackage{imakeidx}[intoc]
\usepackage[backend=biber]{biblatex}
\addbibresource{../refs.bib}
\addbibresource{dailyrefs.bib}

%Additionnal Packages

\usepackage{float}
\usepackage{cancel}

%Macors

\newcommand{\opname}{\operatorname}

%Theorems

\newtheorem{theorem}{Theorem}[section]
\newtheorem{proposition}{Proposition}[section]
\newtheorem{lemma}{Lemma}[section]
\newtheorem{corollary}{Corollary}[section]

\theoremstyle{definition}
\newtheorem{definition}{Definition}[section]
\newtheorem{example}{Example}[section]
\newtheorem{notation}{Notation}[section]

\theoremstyle{remark}
\newtheorem*{remark}{Remark}

\title{Log}
\author{Maxime Willaert}
\begin{document}

\maketitle

\tableofcontents

\section{Things to do}

\subsection{Differential geometry}

(Infinitesimal) automorphism of the tensor algebra $TV$ of a vector field $V$ (require compatibility with tensor product and contraction), are they always induced by an automorphism of $V$? Infinitesimal automorphisms=0-derivations compatible with contraction.
Extend it to the tensor algebra of a vector bundle, for that add locality condition (compatibility with restriction - morphism of sheaves). This is motivated by the bundle morphism $A_X:=\nabla_X-\mathcal{L}_X$ for an affine connection $\nabla$ (see \ref{tue27/08/24}).

The Bianchi identities from the exterior covariant derivative on vector bundles.

\section{Personal, Sunday 25/08/24}

The foundations of differential geometry.

\subsection{Encounters}

$C^0$ manifolds may not have $C^k$ structures for $k\geq 1$. If such structures exist, several inequivalent structures may arise (e.g. uncountably many inequivalent smooth structures on $\mathbb{R}^4$, or several smooth structures on $\mathbb{S}^7$). For $k\geq 1$, a $C^k$ manifold will have a unique (up to iso.) smooth structure, and even a unique (up to iso) analytic structure (based on the work of Whitney) \cite{Overflow:ManifoldAnalyticStructure,Stack:WhitneyAnalyticEmbedding}.

Properties of analytic manifolds. Smooth geometry relies on a handful of theorems from analysis, mainly the inverse mapping theorem and the standard existence uniqueness and regularity results for ODE's. So it is useful to have their analogues in (real or complex) analytic geometry \cite{Narasimhan:AnalysisRealComplexMan,Bourbaki:VarietesDiffAnalytic}.

Defining tangent spaces as quotients of the ring of germs (an approach closer to algebraic geometry?) \cite{Warner:DiffMan}.

Categorical approach to manifolds (fibered manifolds, fiber bundles, transversal map and the pullback,...) \cite{KMS:NatDiffGeo}.

Infinite dimensional manifolds in the online supplement of \cite{Lee:ManDiffGeo}.

General topology \cite{Munkres:Top}. Different axioms, filters and nets (also related to logic and Stone spaces), convergence spaces. \cite{Wiki:AxiomTop,Wiki:FiltersTop,Wiki:ConvergenceSpace}. Categorical approach to topology (quotients, products, subspaces,...) \cite{Lee:IntTopMan}.

\subsection{Writing}

In \cite{personal:BasicsDiffGeo}, topology: basic definitions, maps between topological spaces ((local) homeos, open and closed maps, characterizing continuous maps), generating topologies (link with lattices \cite{personal:Lattices}), metrizable spaces and convergence.

\section{Work, Monday 26/08/24}

Planning of the defence of my master's thesis \cite{personalHand:MasterThesis}.

\subsection{Musings}

Elucidating the bundles associated to a principal $G$-bundle $P$, also called \textbf{$G$-structures}. They are those fiber bundles whose structure group can be reduced to $G$, i.e. they admit a trivializing atlas/cover with transition maps taking value in $G$ (automatically verifies cocycle condition if $G$ acts faithfully on the standard fiber). $P$ is then the principal bundle induced by this cocycle. The usual interpretation: $G$ is usually the Lie subgroup preserving a structure $\xi$ on $F$, a $G$-structure $E$ is then a fiber bundle locally isomorphic to $U\times F$ with transition maps preserving the structure $\xi$ on $F$. So a $G$-structure is a bundle with standard fiber $F$ equipped with a structure in the family of $\xi$ ($\xi$ serves as the standard model for this family).

Let $G$ act on $F$, so that we can construct the associated bundle $E:=P\times_GF=P\times F/G$. The connections on $E$ compatible with the $G$-structure of $E$ are those induced by principal connections on $P$ (what structure do they preserve exactly? -> usually the structure modelled by $\xi$, which will be required to be preserved by parallel transport).

The pullback seems to intervene in this discussion in the following way: the trivial bundle $P\times F$ over $P$ equipped with the quotient map $q:P\times G\to P\times F/G=E$ is a pullback of $E\to M$ along the projection $\pi:P\to M$, i.e. $P\times F\simeq \pi^*E$. Conversely, given a local section $\sigma :U\to P$ of $P$, $E|_U\simeq \sigma^*(P\times F)$, the restriction of $E$ to $U$ is the pullback of $P\times F$ along $\sigma$. Given a principal connection on $P$, can the connection induced on $E$ be obtained via a connection on $P\times F$?

To make this discussion precise, I could maybe consult \cite{KMS:NatDiffGeo} and \cite{vakar:BundlesGauges}.

\subsection{Encounters}

In \cite{KMS:NatDiffGeo} the characterization of \textbf{initial submanifolds} (called \textbf{weakly embedded submanifolds} in \cite{Lee:IntSmMan}), the universal property of \textbf{submersions} (also called \textbf{fibered manifolds}).

\section{Work, Tuesday 27/08/24}\label{tue27/08/24}

Taking inspiration from the chapter of Kobayashi-Nomizu \cite{KN:FundDiffGeo} dedicated to invariant affine connections, we try to give a description of invariant affine connections on homogeneous spaces that does not invoke principal connections. This is handwritten \cite{personalHand:Connections} (in the "probing" notebook).

The way that an invariant affine connection on $M=K/H$ is introduced in Kobayashi-Nomizu is via a principal connection on the bundle of bases $L(M)$. This is achieved by specifying a $K$-invariant connection form $\omega$ on $L(M)$ which, according to the theorem of Wang on invariant principal connections, is encoded in a linear map
\begin{equation*}
	\Lambda:\mathfrak{k}\to gl_n(\mathbb{R})
\end{equation*}
(with $\mathfrak{k}$ the Lie algebra of $K$) after a choice of reference point $u_0\in L_0(M)$ (Note: $0=H\in K/H=M$ is the reference point chosen on $M$, i.e. $K$ acts transitively on $M$ and $H=\text{Stab}_K(0)$). To the linear map $\Lambda$ we then associate the unique $K$-invariant connection form $\omega$ for which
\begin{equation*}
	\omega_{u_0}(\hat{X})=\Lambda(X)
\end{equation*}
for all $X\in\mathfrak{k}$, where $\hat{X}\in\mathfrak{X}(L(M))$ denotes the horizontal lift of $\tilde{X}\in\mathfrak{X}(M)$, the fundamental vector field on $M$ associated to $X$ (whose flow is given by $e^{tX}$). This formula will not yield a well-defined connection form for any choice of $\Lambda$, the class of linear maps $\Lambda:\mathfrak{k}\to gl_n(\mathbb{R})$ for which this process works are precisely the ones satisfying the following two conditions:
\begin{enumerate}[(i)]
	\item $\Lambda(X)=\lambda(X)$ for all $X\in\mathfrak{h}$ (the Lie algebra of $H$), where $\lambda:\mathfrak{h}\to gl_n(\mathbb{R})$ is the Lie algebra morphism induced by the group morhpism $H\to GL_n(\mathbb{R}),h\to u^{-1}_0\circ h_{*0}\circ u_0$, which we also denote by $\lambda:H\to GL_n(\mathbb{R})$ (here $u_0$ is seen as a linear isomorphism $\mathbb{R}^n\to T_0M$). The group morphism $H\to GL(T_0M),h\to h_{*0}$ (to which $\lambda$ is conjugate) is called the \textbf{linear isotropy representation of $H$}.
	\item $\Lambda(ad(h)X)=ad(\lambda(h))\Lambda(X)=\lambda(h)\circ\Lambda(X)\circ\lambda(h)^{-1}$  for all $h\in H$, $X\in\mathfrak{k}$.
\end{enumerate}

As indicated in Kobayashi-Nomizu, the link between the linear map $\Lambda$ and the covariant derivative $\nabla$ of the invariant affine connection is the following:
\begin{equation*}
	u_0\circ\Lambda(X)\circ u^{-1}_0=A_{\tilde{X}}|_0
\end{equation*}
where $A_{\tilde{X}}$ is the endomorphism field $A_{\tilde{X}}:=\nabla_{\tilde{X}}-\mathcal{L}_{\tilde{X}}$. So a theorem of Wang for invariant affine connections that bypasses the formalism of principal connections can take the following form:

\begin{theorem}\label{thm:AffineWangAX}
 There is a one-to-one correspondence between invariant affine connections on $M=K/H$ and linear maps $\Lambda:\mathfrak{k}\to\text{End}(T_0M)$ satisfying
	\begin{enumerate}[(i)]
		\item $\Lambda(X)=\lambda(X)$ for all $X\in\mathfrak{h}$, where $\lambda:\mathfrak{h}\to\text{End}(T_0M)$ is the Lie algebra morphism induced by the linear isotropy representation $\lambda:H\to GL(T_0M),h\to h_{*0}$.
		\item $\Lambda(ad(h)X)=h_{*0}\circ \Lambda(X)\circ h^{-1}_{*0}$ for all $X\in\mathfrak{k}$, $h\in H$.
	\end{enumerate}
\end{theorem}

Once such a linear map $\Lambda$ is fixed, the corresponding affine connection is the unique one for which
\begin{equation*}
	A_{\tilde{X}}|_0=\Lambda(X)
\end{equation*}
for all $X\in\mathfrak{k}$. Note that an affine connection on a manifold $M$ is equivalent to the datum of an $\mathbb{R}$-linear map
\begin{equation*}
A:\mathfrak{X}(M)\to\Omega^1(M,TM),X\to A_X
\end{equation*}
such that $A_{fX}=X\otimes df+fA_X$ for all $X\in\mathfrak{X}(M)$ and $f\in C^{\infty}(M)$. The corresponding covariant derivative $\nabla$ is then the unique one satisfying
\begin{equation*}
A_X=\nabla_X-\mathcal{L}_X.
\end{equation*}
To recover $A$ from $\Lambda$ we start by obtaining a global expression for $A_{\tilde{X}}$ for all $X\in\mathfrak{k}$. This is obtained via the following formula
\begin{equation*}
	A_{\tilde{X}}|_{p=k0}=k_{*0}\Lambda[ad(k^{-1})X]k^{-1}_{*p}
\end{equation*}
(the appearance of $ad(k^{-1})$ is linked to the formula $k_*\tilde{X}=\widetilde{ad(k)X}$ for fundamental vector fields). This formula yields a well-defined endomorphism field $A_{\tilde{X}}$ thanks to condition (ii) in theorem \ref{thm:AffineWangAX}.

Once the fields $A_{\tilde{X}}$ are fixed for all $X\in\mathfrak{k}$, we recover $A_X$ for any vector field $X\in\mathfrak{X}(M)$ in the following way: if, over an open subset $U$ of $M$, we have $X|_U=\sum_{i=1}^af_i\tilde{X}_i|_U$ for $f_j\in C^{\infty}(U)$ and $X_j\in\mathfrak{k}$, we set
\begin{equation*}
A_X=\sum_{i=1}^a\left(\tilde{X}_i\otimes df_i+f_iA_{\tilde{X}_i}\right)
\end{equation*}
(over $U$). This works for the following two reasons:
\begin{itemize}
	\item Since $K$ acts transitively on $M$ (and by the rank theorem), for any point $p\in M$, there exists an open neighborhood $U$ of $p$ and a family $X_1,...,X_m$ in $\mathfrak{k}$ such that $(\tilde{X}_i)_{i=1}^m$ forms a local frame of $TM$ over $U$.
	\item By point (i) in theorem \ref{thm:AffineWangAX}, we can show that if
	\begin{equation*}
		\sum_{i=1}^af_i\tilde{X}_i|_U=\sum_{j=1}^bg_i\tilde{Y}_i|_U
	\end{equation*}
	for $f_i,g_j\in C^{\infty}(M)$ and $X_i,Y_j\in\mathfrak{k}$, then
	\begin{equation*}
	\sum_{i=1}^a\left(\tilde{X}_i\otimes df_i+f_iA_{\tilde{X}_i}\right)=\sum_{j=1}^b\left(\tilde{Y}_j\otimes dg_j+g_jA_{\tilde{X}_j}\right).
	\end{equation*}
	Key observations that allow us to prove this are the following: Firstly that, given a vector field $X\in\mathfrak{X}(M)$ that vanishes at some point $p\in M$, the Lie derivative by $X$ induces a linear map $\mathcal{L}^{(p)}_X:T_pM\to T_pM$ defined by
	\begin{equation}
	\mathcal{L}^{(p)}_XY_p:=\left.\frac{d}{dt}\right|_{t=0}\Phi^X_{-t*p}Y_p
	\end{equation}
	for all $Y_p\in T_pM$ (where $\Phi^X$ denotes the flow of $X$), meaning that for any $Y\in\mathfrak{X}(M)$, we have $[X,Y]_p=\mathcal{L}^{(p)}_XY_p$. Secondly, that when applied to a fundamental vector field $\tilde{Z}$ (for some $Z\in\mathfrak{k}$) with $\tilde{Z}_p=0$, this yields $\lambda^{(p)}(Z)=-\mathcal{L}^{(p)}_{\tilde{Z}}$, with $\lambda^{(p)}:\text{Lie}(\text{Stab}_K(p))\to\text{End}(T_pM)$, the Lie algebra morphism induced by the linear isotropy representation at $p$, $\lambda^{(p)}:\text{Stab}_K(p)\to\text{End}(T_pM),s\to s_{*p}$. Note that if $\tilde{Z}_p=0$ then $Z$ belongs to the Lie algebra of $\text{Stab}_K(p)$.
\end{itemize}

Our final objective is to arrive at a "linear algebraic" description of homogeneous symplectic spaces with invariant symplectic connections. Thus the next steps should be the following:
\begin{enumerate}
	\item Describe a $K$-invariant tensor in terms of its evaluation at $0\in M$.
	\item Describe the exterior derivative of a $K$-invariant form.
	\item Describe an invariant symplectic form.
	\item Describe the curvature and torsion of an invariant affine connection in terms of $\Lambda$.
	\item Describe the covariant derivative of an invariant tensor field by an invariant affine connection in terms of $\Lambda$ and the evaluation of the tensor field at $0$.
	\item lastly, give a description of invariant symplectic connections on homogeneous symplectic spaces.
\end{enumerate}

\section{Work, Thursday 12/09/24}\label{thur12/09/24}

(Handwritten notes "Mémoire/Connexions perso" \cite{personalHand:MasterThesis}) Encoding invariant tensor fields on the homogeneous space $M=K/H$, and dealing with the special case of covariant tensor fields, which we can pullback along the projection
\begin{equation*}
	\pi:K\to M=K/H,k\to ko
\end{equation*}
(with $o\in M=K/H$ a reference point, $H=\text{Stab}_K(o)$) which is equivariant.

\begin{theorem}[Invariant tensor fields on $M=K/H$]
	A tensor $T_o\in T^r_sT_oM$ is said to be $H$-invariant if for all $h\in H$ we have
	\begin{equation*}
		h_{*o}\cdot T_o=T_o.
	\end{equation*}
	Given such a tensor, there exists a unique invariant tensor field $T\in\Gamma(T^r_sM)$ whose evaluation at $o$ is $T_o$. Thus there is a 1:1 correspondence between invariant $(r,s)$ tensor fields and $H$-invariant elements of $T^r_sT_oM$.
\end{theorem}

Given an invariant covariant tensor field $\alpha$ on $M$, we can define $\pi^*\alpha$ over $K$, which is left-invariant. Because $\pi$ is a submersion, $\pi^*\alpha$ determines $\alpha$. The theorem below delineates the class of left-invariant covariant tensor fields on $K$ corresponding to invariant tensor fields on $M$.

\begin{theorem}[Invariant covariant tensor fields]
	A tensor $\bar{\alpha}_e\in T_k\mathfrak{k}$ is said to be $Ad(H)$-invariant if for all $h\in H$ we have
	\begin{equation*}
		Ad(h)^*\bar{\alpha}_e=\bar{\alpha}_e.
	\end{equation*}
	We say that $\bar{\alpha}_e$ annihilates $\mathfrak{h}$ if for all $X_j\in\mathfrak{k}$,
	\begin{equation*}
		\bar{\alpha}_e(X_1,...,X_k)=0
	\end{equation*}
	As soon as $X_j\in\mathfrak{h}$ for some $1\leq j\leq k$. Given such a tensor, there exists a unique invariant tensor field $\alpha\in\Gamma(T_kM)$ such that
	\begin{equation*}
		(\pi_{*e})^*\alpha_0=\bar{\alpha}_e.
	\end{equation*}	 
	Thus there is a 1:1 correspondence between invariant covariant $k$-tensor fields on $M$ and $Ad(H)$-invariant elements of $T_k\mathfrak{k}$.
\end{theorem}
\begin{remark}
	There is a 1:1 correspondence between left-invariant $(r,s)$ tensor fields on $K$ and elements of $T^r_s\mathfrak{k}$, since $Stab_k(e)=e$. So there exists a unique left-invariant covariant $k$-tensor field $\bar{\alpha}$ over $K$ whose evaluation at $e$ is $\bar{\alpha}_e$, and we have $\pi^*\alpha=\bar{\alpha}$.
\end{remark}
\begin{remark}
	The $Ad(H)$ invariance is linked to the fact that for any $h\in H$
	\begin{equation*}
		\pi\circ Ad(h)=h\circ\pi
	\end{equation*}
	This is because $\pi\circ R_h=\pi$ for all $h\in H$.
\end{remark}

\printbibliography

\end{document}